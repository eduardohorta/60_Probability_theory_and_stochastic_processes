\documentclass[12pt]{article}
\usepackage{pmmeta}
\pmcanonicalname{MemorylessRandomVariable}
\pmcreated{2013-03-22 14:39:49}
\pmmodified{2013-03-22 14:39:49}
\pmowner{CWoo}{3771}
\pmmodifier{CWoo}{3771}
\pmtitle{memoryless random variable}
\pmrecord{8}{36261}
\pmprivacy{1}
\pmauthor{CWoo}{3771}
\pmtype{Definition}
\pmcomment{trigger rebuild}
\pmclassification{msc}{60K05}
\pmclassification{msc}{60G07}
%\pmkeywords{memoryless}
%\pmkeywords{lack of memory}
\pmrelated{MarkovChain}

% this is the default PlanetMath preamble.  as your knowledge
% of TeX increases, you will probably want to edit this, but
% it should be fine as is for beginners.

% almost certainly you want these
\usepackage{amssymb,amscd}
\usepackage{amsmath}
\usepackage{amsfonts}

% used for TeXing text within eps files
%\usepackage{psfrag}
% need this for including graphics (\includegraphics)
%\usepackage{graphicx}
% for neatly defining theorems and propositions
%\usepackage{amsthm}
% making logically defined graphics
%%%\usepackage{xypic}

% there are many more packages, add them here as you need them

% define commands here
\begin{document}
A non-negative-valued random variable $X$ is \emph{memoryless} if 
$P(X>s+t\mid X>s)=P(X>t)$  for $s,t\ge0$.
\par
In words, given that a certain event did not occur during time period $s$ \emph{in the past}, the chance that an event will occur after an additional time period $t$ \emph{in the future} is the same as the chance that the event would occur after a time period $t$ from the beginning, regardless of how long or how short the time period $s$ is; the memory is \emph{erased}.  
\par
From the definition, we see that $$P(X>t)=P(X>s+t\mid X>s)=\frac{P(X>s+t\mbox{ and }X>s)}{P(X>s)}=\frac{P(X>s+t)}{P(X>s)},$$
so $P(X>s+t)=P(X>s)P(X>t)$ iff $X$ is memoryless.
\par
An example of a discrete memoryless random variable is the geometric random variable, since $P(X>s+t)=(1-p)^{s+t}=(1-p)^s(1-p)^t=P(X>s)P(X>t)$, where $p$ is the probability of $X$=success.  The exponential random variable is an example of a continuous memoryless random variable, which can be proved similarly with $1-p$ replaced by $e^{-\lambda}$.  In fact, the exponential random variable is the only continuous random variable having the memoryless property.
%%%%%
%%%%%
\end{document}
