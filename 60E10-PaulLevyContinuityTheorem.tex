\documentclass[12pt]{article}
\usepackage{pmmeta}
\pmcanonicalname{PaulLevyContinuityTheorem}
\pmcreated{2013-03-22 13:14:31}
\pmmodified{2013-03-22 13:14:31}
\pmowner{Koro}{127}
\pmmodifier{Koro}{127}
\pmtitle{Paul L\'evy continuity theorem}
\pmrecord{7}{33715}
\pmprivacy{1}
\pmauthor{Koro}{127}
\pmtype{Theorem}
\pmcomment{trigger rebuild}
\pmclassification{msc}{60E10}

\endmetadata

% this is the default PlanetMath preamble.  as your knowledge
% of TeX increases, you will probably want to edit this, but
% it should be fine as is for beginners.

% almost certainly you want these
\usepackage{amssymb}
\usepackage{amsmath}
\usepackage{amsfonts}

% used for TeXing text within eps files
%\usepackage{psfrag}
% need this for including graphics (\includegraphics)
%\usepackage{graphicx}
% for neatly defining theorems and propositions
%\usepackage{amsthm}
% making logically defined graphics
%%%\usepackage{xypic}

% there are many more packages, add them here as you need them

% define commands here
\begin{document}
\PMlinkescapeword{reciprocal}
Let $F_1,F_2,\dots$ be distribution functions with characteristic functions
$\varphi_1,\varphi_2,\dots$, respectively. If $\varphi_n$ converges pointwise
to a limit $\varphi$, and if $\varphi(t)$ is continuous at $t=0$, then
there exists a distribution function $F$ such that $F_n\rightarrow F$ \PMlinkname{weakly}{ConvergenceInDistribution}, and the characteristic function associated to $F$ is $\varphi$.

\textbf{Remark.} The reciprocal of this theorem is a \PMlinkescapetext{simple} corollary to the Helly-Bray theorem; hence $F_n\rightarrow F$ weakly if and only if $\varphi_n\rightarrow\varphi$ pointwise; but this theorem says something stronger than the sufficiency of that \PMlinkescapetext{proposition}: it says that the limit of a sequence of characteristic functions is a characteristic function whenever it is continuous at 0.
%%%%%
%%%%%
\end{document}
