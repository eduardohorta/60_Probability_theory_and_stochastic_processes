\documentclass[12pt]{article}
\usepackage{pmmeta}
\pmcanonicalname{BorelCantelliLemma}
\pmcreated{2013-03-22 13:13:18}
\pmmodified{2013-03-22 13:13:18}
\pmowner{Koro}{127}
\pmmodifier{Koro}{127}
\pmtitle{Borel-Cantelli lemma}
\pmrecord{7}{33688}
\pmprivacy{1}
\pmauthor{Koro}{127}
\pmtype{Theorem}
\pmcomment{trigger rebuild}
\pmclassification{msc}{60A99}

% this is the default PlanetMath preamble.  as your knowledge
% of TeX increases, you will probably want to edit this, but
% it should be fine as is for beginners.

% almost certainly you want these
\usepackage{amssymb}
\usepackage{amsmath}
\usepackage{amsfonts}

% used for TeXing text within eps files
%\usepackage{psfrag}
% need this for including graphics (\includegraphics)
%\usepackage{graphicx}
% for neatly defining theorems and propositions
%\usepackage{amsthm}
% making logically defined graphics
%%%\usepackage{xypic}

% there are many more packages, add them here as you need them

% define commands here
\begin{document}
Let $A_1, A_2,\dots$ be random events in a probability space.

\begin{enumerate}
\item If $\sum_{n=1}^\infty P(A_n)<\infty$, then 
$P(A_n \operatorname{i.o.}) = 0$;

\item If $A_1,A_2,\dots$ are independent, and $\sum_{n=1}^\infty P(A_n)=\infty$,
then $P(A_n \operatorname{i.o.})=1$
\end{enumerate}

where $A=[A_n \operatorname{i.o.}]$ represents the event ``$A_n$ happens for infinitely many values of $n$.'' 
Formally, $A = \limsup A_n$, which is a limit superior of sets.
%%%%%
%%%%%
\end{document}
