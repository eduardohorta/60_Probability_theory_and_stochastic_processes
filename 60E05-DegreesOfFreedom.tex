\documentclass[12pt]{article}
\usepackage{pmmeta}
\pmcanonicalname{DegreesOfFreedom}
\pmcreated{2013-03-22 17:10:53}
\pmmodified{2013-03-22 17:10:53}
\pmowner{Wkbj79}{1863}
\pmmodifier{Wkbj79}{1863}
\pmtitle{degrees of freedom}
\pmrecord{7}{39497}
\pmprivacy{1}
\pmauthor{Wkbj79}{1863}
\pmtype{Definition}
\pmcomment{trigger rebuild}
\pmclassification{msc}{60E05}
\pmclassification{msc}{62A01}

\usepackage{amssymb}
\usepackage{amsmath}
\usepackage{amsfonts}
\usepackage{pstricks}
\usepackage{psfrag}
\usepackage{graphicx}
\usepackage{amsthm}
%%\usepackage{xypic}

\begin{document}
\PMlinkescapeword{distribution}

The number of \emph{degrees of freedom} of a \PMlinkname{distribution}{Distribution3} is the number of parameters that are used to \PMlinkescapetext{calculate} the value of the distribution and are either known or assumed to be independent.

The following use degrees of freedom:

\begin{itemize}
\item chi-squared random variable
\item chi-squared statistic
\item F distribution
\item non-central chi-squared random variable
\item t distribution
\end{itemize}

{\bf Remark:}
The number of \emph{degrees of freedom} has also a physical meaning
related to the possible directions of motion of the system.
%%%%%
%%%%%
\end{document}
