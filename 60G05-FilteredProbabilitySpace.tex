\documentclass[12pt]{article}
\usepackage{pmmeta}
\pmcanonicalname{FilteredProbabilitySpace}
\pmcreated{2013-03-22 18:36:51}
\pmmodified{2013-03-22 18:36:51}
\pmowner{gel}{22282}
\pmmodifier{gel}{22282}
\pmtitle{filtered probability space}
\pmrecord{5}{41348}
\pmprivacy{1}
\pmauthor{gel}{22282}
\pmtype{Definition}
\pmcomment{trigger rebuild}
\pmclassification{msc}{60G05}
%\pmkeywords{probability space}
%\pmkeywords{filtration}
\pmrelated{FiltrationOfSigmaAlgebras}
\pmdefines{stochastic basis}
\pmdefines{usual conditions}
\pmdefines{usual hypotheses}

% almost certainly you want these
\usepackage{amssymb}
\usepackage{amsmath}
\usepackage{amsfonts}

% used for TeXing text within eps files
%\usepackage{psfrag}
% need this for including graphics (\includegraphics)
%\usepackage{graphicx}
% for neatly defining theorems and propositions
\usepackage{amsthm}
% making logically defined graphics
%%%\usepackage{xypic}

% there are many more packages, add them here as you need them

% define commands here
\newtheorem*{theorem*}{Theorem}
\newtheorem*{lemma*}{Lemma}
\newtheorem*{corollary*}{Corollary}
\newtheorem*{definition*}{Definition}
\newtheorem{theorem}{Theorem}
\newtheorem{lemma}{Lemma}
\newtheorem{corollary}{Corollary}
\newtheorem{definition}{Definition}

\begin{document}
\PMlinkescapeword{index set}
\PMlinkescapeword{index}
\PMlinkescapeword{satisfy}
\PMlinkescapeword{complete}
\PMlinkescapeword{completion}
\PMlinkescapeword{types}
\PMlinkescapeword{necessary}
\PMlinkescapeword{decomposition}
\PMlinkescapeword{theorems}
\PMlinkescapeword{weaker}
\PMlinkescapeword{strong}
\PMlinkescapeword{section}
\PMlinkescapeword{filtration}
A filtered probability space, or \emph{stochastic basis}, $(\Omega,\mathcal{F},(\mathcal{F}_t)_{t\in T},\mathbb{P})$ consists of a probability space $(\Omega,\mathcal{F},\mathbb{P})$ and a \PMlinkname{filtration}{FiltrationOfSigmaAlgebras} $(\mathcal{F}_t)_{t\in T}$ contained in $\mathcal{F}$. Here, $T$ is the time index set, and is an ordered set  --- usually a subset of the real numbers --- such that $\mathcal{F}_s\subseteq\mathcal{F}_t$ for all $s<t$ in $T$.

Filtered probability spaces form the setting for defining and studying stochastic processes. A process $X_t$ with time index $t$ ranging over $T$ is said to be adapted if $X_t$ is an $\mathcal{F}_t$-measurable random variable for every $t$.

When the index set $T$ is an \PMlinkname{interval}{Interval} of the real numbers (i.e., continuous-time), it is often convenient to impose further conditions. In this case, the filtered probability space is said to satisfy the \emph{usual conditions} or \emph{usual hypotheses} if the following conditions are met.
\begin{itemize}
\item The probability space $(\Omega,\mathcal{F},\mathbb{P})$ is \PMlinkname{complete}{CompleteMeasure}.
\item The $\sigma$-algebras $\mathcal{F}_t$ contain all the sets in $\mathcal{F}$ of zero probability.
\item The filtration $\mathcal{F}_t$ is right-continuous. That is, for every non-maximal $t\in T$, the $\sigma$-algebra $\mathcal{F}_{t+}\equiv\bigcap_{s>t}\mathcal{F}_s$ is equal to $\mathcal{F}_t$.
\end{itemize}
Given any filtered probability space, it can always be enlarged by passing to the completion of the probability space, adding zero probability sets to $\mathcal{F}_t$, and by replacing $\mathcal{F}_t$ by $\mathcal{F}_{t+}$. This will then satisfy the usual conditions.
In fact, for many types of processes defined on a complete probability space, their natural filtration will already be right-continuous and the usual conditions met.
However, the process of completing the probability space depends on the specific probability measure $\mathbb{P}$ and in many situations, such as the study of Markov processes, it is necessary to study many different measures on the same space. A much weaker condition which can be used is that the $\sigma$-algebras $\mathcal{F}_t$ are universally complete, which is still strong enough to apply much of the `heavy machinery' of stochastic processes, such as the Doob-Meyer decomposition, section theorems, etc.

%%%%%
%%%%%
\end{document}
