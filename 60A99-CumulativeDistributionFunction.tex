\documentclass[12pt]{article}
\usepackage{pmmeta}
\pmcanonicalname{CumulativeDistributionFunction}
\pmcreated{2013-03-22 11:53:38}
\pmmodified{2013-03-22 11:53:38}
\pmowner{bbukh}{348}
\pmmodifier{bbukh}{348}
\pmtitle{cumulative distribution function}
\pmrecord{10}{30504}
\pmprivacy{1}
\pmauthor{bbukh}{348}
\pmtype{Definition}
\pmcomment{trigger rebuild}
\pmclassification{msc}{60A99}
\pmclassification{msc}{46L05}
\pmclassification{msc}{82-00}
\pmclassification{msc}{83-00}
\pmclassification{msc}{81-00}
%\pmkeywords{probability}
\pmrelated{DistributionFunction}
\pmrelated{DensityFunction}

\endmetadata

\usepackage{amssymb}
\usepackage{amsmath}
\usepackage{amsfonts}
%\usepackage{graphicx}
%%%%%\usepackage{xypic}
\DeclareMathOperator{\Prb}{Pr}
\begin{document}
Let $X$ be a random variable. Define $F_X\colon R \to [0,1] $ as
$F_X(x) = \Prb[X \leq x]$
for all $x$. The function $F_X(x)$ is called the \emph{cumulative distribution function} of $X$.

Every cumulative distribution function satisfies the following properties:
\begin{enumerate}
\item $\lim_{x \to -\infty}{F_X(x)}=0$  and  $\lim_{x \to +\infty}{F_X(x)}=1$,
\item $F_X$ is a monotonically nondecreasing function,
\item $F_X$ is continuous from the right,
\item $\Prb[a < X \leq b] = F_X(b) - F_X(a)$.
\end{enumerate}

If $X$ is a discrete random variable, then the cumulative distribution can be expressed as
$F_X(x) = \sum_{k\leq x} \Prb[X = k]$.

Similarly, if $X$ is a continuous random variable, then
$F_X(x) = \int_{-\infty}^{x} f_X(y) dy$ where $f_X$ is the density distribution function.


% THIS IS RATHER USELESS STUFF THAT WAS IN THE ENTRY BEFORE I ADOPTED IT
%Please note that $P[a <  X \leq b] = F_X(b) - F_X(a)$    for    $a<b$. Other %inequality cases are obtained from the right continuous property. These are: \\
%
%$P[a <  X < b] = \lim_{n \to \infty }{F_X(b-\frac{1}{n})} - F_X(a)$ \\
%$P[a \leq  X \leq b] = F_X(b) - lim_{n \to \infty }{F_X(a-\frac{1}{n})} $ \\
%$P[a \leq  X < b] = \lim_{n \to \infty }{F_X(b-\frac{1}{n})} - lim_{n \to \infty %}{F_X(a-\frac{1}{n})} $ \\
%
%\par
%\par
%
%Note that if $X$ is a continous random variable then $F_X(x)$ must be continous. %This implies that\\
%$P[X=x] = P[ x \leq X \leq x ] = F_X(x) - lim_{n \to \infty %}{F_X(x-\frac{1}{n})} = F_X(x) - F_X(x) = 0$   for ALL x.
%%%%%
%%%%%
%%%%%
%%%%%
\end{document}
