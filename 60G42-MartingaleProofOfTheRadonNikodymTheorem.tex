\documentclass[12pt]{article}
\usepackage{pmmeta}
\pmcanonicalname{MartingaleProofOfTheRadonNikodymTheorem}
\pmcreated{2013-03-22 18:34:10}
\pmmodified{2013-03-22 18:34:10}
\pmowner{gel}{22282}
\pmmodifier{gel}{22282}
\pmtitle{martingale proof of the Radon-Nikodym theorem}
\pmrecord{9}{41293}
\pmprivacy{1}
\pmauthor{gel}{22282}
\pmtype{Proof}
\pmcomment{trigger rebuild}
\pmclassification{msc}{60G42}
\pmclassification{msc}{28A15}
%\pmkeywords{measure}
%\pmkeywords{martingale}
\pmrelated{RadonNikodymTheorem}
\pmrelated{MartingaleConvergenceTheorem}

% this is the default PlanetMath preamble.  as your knowledge
% of TeX increases, you will probably want to edit this, but
% it should be fine as is for beginners.

% almost certainly you want these
\usepackage{amssymb}
\usepackage{amsmath}
\usepackage{amsfonts}

% used for TeXing text within eps files
%\usepackage{psfrag}
% need this for including graphics (\includegraphics)
%\usepackage{graphicx}
% for neatly defining theorems and propositions
\usepackage{amsthm}
% making logically defined graphics
%%%\usepackage{xypic}

% there are many more packages, add them here as you need them

% define commands here
\newtheorem*{theorem*}{Theorem}
\newtheorem*{lemma*}{Lemma}
\newtheorem*{corollary*}{Corollary}
\newtheorem{theorem}{Theorem}
\newtheorem{lemma}{Lemma}
\newtheorem{corollary}{Corollary}


\begin{document}
We apply the martingale convergence theorem to prove the Radon-Nikodym theorem, which states that if $\mu$ and $\nu$ are $\sigma$-finite measures on a measurable space $(\Omega,\mathcal{F})$ and $\nu$ is absolutely continuous with respect to $\mu$ then there exists a non-negative and measurable $f\colon\Omega\rightarrow\mathbb{R}$ such that $\nu(A)=\int_Af\,d\mu$ for all measurable sets $A$.

As \PMlinkname{any $\sigma$-finite measure is equivalent to a probability measure}{AnySigmaFiniteMeasureIsEquivalentToAProbabilityMeasure}, it is enough to prove the result in the case where $\mu$ and $\nu$ are probability measures. Furthermore, by the Jordan decomposition, the result generalizes to the case where $\nu$ is a signed measure. So, we just need to prove the following.

\begin{theorem*}[Radon-Nikodym]
Let $\mathbb{P}$ and $\mathbb{Q}$ be probability measures on the measurable space $(\Omega,\mathcal{F})$, such that $\mathbb{Q}$ is absolutely continuous with respect to $\mathbb{P}$.
Then, there exists a non-negative random variable $X$ such that $\mathbb{E}_\mathbb{P}[X]=1$ and $\mathbb{Q}(A)=\mathbb{E}_\mathbb{P}[1_AX]$ for every $A\in\mathcal{F}$.
\end{theorem*}

Here, $X$ is called the Radon-Nikodym derivative of $\mathbb{Q}$ with respect to $\mathbb{P}$.

More generally, for any sub-$\sigma$-algebra $\mathcal{G}$ of $\mathcal{F}$ we can restrict the measures $\mathbb{P}$ and $\mathbb{Q}$ to $\mathcal{G}$ and ask if the Radon-Nikodym derivative of $\mathbb{Q}|_\mathcal{G}$ with respect to $\mathbb{P}|_\mathcal{G}$ exists. If it does we shall denote it by $X_\mathcal{G}$, which by definition is a non-negative $\mathcal{G}$-measurable random variable satisfying $\mathbb{Q}(A)=\mathbb{E}_\mathbb{P}[1_AX_\mathcal{G}]$ for all $A\in\mathcal{G}$.

We note that if $X_\mathcal{G}$ does exist, then it is uniquely defined ($\mathbb{P}$-almost everywhere). Suppose that $\tilde X_{\mathcal{G}}$ also satisfied the required properties, then
\begin{equation*}\begin{split}
\mathbb{E}_{\mathbb{P}}[\max(X_{\mathcal{G}}-\tilde X_{\mathcal{G}},0)]
&=\mathbb{E}_{\mathbb{P}}[X_\mathcal{G}1_{\left\{X_{\mathcal{G}}>\tilde X_{\mathcal{G}}\right\}}]-\mathbb{E}_{\mathbb{P}}[\tilde X_\mathcal{G}1_{\left\{X_{\mathcal{G}}>\tilde X_{\mathcal{G}}\right\}}]\\
&=\mathbb{Q}(X_{\mathcal{G}}>\tilde X_{\mathcal{G}})-\mathbb{Q}(X_{\mathcal{G}}>\tilde X_{\mathcal{G}})=0
\end{split}\end{equation*}
so $X_\mathcal{G}\le \tilde X_\mathcal{G}$ almost surely. Similarly, $\tilde X_\mathcal{G}\le X_\mathcal{G}$ and therefore $\tilde X_\mathcal{G}= X_\mathcal{G}$ (almost surely).


First, the easy case. For a finite $\sigma$-algebra, the Radon-Nikodym derivative can be written out explicitly.

\begin{lemma}
If $\mathcal{G}$ is a finite sub-$\sigma$-algebra of $\mathcal{F}$ then the Radon-Nikodym derivative $X_\mathcal{G}$ exists.
\end{lemma}
\begin{proof}
Let $A_1,A_2,\ldots,A_n$ be the minimal non-empty elements of $\mathcal{G}$. These are pairwise disjoint subsets of $\Omega$ such that every set in $\mathcal{G}$ is a union of a subcollection of the $A_k$. Set
\begin{equation*}
X_\mathcal{G}=\sum_{k=1}^n \frac{\mathbb{Q}(A_k)}{\mathbb{P}(A_k)}1_{A_k}
\end{equation*}
Note that whenever $\mathbb{P}(A_k)=0$ then $\mathbb{Q}(A_k)=0$, and we adopt the convention that $\frac{0}{0}=0$. Clearly, $X_\mathcal{G}$ is $\mathcal{G}$-measurable, and
\begin{equation*}\begin{split}
\mathbb{E}_{\mathbb{P}}[1_{A_k}X_\mathcal{G}]
&=\frac{\mathbb{Q}(A_k)}{\mathbb{P}(A_k)}\mathbb{E}_{\mathbb{P}}[1_{A_k}]+\sum_{j\not=k}\frac{\mathbb{Q}(A_j)}{\mathbb{P}(A_j)}\mathbb{E}_{\mathbb{P}}[1_{A_k\cap A_j}]\\
&=\mathbb{Q}(A_k).
\end{split}\end{equation*}
Here, we have used $\mathbb{E}_{\mathbb{P}}[1_{A_k}]=\mathbb{P}(A_k)$ and $1_{A_k\cap A_j}=0$. By linearity, this equality remains true if both sides are replaced by any union of the $A_k$, and therefore $X_\mathcal{G}$ is the required Radon-Nikodym derivative.
\end{proof}

Next, martingale convergence is used to prove the existence of the Radon-Nikodym derivative in the case where the $\sigma$-algebra $\mathcal{G}$ is separable. By separable, we mean that there is a countable sequence of sets $A_1,A_2,\ldots$ generating $\mathcal{G}$. Note that if we let $\mathcal{G}_n$ be the $\sigma$-algebra generated by $A_1,A_2,\ldots,A_n$, then $\mathcal{G}_n$ is an increasing sequence of finite sub-$\sigma$-algebras such that $\bigcup_n\mathcal{G}_n$ generates $\mathcal{G}$. The following result is general enough to apply in many useful cases, such as with the Boral $\sigma$-algebra on $\mathbb{R}^n$.

\begin{lemma}\label{lem:1}
Let $\mathcal{G}$ be a separable sub-$\sigma$-algebra of $\mathcal{F}$. Then, the Radon-Nikodym derivative $X_\mathcal{G}$ exists.
If furthermore, $\mathcal{G}_n$ is an increasing sequence of finite $\sigma$-algebras satisfying $\mathcal{G}=\sigma(\bigcup_n\mathcal{G}_n)$ then $\mathbb{E}_\mathbb{P}[|X_{\mathcal{G}}-X_{\mathcal{G}_n}|]\rightarrow 0$ as $n\rightarrow\infty$.
\end{lemma}
\begin{proof}
Let us set $X_n\equiv X_{\mathcal{G}_n}$.
If $m<n$ then the conditional expectation $\mathbb{E}_{\mathbb{P}}[X_n\mid\mathcal{G}_m]$ is $\mathcal{G}_m$-measurable, and for every $A\in\mathcal{G}_m$,
\begin{equation*}
\mathbb{E}_\mathbb{P}\left[1_A\mathbb{E}_{\mathbb{P}}[X_n\mid\mathcal{G}_m]\right]
=\mathbb{E}_{\mathbb{P}}\left[1_AX_n\right]=\mathbb{Q}(A).
\end{equation*}
This equality just uses the definition of the conditional expectation and then the definition of $X_n$ as the Radon-Nikodym derivative restricted to $\mathcal{G}_n$. So, $\mathbb{E}_{\mathbb{P}}[X_n\mid\mathcal{G}_m]$ is the Radon-Nikodym derivative restricted to $\mathcal{G}_m$, and equals $X_m$ (almost-surely).

Therefore, $X_n$ is a martingale and the martingale convergence theorem implies that the limit
\begin{equation}\label{eq:1}
X_\mathcal{G}=\lim_{n\rightarrow\infty}X_n
\end{equation}
exists almost surely. We now show that the sequence $X_n$ is uniformly integrable. Choose any $\epsilon>0$. As $\mathbb{Q}$ is absolutely continuous with respect to $\mathbb{P}$, there exists a $\delta>0$ such that $\mathbb{Q}(A)<\epsilon$ whenever $\mathbb{P}(A)<\delta$. Using
\begin{equation*}
\mathbb{P}(X_n>K)=\mathbb{E}_{\mathbb{P}}[1_{\{X_n>K\}}]\le\mathbb{E}_\mathbb{P}\left[\frac{X_n}{K}\right]=\frac{1}{K}
\end{equation*}
we see that $\mathbb{P}(X_n>K)<\delta$ whenever $K>\delta^{-1}$ and, therefore, $\mathbb{Q}(X_n>K)<\epsilon$. So
\begin{equation*}
\mathbb{E}_{\mathbb{P}}[X_n1_{\{X_n>K\}}]=\mathbb{Q}(X_n>K)<\epsilon
\end{equation*}
for every $n$, showing that $X_n$ is a uniformly integrable sequence with respect to $\mathbb{P}$.
Therefore, convergence in (\ref{eq:1}) is in $L^1$, and $\mathbb{E}_{\mathbb{P}}[|X_n-X_\mathcal{G}|]\rightarrow 0$ as $n\rightarrow\infty$. So, for any $A\in\bigcup_n\mathcal{G}_n$,
\begin{equation}\label{eq:3}
\mathbb{E}_\mathbb{P}[X_\mathcal{G}1_A]=\lim_{m\rightarrow\infty}\mathbb{E}_\mathbb{P}[X_m1_A]=\mathbb{Q}(A).
\end{equation}
By linearity and the monotone convergence theorem, the collection of sets $A$ satisfying (\ref{eq:3}) is a Dynkin system containing the $\pi$-system $\bigcup_n\mathcal{G}_n$ so, by Dynkin's lemma, is satisfied for every $A\in\sigma(\bigcup_n\mathcal{G}_n)=\mathcal{G}$ and, by definition, $X_\mathcal{G}$ is the Radon-Nikodym derivative restricted to $\mathcal{G}$.
\end{proof}

Finally, by approximating by finite $\sigma$-algebras we can prove the Radon-Nikodym theorem for arbitrary inseparable $\sigma$-algebras $\mathcal{F}$.

\

\noindent{\bf Proof of the Radon-Nikodym theorem:}

First, we use contradiction to show that for any $\epsilon>0$ there exists a finite $\sigma$-algebra $\mathcal{G}\subseteq\mathcal{F}$ satisfying $\mathbb{E}_\mathbb{P}[|X_\mathcal{G}-X_\mathcal{H}|]<\epsilon$ for every finite $\sigma$-algebra $\mathcal{H}$ with $\mathcal{G}\subseteq\mathcal{H}\subseteq{F}$. If this were not the case, then by induction we could find an increasing sequence of finite sub-$\sigma$-algebras of $\mathcal{F}$ satisfying $\mathbb{E}_\mathbb{P}[|X_{\mathcal{G}_n}-X_{\mathcal{G}_m}|]\ge\epsilon$.
However, letting $\mathcal{G}=\sigma(\bigcup_n\mathcal{G}_n)$, Lemma \ref{lem:1} shows that $X_\mathcal{G}$ exists and
\begin{equation*}
\epsilon\le\lim_{n\rightarrow\infty}\mathbb{E}_\mathbb{P}[|X_{\mathcal{G}_n}-X_{\mathcal{G}_{n+1}}|]
\le\lim_{n\rightarrow\infty}\mathbb{E}_\mathbb{P}[|X_{\mathcal{G}_n}-X_{\mathcal{G}}|]+\lim_{n\rightarrow\infty}\mathbb{E}_\mathbb{P}[|X_{\mathcal{G}_{n+1}}-X_{\mathcal{G}}|]=0
\end{equation*}
--- a contradiction.

So, there exists a sequence of finite sub-$\sigma$-algebras $\mathcal{G}_n$ of $\mathcal{F}$ such that $\mathbb{E}_\mathbb{P}[|X_{\mathcal{G}_n}-X_\mathcal{H}|]<2^{-n}$ for every finite sub-$\sigma$-algebra $\mathcal{H}$ of $\mathcal{F}$ containing $\mathcal{G}_n$. Let $\mathcal{G}$ be the (separable) $\sigma$-algebra generated by $\bigcup_n\mathcal{G}_n$, and set $\mathcal{\tilde G}_n=\sigma(\bigcup_{k=1}^n\mathcal{G}_k)$.
By Lemma \ref{lem:1}, the Radon-Nikodym derivative restricted to $\mathcal{G}$, $X_\mathcal{G}$, exists, and we show that it is the required derivative of $\mathbb{Q}$ with respect to $\mathbb{P}$.

Choose any set $A\in\mathcal{F}$ and let $\mathcal{H}_n$ be the (finite) $\sigma$-algebra generated by $\mathcal{G}_n\cup\{A\}$. Then, $X_{\mathcal{H}_n}$ exists and satisfies $\mathbb{E}_{\mathbb{P}}[X_{\mathcal{H}_n}1_A]=\mathbb{Q}(A)$ and,
\begin{equation*}\begin{split}
\left|\mathbb{E}_\mathbb{P}[X_\mathcal{G}1_A]-\mathbb{Q}(A)\right|
&=\lim_{n\rightarrow\infty}\left|\mathbb{E}_\mathbb{P}[X_{\mathcal{\tilde G}_n}1_A]-\mathbb{Q}(A)\right|\\
&=\lim_{n\rightarrow\infty}\left|\mathbb{E}_\mathbb{P}[X_{\mathcal{\tilde G}_n}1_A]-\mathbb{E}_\mathbb{P}[X_{\mathcal{H}_n}1_A]\right|\\
&\le\lim_{n\rightarrow\infty}\mathbb{E}_\mathbb{P}[|X_{\mathcal{\tilde G}_n}-X_{\mathcal{G}_n}|]+\lim_{n\rightarrow\infty}\mathbb{E}_\mathbb{P}[|X_{\mathcal{H}_n}-X_{\mathcal{G}_n}|]\\
&\le\lim_{n\rightarrow\infty}(2^{-n}+2^{-n})=0.
\end{split}\end{equation*}
So, $\mathbb{E}_{\mathbb{P}}[X_\mathcal{G}1_A]=\mathbb{Q}(A)$ as required.


\begin{thebibliography}{9}
\bibitem{williams}
David Williams, \emph{Probability with martingales},
Cambridge Mathematical Textbooks, Cambridge University Press, 1991.
\end{thebibliography}

%%%%%
%%%%%
\end{document}
