\documentclass[12pt]{article}
\usepackage{pmmeta}
\pmcanonicalname{HoeffdingInequalityForBoundedIndependentRandomVariables}
\pmcreated{2013-03-22 17:46:02}
\pmmodified{2013-03-22 17:46:02}
\pmowner{kshum}{5987}
\pmmodifier{kshum}{5987}
\pmtitle{Hoeffding inequality for bounded independent random variables}
\pmrecord{8}{40222}
\pmprivacy{1}
\pmauthor{kshum}{5987}
\pmtype{Theorem}
\pmcomment{trigger rebuild}
\pmclassification{msc}{60E15}
\pmrelated{ChernoffCramerBound}
\pmdefines{Hoeffding's inequality}

\endmetadata

% this is the default PlanetMath preamble.  as your knowledge
% of TeX increases, you will probably want to edit this, but
% it should be fine as is for beginners.

% almost certainly you want these
\usepackage{amssymb}
\usepackage{amsmath}
\usepackage{amsfonts}

% used for TeXing text within eps files
%\usepackage{psfrag}
% need this for including graphics (\includegraphics)
%\usepackage{graphicx}
% for neatly defining theorems and propositions
%\usepackage{amsthm}
% making logically defined graphics
%%%\usepackage{xypic}

% there are many more packages, add them here as you need them

% define commands here

\begin{document}
Let $X_1$, $X_2, \ldots, X_n$ be independent random variables, such that $\Pr(a_k \leq X_k \leq b_k)= 1$ for all $k$, where $a_k$ and $b_k$ are constant, $a_k<b_k$. Let $S_n$ be the sum $X_1+\ldots+X_n$. Then
\[
\Pr(S_n - E[S_n] > \epsilon) \leq \exp\Big( - \frac{2 \epsilon^2}{\sum_{k=1}^n(b_k-a_k)^2} \Big),
\]
\[
\Pr(|S_n - E[S_n]| > \epsilon) \leq 2 \exp\Big( - \frac{2 \epsilon^2}{\sum_{k=1}^n(b_k-a_k)^2} \Big).
\]



\begin{thebibliography}{1}
\bibitem{ph} W. Hoeffding, ``Probability inequalities for sums of bounded random variables'',  \emph{J. Amer. Statist. Assoc.}, vol. 58, pp.13-30, 1963. 

\end{thebibliography}

%%%%%
%%%%%
\end{document}
