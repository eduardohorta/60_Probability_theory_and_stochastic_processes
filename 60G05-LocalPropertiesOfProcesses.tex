\documentclass[12pt]{article}
\usepackage{pmmeta}
\pmcanonicalname{LocalPropertiesOfProcesses}
\pmcreated{2013-03-22 18:38:50}
\pmmodified{2013-03-22 18:38:50}
\pmowner{gel}{22282}
\pmmodifier{gel}{22282}
\pmtitle{local properties of processes}
\pmrecord{5}{41389}
\pmprivacy{1}
\pmauthor{gel}{22282}
\pmtype{Definition}
\pmcomment{trigger rebuild}
\pmclassification{msc}{60G05}
\pmclassification{msc}{60G40}
\pmclassification{msc}{60G48}
%\pmkeywords{stochastic process}
%\pmkeywords{stopping time}
%\pmkeywords{stopped process}
\pmrelated{LocalMartingale}
\pmdefines{local property}
\pmdefines{local submartingale}
\pmdefines{local martingale}
\pmdefines{locally integrable process}
\pmdefines{locally bounded process}
\pmdefines{prelocalization}
\pmdefines{prelocal property}

% almost certainly you want these
\usepackage{amssymb}
\usepackage{amsmath}
\usepackage{amsfonts}

% used for TeXing text within eps files
%\usepackage{psfrag}
% need this for including graphics (\includegraphics)
%\usepackage{graphicx}
% for neatly defining theorems and propositions
\usepackage{amsthm}
% making logically defined graphics
%%%\usepackage{xypic}

% there are many more packages, add them here as you need them

% define commands here
\newtheorem*{theorem*}{Theorem}
\newtheorem*{lemma*}{Lemma}
\newtheorem*{corollary*}{Corollary}
\newtheorem*{definition*}{Definition}
\newtheorem{theorem}{Theorem}
\newtheorem{lemma}{Lemma}
\newtheorem{corollary}{Corollary}
\newtheorem{definition}{Definition}

\begin{document}
\PMlinkescapeword{index set}
\PMlinkescapeword{constant}
\PMlinkescapeword{localization}
\PMlinkescapeword{theory}
\PMlinkescapeword{locally bounded}
\PMlinkescapeword{satisfy}
\PMlinkescapeword{left limits}

Many properties of stochastic processes, such as the martingale property, can be generalized to a corresponding local property. The local properties can be more useful than the original property because they are often preserved under certain transformations of processes, such as random time changes and stochastic integration.

Let  $(\Omega,\mathcal{F},(\mathcal{F}_t)_{t\in\mathbb{T}},\mathbb{P})$ be a filtered probability space and $\pi$ be a property of stochastic processes with time index set $\mathbb{T}\subseteq\mathbb{R}$.
The property $\pi$ is said to hold \emph{locally} for a process $X$ if there exists a sequence of stopping times $(\tau_n)_{n\in\mathbb{Z}_+}$ taking values in $\mathbb{T}\cup\{\infty\}$ and almost surely increasing to infinity, such that the stopped processes $X^{\tau_n}$ have property $\pi$ for each $n$.

Often, the index set $\mathbb{T}$ has a minimal element $t_0$, in which case it is convenient to extend the concept of localization slightly so that $\pi$ holds locally if there is a sequence of stopping times $\tau_n$ almost surely increasing to infinity and such that $1_{\{\tau_n> t_0\}}X^{\tau_n}$ have property $\pi$.

The property of locally satisfying $\pi$ is often denoted as $\pi_{\textrm{loc}}$. Similarly, if $\pi$ is a class of processes then the processes which are locally in $\pi$ is denoted by $\pi_{\textrm{loc}}$.
Letting $\tau_n$ be the stopping times taking the constant value $\infty$ shows that every process in $\pi$ is also locally in $\pi$, so $\pi\subseteq\pi_{\textrm{loc}}$.

In most cases where localization is used, such as with the class of right-continuous martingales, for any process $X$ in $\pi$ and stopping time $\tau$ then $1_{\{\tau>t_0\}}X^\tau$ is also in $\pi$. If this is the case then it is easily shown that a process is locally in $\pi_{\textrm{loc}}$ if and only if it is locally in $\pi$. So, $(\pi_{\textrm{loc}})_{\textrm{loc}}=\pi_{\textrm{loc}}$.

Examples of commonly used local properties are as follows.
\begin{enumerate}
\item A process $X$ is said to be a \emph{local martingale} if it is locally a right-continuous martingale. That is, if there is a sequence of stopping times $\tau_n$ almost surely increasing to infinity and such that $1_{\{\tau_n>t_0\}}X_{\tau_n\wedge t}$ is integrable and,
\begin{equation*}
1_{\{\tau_n>t_0\}}X_{\tau_n\wedge s}=\mathbb{E}[1_{\{\tau_n>t_0\}}X_{\tau_n\wedge t}\mid\mathcal{F}_s]
\end{equation*}
for all $s<t\in\mathbb{T}$.
In the discrete-time case where $\mathbb{T}=\mathbb{Z}_+$ then it can be shown that a local martingale $X$ is a martingale if and only if $\mathbb{E}[|X_t|]<\infty$ for every $t\in\mathbb{Z}_+$.
More generally, in continuous-time where $\mathbb{T}$ is an interval of the real numbers, then the stronger property that
\begin{equation*}
\left\{X_{\tau}:\tau\le t\textrm{ is a stopping time}\right\}
\end{equation*}
is uniformly integrable for every $t\in\mathbb{T}$ gives a necessary and sufficient condition for a local martingale to be a martingale.

Local martingales form a very important class of processes in the theory of stochastic calculus. This is because the local martingale property is preserved by the stochastic integral, but the martingale property is not.
Examples of local martingales which are not proper martingales are given by solutions to the stochastic differential equation
\begin{equation*}
dX = X^{\alpha}\,dW
\end{equation*}
where $X$ is a nonnegative process, $W$ is a Brownian motion and $\alpha>1$ is a fixed real number.

An alternative definition of local martingales which is sometimes used requires $X^{\tau_n}$ to be a martingale for each $n$. This definition is slightly more restrictive, and is equivalent to the definition given above together with the condition that $X_{t_0}$ must be integrable.

\item A \emph{local submartingale} (resp. \emph{local supermartingale}) is a right-continuous process which is locally a submartingale (resp. supermartingale). A local submartingale can be shown to be a submartingale if and only if $X_{t_0}$ is integrable and the set $\{X_\tau\vee 0:\tau\le t\textrm{ is a stopping time}\}$ is locally integrable for every $t\in\mathbb{T}$. In particular, every nonpositive local submartingale $X$ for which $X_{t_0}$ is integrable is a submartingale. Similarly every nonnegative supermartingale $X$ such that $X_{t_0}$ is integrable is a supermartingale.

\item An increasing and non-negative process $X$ is \emph{locally integrable} if it is locally an integrable process. That is, there is a sequence of stopping times $\tau_n$ increasing to infinity and such that $\mathbb{E}[1_{\{\tau_n>t_0\}}|X_{\tau_n\wedge t}|]<\infty$ for every $n\in\mathbb{Z}_+$ and $t\in\mathbb{T}$. By monotonicity of $X$, this is equivalent to $\mathbb{E}[1_{\{\tau_n>t_0\}}|X_{\tau_n}|]<\infty$. For example, the maximum process $X^*_t\equiv\sup_{s\le t}|X_s|$ of a local martingale $X$ is locally integrable.

\item A process $X$ is said to be \emph{locally bounded} if there is a sequence of stopping times $\tau_n$ almost surely increasing to infinity and such that $1_{\{\tau_n\ge t_0\}}X^{\tau_n}$ are uniformly bounded processes. For example, in discrete-time so $\mathbb{T}=\mathbb{Z}_+$, then every predictable process is locally bounded.
\end{enumerate}

Similarly, in continuous-time, if $\pi$ is a property of stochastic processes and $X$ is a stochastic process such that the left limits of $X_t$ with respect to $t$ exist everywhere, then $X$ is said to \emph{prelocally} satisfy $\pi$ if there is a sequence of stopping times $\tau_n$ almost surely increasing to infinity and such that the prestopped processes $1_{\{\tau_n>t_0\}}X^{\tau_n-}$ satisfy $\pi$.

%%%%%
%%%%%
\end{document}
