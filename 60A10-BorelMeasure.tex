\documentclass[12pt]{article}
\usepackage{pmmeta}
\pmcanonicalname{BorelMeasure}
\pmcreated{2013-03-22 17:34:00}
\pmmodified{2013-03-22 17:34:00}
\pmowner{asteroid}{17536}
\pmmodifier{asteroid}{17536}
\pmtitle{Borel measure}
\pmrecord{24}{39976}
\pmprivacy{1}
\pmauthor{asteroid}{17536}
\pmtype{Definition}
\pmcomment{trigger rebuild}
\pmclassification{msc}{60A10}
\pmclassification{msc}{28C15}
\pmclassification{msc}{28A12}
\pmclassification{msc}{28A10}
%\pmkeywords{Borel measure}
%\pmkeywords{Borel space}
%\pmkeywords{Borel sigma algebra}
\pmrelated{BorelSigmaAlgebra}
\pmrelated{RadonMeasure}
\pmrelated{BorelSpace}
\pmrelated{Measure}
\pmrelated{MeasurableSpace}
\pmrelated{BorelGroupoid}
\pmrelated{BorelMorphism}
\pmrelated{BorelGSpace}

\endmetadata

% this is the default PlanetMath preamble.  as your 

% almost certainly you want these
\usepackage{amssymb}
\usepackage{amsmath}
\usepackage{amsfonts}

% used for TeXing text within eps files
%\usepackage{psfrag}
% need this for including graphics (\includegraphics)
%\usepackage{graphicx}
% for neatly defining theorems and propositions
%\usepackage{amsthm}
% making logically defined graphics
%%%\usepackage{xypic}

% there are many more packages, add them here as you need 

% define commands here

\begin{document}
\PMlinkescapeword{Definition}

{\bf Definition 1 -} Let $X$ be a topological space and $\mathcal{B}$ be its \PMlinkname{Borel $\sigma$-algebra}{BorelSigmaAlgebra}.  A {\bf Borel measure} on $X$ is a measure on the measurable space $(X,\mathcal{B})$.

In the literature one can find other different definitions of Borel measure, like the following:

$\,$

\textbf{Definition 2 -} Let $X$ be a topological space and $\mathcal{B}$ be its Borel $\sigma$-algebra. A {\bf Borel measure} on $X$ is a measure $\mu$ on the measurable space $(X,\mathcal{B})$ such that $\mu (K) < \infty$ for all compact subsets $K \subset X$. (ref.\cite{MRB2k6}).

$\,$

\textbf{Definition 3 -} Let $X$ be a topological space and $\mathcal{B}$ be the $\sigma$-algebra generated by all compact sets of $X$. A {\bf Borel measure} on $X$ is a measure $\mu$ on the measurable space $(X,\mathcal{B})$ such that $\mu (K) < \infty$ for all compact subsets $K \subset X$.

$\,$

{\bf Definition 4 -} The \PMlinkname{restriction}{RestrictionOfAFunction} of the Lebesgue measure to the Borel $\sigma$-algebra of $\mathbb{R}^n$ is also sometimes called ``the'' Borel measure of $\mathbb{R}^n$.

$\,$

{\bf Remark -} Definitions $2$ and $3$ are technically different. For example, when constructing a Haar measure on a locally compact group one considers the $\sigma$-algebra generated by all compact subsets, instead of all closed (or open) sets.

\begin{thebibliography}{9}

\bibitem{MRB2k6}
M.R. Buneci. 2006., 
\PMlinkexternal{Groupoid C*-Algebras.}{http://www.utgjiu.ro/math/mbuneci/preprint/p0024.pdf}, 
{\em Surveys in Mathematics and its Applications}, Volume 1: 71--98.

\bibitem{AC79}
A. Connes.1979. Sur la th\'eorie noncommutative de l' integration, {\em Lecture Notes in
Math.},  Springer-Verlag, Berlin, {\bf 725}: 19-14.

\end{thebibliography}

%%%%%
%%%%%
\end{document}
