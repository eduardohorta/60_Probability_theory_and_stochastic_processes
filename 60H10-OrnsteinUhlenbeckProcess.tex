\documentclass[12pt]{article}
\usepackage{pmmeta}
\pmcanonicalname{OrnsteinUhlenbeckProcess}
\pmcreated{2013-03-22 17:19:26}
\pmmodified{2013-03-22 17:19:26}
\pmowner{stevecheng}{10074}
\pmmodifier{stevecheng}{10074}
\pmtitle{Ornstein-Uhlenbeck process}
\pmrecord{4}{39675}
\pmprivacy{1}
\pmauthor{stevecheng}{10074}
\pmtype{Definition}
\pmcomment{trigger rebuild}
\pmclassification{msc}{60H10}
\pmclassification{msc}{60-00}
\pmsynonym{Ornstein-Uhlenbeck equation}{OrnsteinUhlenbeckProcess}
%\pmkeywords{mean-reverting}

% The standard font packages
\usepackage{amssymb}
\usepackage{amsmath}
\usepackage{amsfonts}

% For neatly defining theorems and definitions
%\usepackage{amsthm}

% Including EPS/PDF graphics (\includegraphics)
%\usepackage{graphicx}

% Making matrix-based graphics
%%%\usepackage{xypic}

% Enumeration lists with different styles
%\usepackage{enumerate}

% Set up the theorem environments
%\newtheorem{thm}{Theorem}
%\newtheorem*{thm*}{Theorem}

\providecommand{\defnterm}[1]{\emph{#1}}

% The standard number systems
\newcommand{\complex}{\mathbb{C}}
\newcommand{\real}{\mathbb{R}}
\newcommand{\rat}{\mathbb{Q}}
\newcommand{\nat}{\mathbb{N}}
\newcommand{\intset}{\mathbb{Z}}

% Absolute values and norms
% Normal, wide, and big versions of the delimeters
\providecommand{\abs}[1]{\lvert#1\rvert}
\providecommand{\absW}[1]{\left\lvert#1\right\rvert}
\providecommand{\absB}[1]{\Bigl\lvert#1\Bigr\rvert}
\providecommand{\norm}[1]{\lVert#1\rVert}
\providecommand{\normW}[1]{\left\lVert#1\right\rVert}
\providecommand{\normB}[1]{\Bigl\lVert#1\Bigr\rVert}

% Differentiation operators
\providecommand{\od}[2]{\frac{d #1}{d #2}}
\providecommand{\pd}[2]{\frac{\partial #1}{\partial #2}}
\providecommand{\pdd}[2]{\frac{\partial^2 #1}{\partial #2}}
\providecommand{\ipd}[2]{\partial #1 / \partial #2}

% Differentials on integrals
\newcommand{\dx}{\, dx}
\newcommand{\dt}{\, dt}
\newcommand{\dmu}{\, d\mu}

% Inner products
\providecommand{\ip}[2]{\langle {#1}, {#2} \rangle}

% Calligraphic letters
\newcommand{\sF}{\mathcal{F}}
\newcommand{\sD}{\mathcal{D}}

% Standard spaces
\newcommand{\Hilb}{\mathcal{H}}
\newcommand{\Le}{\mathbf{L}}

% Operators and functions occassionally used in my articles
\DeclareMathOperator{\D}{D}
\DeclareMathOperator{\linspan}{span}
\DeclareMathOperator{\rank}{rank}
\DeclareMathOperator{\lindim}{dim}
\DeclareMathOperator{\sinc}{sinc}

% Probability stuff
\newcommand{\PP}{\mathbb{P}}
\newcommand{\E}{\mathbb{E}}

\begin{document}
\subsection*{Definition}

The \emph{Ornstein-Uhlenbeck} process is a stochastic process
that satisfies the following stochastic differential equation:
\begin{align}\label{eq:sde}
dX_t = \kappa ( \theta - X_t) \, dt + \sigma \, dW_t\,,
\end{align}
where $W_t$ is a standard Brownian motion on $t \in [0, \infty)$.

The constant parameters are:
\begin{itemize}
\item
$\kappa > 0$ is the rate of mean reversion;
\item
$\theta$ is the long-term mean of the process;
\item
$\sigma>0$ is the volatility or average magnitude, per square-root time,
of the random fluctuations that are modelled as Brownian motions.
\end{itemize}



\subsection*{Mean-reverting property}

If we ignore the random fluctuations in the process
due to $dW_t$, then we see that $X_t$ has an overall drift
towards a mean value $\theta$.
The process $X_t$ reverts to this mean exponentially, at rate $\kappa$,
with a magnitude in direct proportion to the distance
between the current value of $X_t$ and $\theta$.

This can be seen by looking at the solution to the 
ordinary differential equation $dx_t = \kappa (\theta - x) dt$
which is
\begin{align}\label{eq:ode}
\frac{\theta - x_t}{\theta-x_0} = e^{-\kappa(t-t_0) } \,,
\quad \text{ or } 
x_t = \theta + (x_0 - \theta) e^{-\kappa(t-t_0) }\,. 
\end{align}

For this reason, the Ornstein-Uhlenbeck process
is also called a \emph{mean-reverting process},
although the latter name applies to other types
of stochastic processes exhibiting the same property as well.


\subsection*{Solution}

The solution to the stochastic differential equation \eqref{eq:sde}
defining the Ornstein-Uhlenbeck process is, for any $0 \leq s \leq t$, is
\[
X_t = \theta + (X_s - \theta) e^{-\kappa (t-s)}
+ \sigma \int_s^t e^{-\kappa(t-u)} \, dW_u\,.
\]
where the integral on the right is the It\^o integral.

For any fixed $s$ and $t$, the random variable $X_t$, conditional
upon $X_s$, is normally distributed with 
\[
\text{mean} = \theta + (X_s - \theta) e^{-\kappa(t-s)}
\,,
\quad
\text{variance} = \frac{\sigma^2}{2\kappa} (1 - e^{-2\kappa (t-s)})\,.
\]
Observe that the mean of $X_t$ is exactly
the value derived heuristically
in the solution \eqref{eq:ode} of the ODE.

The Ornstein-Uhlenbeck process is a time-homogeneous It\^o diffusion.

\subsection*{Applications}

The Ornstein-Uhlenbeck process is widely used for modelling
biological processes such as neuronal response,
and in mathematical finance,
the modelling of the dynamics of interest rates
and volatilities of asset prices.

\begin{thebibliography}{6}
\bibitem{Jacobsen}
Martin Jacobsen. ``Laplace and the Origin of the Ornstein-Uhlenbeck Process''.
Bernoulli, Vol. 2, No. 3. (Sept. 1996), pp. 271 -- 286.
\bibitem{Oksendal}
Bernt \O{}ksendal.
\emph{Stochastic Differential Equations,
An Introduction with Applications}, 5th edition. Springer, 1998.
\bibitem{Shreve}
Steven E. Shreve. \emph{Stochastic Calculus for Finance II: 
Continuous-Time Models}. Springer, 2004.
\bibitem{Jaimungal}
Sebastian Jaimungal. Lecture notes for \emph{Pricing Theory}.
University of Toronto.
\bibitem{Rubisov}
Dmitri Rubisov. Lecture notes for \emph{Risk Management}.
University of Toronto.
\end{thebibliography}

%%%%%
%%%%%
\end{document}
