\documentclass[12pt]{article}
\usepackage{pmmeta}
\pmcanonicalname{DiscreteDensityFunction}
\pmcreated{2013-03-22 11:53:14}
\pmmodified{2013-03-22 11:53:14}
\pmowner{drini}{3}
\pmmodifier{drini}{3}
\pmtitle{discrete density function}
\pmrecord{16}{30486}
\pmprivacy{1}
\pmauthor{drini}{3}
\pmtype{Algorithm}
\pmcomment{trigger rebuild}
\pmclassification{msc}{60E99}
\pmclassification{msc}{00-02}
\pmsynonym{discrete probability function}{DiscreteDensityFunction}
\pmrelated{Distribution}

\usepackage{amssymb}
\usepackage{amsmath}
\usepackage{amsfonts}
\usepackage{graphicx}
%%%%\usepackage{xypic}
\begin{document}
\PMlinkescapeword{difference}
\PMlinkescapeword{properties}
\PMlinkescapeword{satisfy}
\PMlinkescapeword{syntax}

Let $X$ be a discrete random variable. The function $f_X\colon\mathbb{R} \to [0,1]$ defined as  $f_X(x)=P[X=x]$ is called the {\it discrete probability function} of $X$. Sometimes the syntax $p_X(x)$ is used, to mark the difference between this function and the continuous density function.

If $X$ has discrete density function $f_X(x)$, it is said that the random variable $X$ {\it has the distribution} or {\it is distributed} $f_X(x)$, and this fact is denoted as $X \sim f_X(x)$.

Discrete density functions are required to satisfy the following properties:
\begin{itemize}
\item $f_X(x) \geq 0$ for all $x$
\item $\sum_{x}f_X(x) = 1$
\end{itemize}
%%%%%
%%%%%
%%%%%
%%%%%
\end{document}
