\documentclass[12pt]{article}
\usepackage{pmmeta}
\pmcanonicalname{PredictableStoppingTime}
\pmcreated{2013-03-22 18:37:19}
\pmmodified{2013-03-22 18:37:19}
\pmowner{gel}{22282}
\pmmodifier{gel}{22282}
\pmtitle{predictable stopping time}
\pmrecord{6}{41357}
\pmprivacy{1}
\pmauthor{gel}{22282}
\pmtype{Definition}
\pmcomment{trigger rebuild}
\pmclassification{msc}{60G40}
\pmclassification{msc}{60G05}
\pmsynonym{predictable time}{PredictableStoppingTime}
\pmsynonym{previsible time}{PredictableStoppingTime}
\pmsynonym{previsible stopping time}{PredictableStoppingTime}
%\pmkeywords{filtration}
%\pmkeywords{stopping time}
\pmrelated{StoppingTime}
\pmrelated{PredictableProcess}

% almost certainly you want these
\usepackage{amssymb}
\usepackage{amsmath}
\usepackage{amsfonts}

% used for TeXing text within eps files
%\usepackage{psfrag}
% need this for including graphics (\includegraphics)
%\usepackage{graphicx}
% for neatly defining theorems and propositions
\usepackage{amsthm}
% making logically defined graphics
%%%\usepackage{xypic}

% there are many more packages, add them here as you need them

% define commands here
\newtheorem*{theorem*}{Theorem}
\newtheorem*{lemma*}{Lemma}
\newtheorem*{corollary*}{Corollary}
\newtheorem*{definition*}{Definition}
\newtheorem{theorem}{Theorem}
\newtheorem{lemma}{Lemma}
\newtheorem{corollary}{Corollary}
\newtheorem{definition}{Definition}

\begin{document}
\PMlinkescapeword{predictable}
\PMlinkescapeword{previsible}
\PMlinkescapeword{filtration}
\PMlinkescapeword{level}
\PMlinkescapeword{runnning}
\PMlinkescapeword{index set}
A predictable, or \emph{previsible} stopping time is a random time which is possible to predict just before the event.
Letting  $(\mathcal{F}_t)_{t\in\mathbb{R}_+}$ be a \PMlinkname{filtration}{FiltrationOfSigmaAlgebras} on a measurable space $(\Omega,\mathcal{F})$, then, a stopping time $\tau$ is predictable if there exists an increasing sequence of stopping times $\tau_n$ satisfying the following.
\begin{itemize}
\item $\tau_n<\tau$ whenever $\tau>0$.
\item $\tau_n\rightarrow\tau$ as $n\rightarrow\infty$.
\end{itemize}
The sequence $\tau_n$ is said to announce or foretell $\tau$.

For example, if $X$ is a continuous adapted process with $X_0=0$, such as Brownian motion, then the first time $\tau$ at which it hits a given level $K\not=0$ is a predictable stopping time.
In this case, if $\tau_n$ is the first time at which $X$ hits the level $K(1-1/n)$, then the sequence $\tau_n$ announces $\tau$.

On the other hand, if $X$ is a Poisson process then the first time $\tau$ at which it is nonzero is not predictable. To show this, suppose that $\tau_n<\tau$ are stopping times. The fact that $X_t-\lambda t$ is a martingale means that Doob's optional sampling theorem can be applied, giving $\mathbb{E}[X_{\tau_n}-\lambda\tau_n]=0$.
Then, $X_t=0$ for $t<\tau$ gives $\mathbb{E}[\tau_n]=0$. So, $\tau_n=0$ with probability one, and the sequence $\tau_n$ cannot announce $\tau$.


In discrete time, where the filtration $(\mathcal{F}_t)$ has time $t$ running over the index set $\mathbb{Z}_+$, then a stopping time is said to be predictable if $\{\tau\le t\}$ is $\mathcal{F}_{t-1}$-measurable for every time $t=1,2,\ldots$.

This can be generalized to an arbitrary index set $\mathbb{T}$, where a stopping time $\tau\colon\Omega\rightarrow\mathbb{T}\cup\{\infty\}$ is predictable if there exists an increasing sequence of stopping times $\tau_n\le\tau$ such that $\tau_n<\tau$ whenever $\tau$ is not equal to a minimal element of $\mathbb{T}$, and $\bigcap_n(\tau_n,\tau)$ contains no elements of $\mathbb{T}$.

%%%%%
%%%%%
\end{document}
