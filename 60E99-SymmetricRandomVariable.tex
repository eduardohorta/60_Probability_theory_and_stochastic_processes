\documentclass[12pt]{article}
\usepackage{pmmeta}
\pmcanonicalname{SymmetricRandomVariable}
\pmcreated{2013-03-22 16:25:45}
\pmmodified{2013-03-22 16:25:45}
\pmowner{CWoo}{3771}
\pmmodifier{CWoo}{3771}
\pmtitle{symmetric random variable}
\pmrecord{8}{38581}
\pmprivacy{1}
\pmauthor{CWoo}{3771}
\pmtype{Definition}
\pmcomment{trigger rebuild}
\pmclassification{msc}{60E99}
\pmclassification{msc}{60A99}
\pmdefines{symmetric distribution function}

\endmetadata

\usepackage{amssymb,amscd}
\usepackage{amsmath}
\usepackage{amsfonts}

% used for TeXing text within eps files
%\usepackage{psfrag}
% need this for including graphics (\includegraphics)
%\usepackage{graphicx}
% for neatly defining theorems and propositions
%\usepackage{amsthm}
% making logically defined graphics
%%\usepackage{xypic}
\usepackage{pst-plot}
\usepackage{psfrag}

% define commands here

\begin{document}
\PMlinkescapeword{symmetric}

Let $(\Omega,\mathcal{F},P)$ be a probability space and $X$ a real random variable defined on $\Omega$.  $X$ is said to be \emph{symmetric} if $-X$ has the same distribution function as $X$.  A distribution function $F:\mathbb{R}\to [0,1]$ is said to be \emph{symmetric} if it is the distribution function of a symmetric random variable.

\textbf{Remark}.  By definition, if a random variable $X$ is symmetric, then $E[X]$ exists ($<\infty$).  Furthermore, $E[X]=E[-X]=-E[X]$, so that $E[X]=0$.  Furthermore, let $F$ be the distribution function of $X$.  If $F$ is continuous at $x\in\mathbb{R}$, then $$F(-x)=P(X\le -x)=P(-X\le -x)=P(X\ge x)=1-P(X\le x)=1-F(x),$$ so that $F(x)+F(-x)=1$.  This also shows that if $X$ has a density function $f(x)$, then $f(x)=f(-x)$.

There are many examples of symmetric random variables, and the most common one being the normal random variables centered at $0$.  For any random variable $X$, then the difference $\Delta X$ of two independent random variables, identically distributed as $X$ is symmetric.
%%%%%
%%%%%
\end{document}
