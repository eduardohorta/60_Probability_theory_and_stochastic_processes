\documentclass[12pt]{article}
\usepackage{pmmeta}
\pmcanonicalname{RelationBetweenAlmostSurelyAbsolutelyBoundedRandomVariablesAndTheirAbsoluteMoments}
\pmcreated{2013-03-22 16:14:33}
\pmmodified{2013-03-22 16:14:33}
\pmowner{Andrea Ambrosio}{7332}
\pmmodifier{Andrea Ambrosio}{7332}
\pmtitle{relation between almost surely absolutely bounded random variables and their absolute moments}
\pmrecord{8}{38346}
\pmprivacy{1}
\pmauthor{Andrea Ambrosio}{7332}
\pmtype{Theorem}
\pmcomment{trigger rebuild}
\pmclassification{msc}{60A10}

\endmetadata

% this is the default PlanetMath preamble.  as your knowledge
% of TeX increases, you will probably want to edit this, but
% it should be fine as is for beginners.

% almost certainly you want these
\usepackage{amssymb}
\usepackage{amsmath}
\usepackage{amsfonts}

% used for TeXing text within eps files
%\usepackage{psfrag}
% need this for including graphics (\includegraphics)
%\usepackage{graphicx}
% for neatly defining theorems and propositions
\usepackage{amsthm}
% making logically defined graphics
%%%\usepackage{xypic}

% there are many more packages, add them here as you need them

% define commands here

\begin{document}
Let $\{\Omega ,E,P\}$ a probability space and let $X$ be a random
variable; then, the following are equivalent:

1) $\Pr \left\{ \left\vert X\right\vert \leq M\right\} =1$ \ i.e. $X$ is
absolutely bounded almost surely;

2) $E[\left\vert X\right\vert ^{k}]\leq M^{k}$ \ \ \ \ \ \ $\forall k\geq
1,k\in N$


\begin{proof}

1) $\Longrightarrow $ 2)

Let's define%
\[
F=\left\{ \omega \in \Omega :\left\vert X\left( \omega \right) \right\vert
>M\right\} ;
\]

Then by hypothesis 
\[
\Pr \left\{ \Omega \backslash F\right\} =1
\]

and
\[
\Pr \left\{ F\right\} =0.
\]

We have:
\begin{eqnarray*}
E[\left\vert X\right\vert ^{k}] &=&\int_{\Omega }\left\vert X\right\vert
^{k}dP \\
&=&\int_{\Omega \backslash F}\left\vert X\right\vert
^{k}dP+\int_{F}\left\vert X\right\vert ^{k}dP \\
&=&\int_{\Omega \backslash F}\left\vert X\right\vert ^{k}dP \\
&\leq &\int_{\Omega \backslash F}M^{k}dP \\
&=&M^{k}\Pr \left\{ \Omega \backslash F\right\} =M^{k}.
\end{eqnarray*}

2) $\Longrightarrow $ 1)

Let's define
\begin{eqnarray*}
F &=&\left\{ \omega \in \Omega :\left\vert X\left( \omega \right)
\right\vert >M\right\}  \\
F_{n} &=&\left\{ \omega \in \Omega :\left\vert X\left( \omega \right)
\right\vert >M+\frac{1}{n}\right\} \text{ \ }\forall n\geq 1.
\end{eqnarray*}

Then we have obviously $F_{n}\subseteq F_{n+1}$ (in fact, if $\omega \in
F_{n}\Longrightarrow \left\vert X\left( \omega \right) \right\vert >M+\frac{1%
}{n}>M+\frac{1}{n+1}\Longrightarrow \omega \in F_{n+1}$) and $%
F=\bigcup_{n=1}^{\infty }F_{n}$ (in fact, let $\omega \in F$; let $%
N=\left\lceil \frac{1}{\left\vert X\left( \omega \right) \right\vert -M}%
\right\rceil $; then $\left\vert X\left( \omega \right) \right\vert >M+\frac{%
1}{N}$, that is $\omega \in F_{N}$); this means that%
\[
F=\lim_{n\rightarrow \infty }F_{n}
\]

in the meaning of \PMlinkname{sets sequences convergence}{SequenceOfSetsConvergence}.

So \PMlinkname{the continuity from below property}{PropertiesForMeasure} of probability can be applied:
\[
\Pr \left\{ F\right\} =\Pr \left\{ \lim_{n\rightarrow \infty }F_{n}\right\}
=\lim_{n\rightarrow \infty }\Pr \left\{ F_{n}\right\} .
\]

Now, for any $k\geq 1$,
\begin{eqnarray*}
M^{k} &\geq &E\left[ \left\vert X\right\vert ^{k}\right] \\
&=&\int_{\Omega }\left\vert X\left( \omega \right) \right\vert ^{k}dP \\
&=&\int_{\Omega \backslash F_{n}}\left\vert X\left( \omega \right)
\right\vert ^{k}dP+\int_{F_{n}}\left\vert X\left( \omega \right) \right\vert
^{k}dP \\
&\geq &\int_{F_{n}}\left\vert X\left( \omega \right) \right\vert ^{k}dP \\
&\geq &\int_{F_{n}}\left( M+\frac{1}{n}\right) ^{k}dP \\
&=&\left( M+\frac{1}{n}\right) ^{k}\Pr \left\{ F_{n}\right\} .
\end{eqnarray*}

that is%
\[
\Pr \left\{ F_{n}\right\} \leq\left( \frac{M}{M+\frac{1}{n}}\right) ^{k}\text{
\ for any }k\geq 1
\]

so that the only acceptable value for $\Pr \left\{ F_{n}\right\} $ is
\[
\Pr \left\{ F_{n}\right\} =0
\]

whence the thesis.
\end{proof}

Acknowledgements: due to helpful discussions with Mathprof.
%%%%%
%%%%%
\end{document}
