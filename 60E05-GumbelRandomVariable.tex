\documentclass[12pt]{article}
\usepackage{pmmeta}
\pmcanonicalname{GumbelRandomVariable}
\pmcreated{2013-03-22 15:55:40}
\pmmodified{2013-03-22 15:55:40}
\pmowner{georgiosl}{7242}
\pmmodifier{georgiosl}{7242}
\pmtitle{Gumbel random variable}
\pmrecord{4}{37935}
\pmprivacy{1}
\pmauthor{georgiosl}{7242}
\pmtype{Definition}
\pmcomment{trigger rebuild}
\pmclassification{msc}{60E05}

% this is the default PlanetMath preamble.  as your knowledge
% of TeX increases, you will probably want to edit this, but
% it should be fine as is for beginners.

% almost certainly you want these
\usepackage{amssymb}
\usepackage{amsmath}
\usepackage{amsfonts}

% used for TeXing text within eps files
%\usepackage{psfrag}
% need this for including graphics (\includegraphics)
%\usepackage{graphicx}
% for neatly defining theorems and propositions
%\usepackage{amsthm}
% making logically defined graphics
%%%\usepackage{xypic}

% there are many more packages, add them here as you need them

% define commands here

\begin{document}
$X$ is a \emph{Gumbel random variable} if it has a probability density function, given by
$$f_X(x)=\frac{1}{\sigma}\exp(\frac{x-\mu}{\sigma})S(x)$$
where $-\infty <x<\infty $, $\mu$ is the \emph{location parameter}, $\sigma$ is the \emph{scale parameter}, and $S(x)$ is the survivor function, 
$S(x)=\exp[-\exp(\frac{x-\mu}{\sigma})]$ .

Notation for $X$ having a Gumbel distribution is $X\sim \mbox{Gum}(\mu,\sigma)$.

\textbf{\PMlinkescapetext{Properties}}:
Given a Gumbel distribution $X\sim \mbox{Gum}(\mu,\sigma)$:
\begin{enumerate}
\item
E[X]=$\mu-\gamma\sigma$, where $\gamma$ is the Euler's constant
\item
Var[X]=$\frac{\pi^2}{6}\sigma^2$
\end{enumerate}

\textbf{Remark.}
Nevertheless the interval $(-\infty,\infty)$ in which is defined, the Gumbel distribution is often used to model reliability or lifetime of products.

%%%%%
%%%%%
\end{document}
