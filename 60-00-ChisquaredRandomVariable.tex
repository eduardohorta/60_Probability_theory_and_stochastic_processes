\documentclass[12pt]{article}
\usepackage{pmmeta}
\pmcanonicalname{ChisquaredRandomVariable}
\pmcreated{2013-03-22 11:54:49}
\pmmodified{2013-03-22 11:54:49}
\pmowner{mathcam}{2727}
\pmmodifier{mathcam}{2727}
\pmtitle{chi-squared random variable}
\pmrecord{16}{30551}
\pmprivacy{1}
\pmauthor{mathcam}{2727}
\pmtype{Definition}
\pmcomment{trigger rebuild}
\pmclassification{msc}{60-00}
\pmclassification{msc}{11-00}
\pmclassification{msc}{20-01}
\pmclassification{msc}{20A05}
\pmsynonym{central chi-squared distribution}{ChisquaredRandomVariable}
\pmrelated{ChiSquaredStatistic}

\endmetadata

\usepackage{amssymb}
\usepackage{amsmath}
\usepackage{amsfonts}
\usepackage{graphicx}

\def\N{\mathbb{N}}
\def\Var{\operatorname{Var}}



\begin{document}
\PMlinkescapeword{degrees}
\PMlinkescapeword{intervals}
\PMlinkescapeword{represents}
\PMlinkescapeword{syntax}
\PMlinkescapeword{sum}
\PMlinkescapeword{squares}

A \emph{central chi-squared random variable} $X$ with $n>0$ degrees of freedom is given by the probability density function
\[
  f_X(x) = \frac{ (\frac{1}{2})^{\frac{n}{2}} } {\Gamma(\frac{n}{2})}
           x^{\frac{n}{2} - 1} e^{- \frac{1}{2} x} 
\]
for $x > 0$, where $\Gamma$ represents the gamma function.

The parameter $n$ is usually, but not always, an integer, in which case the distribution is that of the sum of the squares of a sequence of $n$ independent standard \PMlinkname{normal variables}{NormalRandomVariable} $X_1,X_2,\ldots,X_n$,
\begin{equation*}
X=X_1^2+X_2^2+\cdots+X_n^2.
\end{equation*}

Parameters: $n\in(0,\infty)$.

Syntax: $X\sim \chi_{(n)}^{2}$

\begin{figure}[h]
\centering
\includegraphics{chisquared}
\caption{Densities of the chi-squared distribution for different degrees of freedom.}
\end{figure}

Notes:

\begin{enumerate}

\item This distribution is very widely used in statistics, such as in hypothesis tests and confidence intervals.
\item The chi-squared distribution with $n$ degrees of freedom is a result of evaluating the gamma distribution with $\alpha = \frac{n}{2}$ and $\lambda = \frac{1}{2}$.
\item $E[X] = n$
\item $\Var[X] = 2n$
\item The moment generating function is
\begin{equation*}
M_X(t) = \left(1 - 2t\right)^{-\frac{n}{2}},
\end{equation*}
and is defined for all $t\in\mathbb{C}$ with \PMlinkname{real part}{Complex} less than $1/2$.
\item The sum of independent $\chi_{(m)}^2$ and $\chi_{(n)}^2$ random variables has the $\chi_{(m+n)}^2$ distribution.
\end{enumerate}

%%%%%
%%%%%
%%%%%
%%%%%
\end{document}
