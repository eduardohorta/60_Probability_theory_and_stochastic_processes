\documentclass[12pt]{article}
\usepackage{pmmeta}
\pmcanonicalname{MedianOfADistribution}
\pmcreated{2013-03-22 14:24:10}
\pmmodified{2013-03-22 14:24:10}
\pmowner{CWoo}{3771}
\pmmodifier{CWoo}{3771}
\pmtitle{median of a distribution}
\pmrecord{12}{35900}
\pmprivacy{1}
\pmauthor{CWoo}{3771}
\pmtype{Definition}
\pmcomment{trigger rebuild}
\pmclassification{msc}{60A99}
\pmclassification{msc}{62-07}
\pmsynonym{second quartile}{MedianOfADistribution}
\pmdefines{median}

\endmetadata

% this is the default PlanetMath preamble.  as your knowledge
% of TeX increases, you will probably want to edit this, but
% it should be fine as is for beginners.

% almost certainly you want these
\usepackage{amssymb,amscd}
\usepackage{amsmath}
\usepackage{amsfonts}

% used for TeXing text within eps files
%\usepackage{psfrag}
% need this for including graphics (\includegraphics)
%\usepackage{graphicx}
% for neatly defining theorems and propositions
%\usepackage{amsthm}
% making logically defined graphics
%%%\usepackage{xypic}

% there are many more packages, add them here as you need them

% define commands here
\begin{document}
Given a probability distribution (density) function $f_X(x)$ on $\Omega$ over a random variable $X$, with the associated probability measure $P$, a \emph{median} $m$ of $f_X$ is a real number such that 
\begin{enumerate}
\item
$P(X\leq m)\geq \frac{1}{2},$
\item
$P(X\geq m)\geq \frac{1}{2}.$
\end{enumerate}

The median is also known as the $50^{\text{th}}$-percentile or the second quartile.

\textbf{Examples:}
\begin{itemize}
\item
An example from a discrete distribution.  Let $\Omega=\mathbb{R}$.  Suppose the random variable $X$ has the following distribution: $P(X=0)=0.99$ and $P(X=1000)=0.01$.  Then we can easily see the median is 0.
\item
Another example from a discrete distribution.  Again, let $\Omega=\mathbb{R}$.  Suppose the random variable $X$ has distribution $P(X=0)=0.5$ and $P(X=1000)=0.5$.  Then we see that the median is not unique.  In fact, all real values in the interval $[0,1000]$ are medians.
\item
In practice, however, the median may be calculated as follows: if there are $N$ numeric data points, then by ordering the data values (either non-decreasingly or non-increasingly), 
\begin{enumerate}
\item
the $(\frac{N+1}{2})$-th data point is the median if $N$ is odd, and 
\item
the midpoint of the $(N-1)$th and the $(N+1)$th data points is the median if $N$ is even.
\end{enumerate}
\item
The median of a normal distribution (with mean $\mu$ and variance $\sigma^2$) is $\mu$.  In fact, for a normal distribution, mean = median = mode.
\item
The median of a uniform distribution in the interval $[a,b]$ is $(a+b)/2$.
\item
The median of a Cauchy distribution with location parameter t and scale parameter s is the location parameter.
\item
The median of an exponential distribution with location parameter $\mu$ and scale parameter $\beta$ is the scale parameter times the natural log of 2, $\beta\operatorname{ln}2$.
\item
The median of a Weibull distribution with shape parameter $\gamma$, location parameter $\mu$, and scale parameter $\alpha$ is $\alpha(\operatorname{ln}2)^{1/\gamma}+\mu$.
\end{itemize}
%%%%%
%%%%%
\end{document}
