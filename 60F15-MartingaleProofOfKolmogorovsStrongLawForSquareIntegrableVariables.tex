\documentclass[12pt]{article}
\usepackage{pmmeta}
\pmcanonicalname{MartingaleProofOfKolmogorovsStrongLawForSquareIntegrableVariables}
\pmcreated{2013-03-22 18:33:51}
\pmmodified{2013-03-22 18:33:51}
\pmowner{gel}{22282}
\pmmodifier{gel}{22282}
\pmtitle{martingale proof of Kolmogorov's strong law for square integrable variables}
\pmrecord{4}{41287}
\pmprivacy{1}
\pmauthor{gel}{22282}
\pmtype{Proof}
\pmcomment{trigger rebuild}
\pmclassification{msc}{60F15}
\pmclassification{msc}{60G42}
%\pmkeywords{martingale}
\pmrelated{MartingaleConvergenceTheorem}
\pmrelated{KolmogorovsStrongLawOfLargeNumbers}
\pmrelated{StrongLawOfLargeNumbers}
\pmrelated{ProofOfKolmogorovsStrongLawForIIDRandomVariables}

% this is the default PlanetMath preamble.  as your knowledge
% of TeX increases, you will probably want to edit this, but
% it should be fine as is for beginners.

% almost certainly you want these
\usepackage{amssymb}
\usepackage{amsmath}
\usepackage{amsfonts}

% used for TeXing text within eps files
%\usepackage{psfrag}
% need this for including graphics (\includegraphics)
%\usepackage{graphicx}
% for neatly defining theorems and propositions
\usepackage{amsthm}
% making logically defined graphics
%%%\usepackage{xypic}

% there are many more packages, add them here as you need them

% define commands here
\newtheorem*{theorem*}{Theorem}
\newtheorem*{lemma*}{Lemma}
\newtheorem*{corollary*}{Corollary}
\newtheorem{theorem}{Theorem}
\newtheorem{lemma}{Lemma}
\newtheorem{corollary}{Corollary}


\begin{document}
We apply the martingale convergence theorem to prove the following result.

\begin{theorem*}
Let $X_1,X_2,\ldots$ be independent random variables such that $\sum_n \operatorname{Var}[X_n]/n^2<\infty$. Then, setting
\begin{equation*}
S_n=\frac{1}{n}\sum_{k=1}^n(X_k-\mathbb{E}[X_k])
\end{equation*}
we have $S_n\rightarrow 0$ as $n\rightarrow\infty$, with probability one.
\end{theorem*}

To prove this, we start by constructing a martingale,
\begin{equation*}
M_n=\sum_{k=1}^n\frac{X_k-\mathbb{E}[X_k]}{k}.
\end{equation*}
If $\mathcal{F}_n$ is the \PMlinkname{$\sigma$-algebra}{SigmaAlgebra} generated by $X_1,\ldots X_n$ then
\begin{equation*}
\mathbb{E}[M_{n+1}\mid\mathcal{F}_n]=M_n+\frac{\mathbb{E}[X_{n+1}\mid\mathcal{F}_n]-\mathbb{E}[X_{n+1}]}{n+1}=M_n.
\end{equation*}
Here, the independence of $X_{n+1}$ and $\mathcal{F}_n$ has been used to imply that $\mathbb{E}[X_{n+1}\mid\mathcal{F}_n]=\mathbb{E}[X_{n+1}]$. So, $M$ is a martingale with respect to the filtration $(\mathcal{F}_n)_{n\in\mathbb{N}}$.

Also, by the independence of the $X_n$, the variance of $M_n$ is
\begin{equation*}
\operatorname{Var}[M_n]=\sum_{k=1}^n\operatorname{Var}[X_k/k]\le\sum_{k=1}^\infty\frac{\operatorname{Var}[X_k]}{k^2}<\infty.
\end{equation*}
So, the inequality $\mathbb{E}[|M_n|]\le\sqrt{\mathbb{E}[M_n^2]}=\sqrt{\operatorname{Var}[M_n]}$ shows that $M$ is an $L^1$-bounded martingale, and the martingale convergence theorem says that the limit $M_\infty=\lim_{n\rightarrow\infty}M_n$ exists and is finite, with probability one.

The strong law now follows from Kronecker's lemma, which states that for sequences of real numbers $x_1,x_2,\ldots$ and $0<b_1,b_2,\ldots$ such that $b_n$ strictly increases to infinity and $\sum_nx_n/b_n$ converges to a finite limit, then $b_n^{-1}\sum_{k=1}^nx_k$ tends to $0$ as $n\rightarrow\infty$. In our case, we take $x_n=X_n-\mathbb{E}[X_n]$ and $b_n=n$ to deduce that $n^{-1}\sum_{k=1}^n(X_k-\mathbb{E}[X_k])$ converges to zero with probability one.

\begin{thebibliography}{9}
\bibitem{williams}
David Williams, \emph{Probability with martingales},
Cambridge Mathematical Textbooks, Cambridge University Press, 1991.
\bibitem{kallenberg}
Olav Kallenberg, \emph{Foundations of modern probability}, Second edition. Probability and its Applications. Springer-Verlag, 2002.
\end{thebibliography}

%%%%%
%%%%%
\end{document}
