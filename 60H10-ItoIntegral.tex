\documentclass[12pt]{article}
\usepackage{pmmeta}
\pmcanonicalname{ItoIntegral}
\pmcreated{2013-03-22 16:13:29}
\pmmodified{2013-03-22 16:13:29}
\pmowner{stevecheng}{10074}
\pmmodifier{stevecheng}{10074}
\pmtitle{It\^o integral}
\pmrecord{21}{38323}
\pmprivacy{1}
\pmauthor{stevecheng}{10074}
\pmtype{Topic}
\pmcomment{trigger rebuild}
\pmclassification{msc}{60H10}
\pmsynonym{Ito integral}{ItoIntegral}
\pmsynonym{stochastic integral}{ItoIntegral}
\pmrelated{StochasticCalculusAndSDE}
\pmrelated{StochasticIntegration}
\pmdefines{It\^o integral}

\endmetadata

% The standard font packages
\usepackage{amssymb}
\usepackage{amsmath}
\usepackage{amsfonts}

% For neatly defining theorems and definitions
%\usepackage{amsthm}

% Including EPS/PDF graphics (\includegraphics)
%\usepackage{graphicx}

% Making matrix-based graphics
%%%\usepackage{xypic}

% Enumeration lists with different styles
%\usepackage{enumerate}

% Set up the theorem environments
%\newtheorem{thm}{Theorem}
%\newtheorem*{thm*}{Theorem}

\providecommand{\defnterm}[1]{\emph{#1}}

% The standard number systems
\newcommand{\complex}{\mathbb{C}}
\newcommand{\real}{\mathbb{R}}
\newcommand{\rat}{\mathbb{Q}}
\newcommand{\nat}{\mathbb{N}}
\newcommand{\intset}{\mathbb{Z}}

% Absolute values and norms
% Normal, wide, and big versions of the delimeters
\providecommand{\abs}[1]{\lvert#1\rvert}
\providecommand{\absW}[1]{\left\lvert#1\right\rvert}
\providecommand{\absB}[1]{\Bigl\lvert#1\Bigr\rvert}
\providecommand{\norm}[1]{\lVert#1\rVert}
\providecommand{\normW}[1]{\left\lVert#1\right\rVert}
\providecommand{\normB}[1]{\Bigl\lVert#1\Bigr\rVert}

% Differentiation operators
\providecommand{\od}[2]{\frac{d #1}{d #2}}
\providecommand{\pd}[2]{\frac{\partial #1}{\partial #2}}
\providecommand{\pdd}[2]{\frac{\partial^2 #1}{\partial #2}}
\providecommand{\ipd}[2]{\partial #1 / \partial #2}

% Differentials on integrals
\newcommand{\dx}{\, dx}
\newcommand{\dt}{\, dt}
\newcommand{\dmu}{\, d\mu}

% Inner products
\providecommand{\ip}[2]{\langle {#1}, {#2} \rangle}

% Calligraphic letters
\newcommand{\sF}{\mathcal{F}}
\newcommand{\sB}{\mathcal{B}}
\newcommand{\sD}{\mathcal{D}}

% Standard spaces
\newcommand{\Hilb}{\mathcal{H}}
\newcommand{\Le}{\mathbf{L}}

% Operators and functions occassionally used in my articles
\DeclareMathOperator{\D}{D}
\DeclareMathOperator{\linspan}{span}
\DeclareMathOperator{\rank}{rank}
\DeclareMathOperator{\lindim}{dim}
\DeclareMathOperator{\sinc}{sinc}

% Probability stuff
\newcommand{\PP}{\mathbb{P}}
\newcommand{\E}{\mathbb{E}}

\begin{document}
\PMlinkescapeword{clear}
\PMlinkescapeword{scaling}
\PMlinkescapeword{factor}
\PMlinkescapeword{terms}
\PMlinkescapeword{right hand side}
\PMlinkescapeword{infinite}
\PMlinkescapeword{increasing}
\PMlinkescapeword{solution}
\PMlinkescapeword{variables}
\PMlinkescapeword{identity}
\PMlinkescapeword{equation}
\PMlinkescapeword{finite}
\PMlinkescapeword{restriction}
\PMlinkescapeword{applications}
\PMlinkescapeword{even}
\PMlinkescapeword{infinity}
\PMlinkescapeword{bounded}
\PMlinkescapeword{solutions}
\PMlinkescapeword{necessary}

\section{The need for a stochastic integral}

Suppose that we wish to model a system which is continuously subject to random shocks. Then, we may want to consider a differential equation of the form
\begin{equation}\label{eq:1}
\frac{dX_t}{dt}=b(X_t)+a(X_t)\xi_t.
\end{equation}
Here, $X_t$ is the stochastic process describing the state of the system at each time $t\ge 0$, $b$ describes the behavior in the absence of any random shocks and $a$ is the sensitivity to the the random  noise $\xi_t$. The random variables $\xi_t$ are assumed to be independent and identically distributed with mean zero for each $t\ge 0$, and defined on an underlying probability space $(\Omega,\mathcal{F},\mathbb{P})$.

The first problem is that it is not clear how to define $\xi$ continuously in time such that (\ref{eq:1}) can be solved. Let $W$ be the integral
\begin{equation*}
W_t=\int_0^t\xi_s\,ds.
\end{equation*}
Then, $W_t$ should be continuous in $t$ and, for each $h>0$, the sequence of increments $(W_{(n+1)h}-W_{nh})_{n=0,1,\ldots}$ should be independent and identically distributed random variables. A possible candidate for $W$ is Brownian motion, and it can be shown that, up to a constant scaling factor, this is the only possibility.
However, Brownian motion is nowhere differentiable and $\xi_t=dW_t/dt$ cannot exist. We therefore re-express (\ref{eq:1}) in terms of $W$ by integrating it
\begin{equation}\label{eq:2}
X_t=X_0+\int_0^tb(X_s)\,ds+\int_0^ta(X_s)\,dW_s.
\end{equation}
The idea of It\^o integration is to give meaning to the final integral on the right hand side of (\ref{eq:2}).
Defining the integral with respect to piecewise constant functions of the form
\begin{equation}\label{eq:3}
\alpha_t = \sum_{k=1}^n c_k 1_{\{t_{k-1}<t\le t_k\}}
\end{equation}
is easy. Here, $0\le t_0\le t_1\le\cdots\le t_n$ is an increasing sequence of times and $c_k$ are random variables. Then, the integral of $\alpha$ with respect to $W$ is
\begin{equation}\label{eq:4}
\int_0^t\alpha_s\,dW_s = \sum_{k=1}^n c_k \left(W_{t_{k}\wedge t}-W_{t_{k-1}\wedge t}\right).
\end{equation}
We could attempt to define the integral with respect to $W$ on the right hand side of (\ref{eq:2}) by approximating $a(X_t)$ by such piecewise constant functions. However, the sample paths of Brownian motion are of infinite total variation over all intervals. This means that if we let $\alpha^n$ be
\begin{equation*}
\alpha^n_t = \sum_{k=1}^n \operatorname{sign}(W_{k/n}-W_{(k-1)/n}) 1_{\{(k-1)/n<t\le k/n\}}
\end{equation*}
then the integrals
\begin{equation*}
\int_0^1\alpha^n\,dW=\sum_{k=1}^n\left|W_{k/n}-W_{(k-1)/n}\right|
\end{equation*}
will tend to infinity as $n\rightarrow\infty$, despite the fact that $\alpha^n$ are all bounded by one. This means that even if we approximate $a(X_t)$ as closely as we like by piecewise constant functions, the integral can diverge to infinity.

\section{The It\^o integral}

It\^o's solution was to first note that if we can define the integral in a way such that (\ref{eq:2}) has a unique solution, then $X_t$ should only depend on the values of $W_s$ for $s\le t$. So, define $\mathcal{F}_t$ to be the collection of events in the probability space observable up until time $t$. This is the smallest $\sigma$-algebra with respect to which $W_s$ is measurable for $s\le t$,
\begin{equation*}
\mathcal{F}_t=\sigma\left(W_s:s\le t\right).
\end{equation*}
Then, $(\mathcal{F}_t)_{t\in\mathbb{R}_+}$ is a \PMlinkname{filtration of $\sigma$-algebras}{FiltrationOfSigmaAlgebras} to which $W$ is adapted.
If we assume that the piecewise constant process $\alpha_t$ is adapted, then the variables $c_k$ in (\ref{eq:1}) will be $\mathcal{F}_{t_{k-1}}$-measurable. Such processes are called \emph{elementary predictable}, and the following identity is obtained
\begin{equation}\label{eq:5}
\mathbb{E}\left[\left(\int_0^t\alpha\,dW\right)^2\right]=\mathbb{E}\left[\int_0^t\alpha^2_s\,ds\right].
\end{equation}
This is known as the \emph{It\^o isometry}. It ensures that, for bounded and adapted integrands $\alpha$, the integral with respect to $W$ cannot become too large, on average. Furthermore, the right hand side is defined for all jointly measurable processes $\alpha$, and allows us extend the integral with respect to $W$ to integrands which can be approximated by piecewise constant and adapted processes. Such processes are called \emph{predictable} and include all continuous and adapted processes, so it gives meaning to (\ref{eq:2}). If we define
\begin{equation*}
\Vert \alpha\Vert _{2,t}\equiv\mathbb{E}\left[\int_0^t \alpha^2_s\,ds\right]^{\frac{1}{2}}
\end{equation*}
then this is a norm on the set of predictable processes --- in fact, it is the $L^2(\mathbb{P}\times\lambda)$-norm, where $\lambda$ is the Lebesgue measure on $[0,t]$. Equation (\ref{eq:5}) shows that the map taking the adapted and piecewise constant function $\alpha$ to $\int_0^t\alpha\,dW$ is an \emph{isometry} with respect to the norms $\Vert\cdot\Vert_{2,t}$ and the $L^2(\mathbb{P})$-norm $\Vert\cdot\Vert_2$. Furthermore, it can be shown, by using the functional monotone class theorem, that every predictable process $\alpha$ with finite $\Vert\cdot\Vert_{2,t}$ norm can be approximated in this norm by a sequence elementary predictable processes $\alpha^n$. That is, the elementary predictable processes are \PMlinkname{dense}{Dense} under the $\Vert\cdot\Vert_{2,t}$ norm, and the It\^o integral is defined to be the unique \PMlinkname{continuous extension}{BoundedLinearExtension} from the elementary predictable processes. So, if $\Vert\alpha-\alpha^n\Vert_{2,t}\rightarrow 0$ then,
\begin{equation*}
\int_0^t\alpha\,dW=\lim_{n\rightarrow\infty}\int_0^t\alpha^n\,dW
\end{equation*}
where convergence is in the $L^2(\mathbb{P})$ norm.

\section{Extension of the integral}

The definition above gives a construction of the It\^o integral which is useful in many applications.
However, the restriction that $\Vert\alpha\Vert_{2,t}$ must be finite is too restrictive for some situations.
For example, it is possible to construct solutions to the stochastic differential equation
\begin{equation*}
dX = X^c\,dW
\end{equation*}
for any constant $c>0$ and any given initial value $X_0>0$. When $c>1$ then it is known that $X^{2c}_t$ does not have finite expectation for $t>0$ and, therefore, $X^c$ does not have finite $\Vert\cdot\Vert_{2,t}$ norm.

The It\^o integral can be extended to all predictable processes $\alpha$ such that
\begin{equation*}
\int_0^t\alpha^2_s\,ds<\infty
\end{equation*}
with probability one, for each $t>0$. Such processes are said to be $W$-integrable.
Given any such $\alpha$, the functional monotone class theorem can be used to show that there exists a sequence of elementary predictable processes $\alpha^n$ such that
\begin{equation*}
\int_0^t(\alpha^n_s-\alpha_s)^2\,ds\rightarrow 0
\end{equation*}
\PMlinkname{in probability}{ConvergenceInProbability} for every $t>0$. Then, using the It\^o isometry it can be shown that the limit
\begin{equation*}
\int_0^t\alpha\,dW\equiv \lim_{n\rightarrow\infty}\int_0^t\alpha^n\,dW
\end{equation*}
exists, where convergence is in probability.
Finally, we note that that this only defines the integral $\mathbb{P}$-almost everywhere, at each time. However, as probability measures only satisfy countable additivity, this only simultaneously defines the values of the integral at different times $t$ on countable subsets of $\mathbb{R}_+$. The additional constraint is added that the sample paths $t\mapsto\int_0^t\alpha\,dW$ are continuous, and it can be shown that it is possible to take such continuous versions of the integral. This defines the integral simultaneously at all times up to a zero probability set.

We have presented a definition of the It\^o integral with respect to a Brownian motion $W$. Then, an important result for manipulating such integrals is \PMlinkname{It\^o's lemma}{ItosFormula}.
Furthermore, it is sometimes necessary to define integration for more general processes. See \PMlinkname{stochastic integral}{StochasticIntegration} for a definition of the integral with respect to general semimartingales, which includes discontinuous processes, such as L\'evy processes.


\begin{thebibliography}{9}
\bibitem{Oksendal:several}
Bernt \O{}ksendal.
{\em \PMlinkescapetext{Stochastic Differential Equations},
An Introduction with Applications.} 5th ed. Springer, 1998.
\end{thebibliography}

%%%%%
%%%%%
\end{document}
