\documentclass[12pt]{article}
\usepackage{pmmeta}
\pmcanonicalname{ClassStructure}
\pmcreated{2013-03-22 14:18:21}
\pmmodified{2013-03-22 14:18:21}
\pmowner{CWoo}{3771}
\pmmodifier{CWoo}{3771}
\pmtitle{class structure}
\pmrecord{12}{35765}
\pmprivacy{1}
\pmauthor{CWoo}{3771}
\pmtype{Definition}
\pmcomment{trigger rebuild}
\pmclassification{msc}{60J10}
\pmrelated{MarkovChain}
\pmdefines{communicating class}
\pmdefines{irreducible chain}
\pmdefines{closed class}
\pmdefines{absorbing state}

\endmetadata

\usepackage{amssymb}
\usepackage{amsmath}
\usepackage{amsfonts}
\begin{document}
\PMlinkescapeword{chain}
Let $(X_n)_{n\ge 1}$ be a stationary Markov chain and let $i$ and $j$ be states in the indexing set. We say that $i$ leads to $j$ or $j$ is accessible from $i$, and write $i\to j$, if it is possible for the chain to get from state $i$ to state $j$:
\[i\to j \iff P(X_n = j : X_0 = i) > 0 \quad \textrm{for some} \quad n\ge 0 \]

If $i\to j$ and $j\to i$ we say $i$ communicates with $j$ and write $i\leftrightarrow j$. $\leftrightarrow$ is an equivalence relation (easy to prove).  The equivalence classes of this relation are the \emph{communicating classes} of the chain. If there is just one class, we say the chain is an \emph{irreducible chain}.

A class $C$ is a \emph{closed class} if $i\in C$ and $i\to j$ implies that $j\in C$ ``Once the chain enters a closed class, it cannot leave it''

A state $i$ is an \emph{absorbing state} if $\{i\}$ is a closed class.
%%%%%
%%%%%
\end{document}
