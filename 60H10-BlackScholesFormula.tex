\documentclass[12pt]{article}
\usepackage{pmmeta}
\pmcanonicalname{BlackScholesFormula}
\pmcreated{2013-03-22 16:29:59}
\pmmodified{2013-03-22 16:29:59}
\pmowner{stevecheng}{10074}
\pmmodifier{stevecheng}{10074}
\pmtitle{Black-Scholes formula}
\pmrecord{15}{38675}
\pmprivacy{1}
\pmauthor{stevecheng}{10074}
\pmtype{Topic}
\pmcomment{trigger rebuild}
\pmclassification{msc}{60H10}
\pmclassification{msc}{91B28}
\pmsynonym{Black-Scholes pricing formula}{BlackScholesFormula}
\pmrelated{AnalyticSolutionOfBlackScholesPDE}
\pmrelated{DerivationOfBlackScholesFormulaInMartingaleForm}

% The standard font packages
\usepackage{amssymb}
\usepackage{amsmath}
\usepackage{amsfonts}

% Including EPS/PDF graphics (\includegraphics)
%\usepackage{graphicx}

% Making matrix-based graphics
%%%\usepackage{xypic}

% Enumeration lists with different styles
%\usepackage{enumerate}

% The standard number systems
\newcommand{\real}{\mathbb{R}}

% Absolute values and norms
% Normal, wide, and big versions of the delimeters
\providecommand{\abs}[1]{\lvert#1\rvert}
\providecommand{\absW}[1]{\left\lvert#1\right\rvert}
\providecommand{\absB}[1]{\Bigl\lvert#1\Bigr\rvert}

% Differentiation operators
\providecommand{\od}[2]{\frac{d #1}{d #2}}
\providecommand{\pd}[2]{\frac{\partial #1}{\partial #2}}
\providecommand{\pdd}[2]{\frac{\partial^2 #1}{\partial #2}}
\providecommand{\ipd}[2]{\partial #1 / \partial #2}

% Calligraphic letters
\newcommand{\sF}{\mathcal{F}}

% Things related to risk-neutral measure
\newcommand{\EQ}{\mathbb{E}^\mathbb{Q}}
\newcommand{\PQ}{\mathbb{Q}}
\newcommand{\PP}{\mathbb{P}}

\newcommand{\Le}{\mathbf{L}}

\begin{document}
\PMlinkescapeword{real}
\PMlinkescapeword{unit}
\PMlinkescapeword{units}
\PMlinkescapeword{dividends}
\PMlinkescapeword{thin}
\PMlinkescapeword{term}
\PMlinkescapeword{representation}
\PMlinkescapeword{divisible}
\PMlinkescapeword{specification}
\PMlinkescapeword{terminal}
\PMlinkescapeword{holder}

\tableofcontents

\medskip

The \emph{Black-Scholes formula}
gives the theoretical ``arbitrage-free'' price of a stock option
at any time $t$ before the maturity time $T$.

In financial parlance, an ``arbitrage'' is a riskless
profit.  So the prices being ``arbitrage-free''
means that it is impossible to construct
 a strategy\footnote{Strictly speaking, the strategies described here must be construed as the \emph{limiting case}, in continuous time, of no-arbitrage trading
strategies in discrete time, as the time period between trades become
small.  Otherwise, it is not even obvious what trading
in ``continuous time'' means, and how to actually execute such trades.}
 of trading the stock option and the underlying stock,
at the prices given,
that starts with no money at time $0$,
ends with a balance at time $T$ that
is non-negative almost surely,
and is positive 
with some positive probability.

A stock \emph{call option} is a contract, in which the vendor gives
the buyer a right to purchase some shares of stock in the future,
at a fixed \emph{strike price} agreed upon the writing of the contract.
However, the right need not be exercised by the buyer.
In fact it would be disadvantageous to do so if the stock price
evolves below the strike price since the writing of the contract;
if the buyer still wants the stock, it would be cheaper to just
buy it from the market.

\subsection{Black-Scholes pricing formula}

Let 
\begin{itemize}
\item
$K$ be the \emph{strike price} of the call option
\item
$T$ be the \emph{maturity date} of the call option
\item
$r$ be the prevailing \emph{interest rate}, assumed to be constant 
\item
$\sigma > 0$ be the volatility of the stock price (per unit square-root time)
\item
$X(t)$ be the stochastic process representing the price of the stock itself at time $t$
\end{itemize}

Then the fair price $V(t)$, at time $t \leq T$, of the 
call option with the above contract parameters,
under the Black-Scholes modelling assumptions,
is given by:

\begin{equation}\label{eq:bs-formula}
\begin{split}
V(t) &= x \, \Phi\bigl( d_+( \tau, x ) \bigr)
- K e^{-r\tau} \, \Phi\bigl( d_-(\tau, x) \bigr)\,, \\
d_\pm (\tau, x) &= \frac{ \log(x/K) + (r \pm \tfrac12 \sigma^2) \tau }{\sigma \sqrt{\tau} }\,, \quad \tau  = T-t, \quad x = X(t)\,,
\end{split}
\end{equation}
where $\Phi$ is the cumulative distribution function of a standard normal random variable.


The result \eqref{eq:bs-formula}
is commonly referred to as the \emph{Black-Scholes pricing formula}
or just as the \emph{Black-Scholes formula},
after Fischer Black and Myron Scholes, who derived it in the 1970s.

F. Black and M. Scholes first arrived at the solution for $V(t)$
not with our probabilistic calculations,
but by describing $V(t)$ in terms of a 
partial differential equation, and solving the PDE analytically.  
That PDE, now called, appropriately, 
the \emph{Black-Scholes partial differential equation},
gives an important alternative formulation of the results here.

\subsection{General representation of option price}

$V(t)$ in equation \eqref{eq:bs-formula} is a stochastic process itself, 
but it happens to depend on $X(t)$ only
for its random part.

More generally, for any financial contract, not necessarily
a call option, that pays an amount $V(T)$,
the Black-Scholes model shows that its fair price $V(t)$
at time $t$
must be given by:
\begin{align}\label{eq:port-result}
V(t) = e^{-r(T-t)} \, \EQ\bigl[V(T) \mid \sF_t\bigr]\,.
\end{align}
for some probability measure $\PQ$,
and $\{ \sF_t \}$ is the filtration
describing the information available about the stock price up to time $t$.

If we set $V(T)= \max(X(T)- K, 0)$, then 
equation \eqref{eq:bs-formula}
follows from a straightforward 
calculation starting with equation \eqref{eq:port-result}.

Intuitively, \eqref{eq:port-result} states that
the option price at the current time is the expected value
of the promised pay-off amount at the future time,
discounted back to the current time at the rate of interest in force.

\subsection{The risk-neutral measure}

The probability measure $\PQ$ appearing in \eqref{eq:port-result},
is called the \emph{risk-neutral measure}.
It depends on the underlying risk factors that drive the contract pay-off,
but it does not depend on the contract itself.
This probability measure is distinct from the real-world or historical 
probability measure.
It does not actually describe ``probabilities'' in the intuitive sense
of the word, but is a mathematical tool to express the solutions
to the underlying stochastic differential equations.

Equation \eqref{eq:port-result} is equivalent to
saying that $V(t) e^{-rt}$ is a martingale under the probability
measure $\PQ$.




\subsection{Model assumptions}
The Black-Scholes formula is derived under the following
model assumptions.

In the real world, of course, all of these simplifying assumptions
are wrong in one way or another; however,
the Black-Scholes formula is often used
to give ballpark figures when quoting option prices.

\subsubsection{Stock price}
The stock 
(or any other underlying tradable asset)
has a price that follows a \emph{geometric Brownian motion
process} described by the stochastic differential equation:
\begin{align}\label{eq:stock}
d X(t) = \mu X(t) \, dt + \sigma X(t) \, dW(t)\,,
\end{align}
where
\begin{itemize}
\item
$X(t)$ is the price of the stock at time $t \geq 0$; the quantity $X(t)$ is
a random variable. 

(That is, for each $t$, $X(t)$ is a function of $\omega$,
for $\omega$ ranging over the underlying sample space; but the implicit
notation is more convenient and widely used.)
\item
$\mu > 0$ is the growth rate of the stock.  The quantity $\mu$ has units
of reciprocal time.  

$\mu$ is assumed to be a constant (independent of time and the state
of the system).
\item
$\sigma > 0$ is the volatility of the stock.
It can be thought as the standard deviation normalized by time;
$\sigma$ has units of the reciprocal square root of time.

$\sigma$ is assumed to be constant.
\item
$W(t)$ is a standard Brownian motion, also called a Wiener process.
\end{itemize}

The assumption of \emph{geometric} Brownian motion means
that the \emph{relative} price movements $X(t + \Delta t)/X(t)$
after time $t$
are independent of the stock price history before time $t$.
In general,
 relative changes in price 
are more financially significant and useful to model
(e.g. ``Google stock has gone up 40\% this year'')
than absolute changes in price.

We also assume that the stock does not pay dividends,
that there are no \PMlinkescapetext{limits} on the trading of any asset,
or borrowing or lending of any amount of cash at the risk-free interest rate, 
and no transaction costs.
Trading is also assumed to be conducted continuously
in perfectly divisible amounts,
at a single price at each point in time.

\subsubsection{Interest on cash}

Moreover, cash in the model economy is assumed to grow
with time at a \emph{risk-free continuously-compounded interest rate} of $r$,
and $r$ is constant.  Thus the amount $M(t)$ in the ``money-market account''
or ``bank account''
at time $t$,
with an initial infusion of amount $M(0)$ at time $0$,
is given by
\begin{align*}
M(t) = M(0) \, e^{rt}\,.
\end{align*}


%%%%%
%%%%%
\end{document}
