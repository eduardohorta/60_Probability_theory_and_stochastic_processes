\documentclass[12pt]{article}
\usepackage{pmmeta}
\pmcanonicalname{WaldsEquation}
\pmcreated{2013-03-22 14:40:04}
\pmmodified{2013-03-22 14:40:04}
\pmowner{CWoo}{3771}
\pmmodifier{CWoo}{3771}
\pmtitle{Wald's equation}
\pmrecord{8}{36267}
\pmprivacy{1}
\pmauthor{CWoo}{3771}
\pmtype{Theorem}
\pmcomment{trigger rebuild}
\pmclassification{msc}{60K05}
\pmclassification{msc}{60G40}

\endmetadata

% this is the default PlanetMath preamble.  as your knowledge
% of TeX increases, you will probably want to edit this, but
% it should be fine as is for beginners.

% almost certainly you want these
\usepackage{amssymb,amscd}
\usepackage{amsmath}
\usepackage{amsfonts}

% used for TeXing text within eps files
%\usepackage{psfrag}
% need this for including graphics (\includegraphics)
%\usepackage{graphicx}
% for neatly defining theorems and propositions
%\usepackage{amsthm}
% making logically defined graphics
%%%\usepackage{xypic}

% there are many more packages, add them here as you need them

% define commands here
\begin{document}
Let $X_1, X_2,\ldots, X_N$ be a sequence of $N$ iid random variables distributed as random variable $X$, such that 
\begin{enumerate}
\item $N>0$ is itself a random variable (integer-valued),
\item the expectation of $X$, $\operatorname{E}[X]<\infty$, and
\item $\operatorname{E}[N]<\infty$. 
\end{enumerate}
Then 
$$\operatorname{E}\Big[\sum_{i=1}^{N}X_i\Big]=\operatorname{E}[N]\operatorname{E}[X].$$
\par
The integer $N$ from above can be viewed as a stopping time for the stochastic process $\lbrace X_i \mid i\in\mathbb{Z}^+ \rbrace$.
%%%%%
%%%%%
\end{document}
