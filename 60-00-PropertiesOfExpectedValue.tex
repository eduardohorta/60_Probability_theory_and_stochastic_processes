\documentclass[12pt]{article}
\usepackage{pmmeta}
\pmcanonicalname{PropertiesOfExpectedValue}
\pmcreated{2013-03-22 16:16:05}
\pmmodified{2013-03-22 16:16:05}
\pmowner{Andrea Ambrosio}{7332}
\pmmodifier{Andrea Ambrosio}{7332}
\pmtitle{properties of expected value}
\pmrecord{6}{38378}
\pmprivacy{1}
\pmauthor{Andrea Ambrosio}{7332}
\pmtype{Theorem}
\pmcomment{trigger rebuild}
\pmclassification{msc}{60-00}

% this is the default PlanetMath preamble.  as your knowledge
% of TeX increases, you will probably want to edit this, but
% it should be fine as is for beginners.

% almost certainly you want these
\usepackage{amssymb}
\usepackage{amsmath}
\usepackage{amsfonts}

% used for TeXing text within eps files
%\usepackage{psfrag}
% need this for including graphics (\includegraphics)
%\usepackage{graphicx}
% for neatly defining theorems and propositions
\usepackage{amsthm}
% making logically defined graphics
%%%\usepackage{xypic}

% there are many more packages, add them here as you need them

% define commands here

\begin{document}
1) \emph{(normalization) } Let $X$ be almost surely constant random
variable, i.e. $\Pr \left\{ X=c\right\} =1$; then $E\left[ X\right] =c$.

2) \emph{(linearity) } Let $X$, $Y$ be random variables such that $E\left[
\left\vert X\right\vert \right] <\infty $ and $E\left[ \left\vert
Y\right\vert \right] <\infty $ and let $a$, $b$ be real numbers; then $E%
\left[ \left\vert aX+bY\right\vert \right] <\infty $ and $E\left[ aX+bY%
\right] =aE\left[ X\right] +bE\left[ Y\right] $.

3) \emph{(monotonicity) } Let $X$, $Y$ be random variables such that 
$\Pr \left\{ X\leq Y\right\} =1$ and $E\left[ \left\vert X\right\vert \right]
<\infty $, $E\left[ \left\vert Y\right\vert \right] <\infty $; then $E\left[
X\right] \leq E\left[ Y\right] $.

\begin{proof}

1) Let's define%
\[
F=\left\{ \omega \in \Omega :X\left( \omega \right) =c\right\} ;
\]

Then by hypothesis 
\[
\Pr \left\{ \Omega \backslash F\right\} =0
\]

and
\[
\Pr \left\{ F\right\} =1.
\]

We have:

\begin{eqnarray*}
E[X] &=&\int_{\Omega }X\left( \omega \right) dP \\
&=&\int_{\Omega \backslash F}X\left( \omega \right) dP+\int_{F}X\left(
\omega \right) dP \\
&=&\int_{F}X\left( \omega \right) dP \\
&=&\int_{F}cdP \\
&=&c\Pr \left\{ F\right\} =c.
\end{eqnarray*}


2) \textit{[to be done]}.


3) Let's define%
\[
F=\left\{ \omega \in \Omega :X\left( \omega \right) \leq Y\left( \omega
\right) \right\} ;
\]

Then by hypothesis 
\[
\Pr \left\{ \Omega \backslash F\right\} =0
\]

and
\[
\Pr \left\{ F\right\} =1.
\]

We have, keeping in mind property 2),

\begin{eqnarray*}
E[Y]-E[X] &=&E[Y-X] \\
&=&\int_{\Omega }\left[ Y\left( \omega \right) -X\left( \omega \right) %
\right] dP \\
&=&\int_{\Omega \backslash F}\left[ Y\left( \omega \right) -X\left( \omega
\right) \right] dP+\int_{F}\left[ Y\left( \omega \right) -X\left( \omega
\right) \right] dP \\
&=&\int_{F}\left[ Y\left( \omega \right) -X\left( \omega \right) \right]
dP\geq 0.
\end{eqnarray*}
\end{proof}
%%%%%
%%%%%
\end{document}
