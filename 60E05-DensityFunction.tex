\documentclass[12pt]{article}
\usepackage{pmmeta}
\pmcanonicalname{DensityFunction}
\pmcreated{2013-03-22 13:02:49}
\pmmodified{2013-03-22 13:02:49}
\pmowner{drini}{3}
\pmmodifier{drini}{3}
\pmtitle{density function}
\pmrecord{12}{33450}
\pmprivacy{1}
\pmauthor{drini}{3}
\pmtype{Definition}
\pmcomment{trigger rebuild}
\pmclassification{msc}{60E05}
\pmsynonym{probability function}{DensityFunction}
\pmsynonym{density}{DensityFunction}
\pmsynonym{probabilities function}{DensityFunction}
\pmrelated{DistributionFunction}
\pmrelated{CumulativeDistributionFunction}
\pmrelated{RandomVariable}
\pmrelated{Distribution}
\pmrelated{GeometricDistribution2}

\usepackage{graphicx}
\usepackage{amsmath}
%%%\usepackage{xypic} 
\usepackage{bbm}
\newcommand{\Z}{\mathbbmss{Z}}
\newcommand{\C}{\mathbbmss{C}}
\newcommand{\R}{\mathbbmss{R}}
\newcommand{\Q}{\mathbbmss{Q}}
\newcommand{\mathbb}[1]{\mathbbmss{#1}}
\newcommand{\figura}[1]{\begin{center}\includegraphics{#1}\end{center}}
\newcommand{\figuraex}[2]{\begin{center}\includegraphics[#2]{#1}\end{center}}
\begin{document}
Let $X$ be a discrete random variable with sample space $\{x_1,x_2,\ldots\}$.
Let $p_k$ be the probability of $X$ taking the value $x_k$.

The function
$$
f(x)=\
\begin{cases}
p_k & \text{if }x=x_k\\
0 & \text{otherwise}
\end{cases}
$$
is called the \emph{probability function} or \emph{density function}.

It must hold:
$$\sum_{j=1}^{\infty} f(x_j)=1$$

If the density function for a random variable is known, we can calculate the probability of $X$ being on certain interval:
$$P[a<X\leq b] = \sum_{a<x_j\leq b}f(x_j) = \sum_{a<x_j\leq b}p_j.$$

The definition can be extended to continuous random variables in a direct way: The probability of $x$ being on a given interval is calculated with an integral instead of using a summation:
$$P[a<X\leq b] = \int_a^b f(x) dx.$$

For a more formal approach using measure theory, look at probability distribution function entry.
%%%%%
%%%%%
\end{document}
