\documentclass[12pt]{article}
\usepackage{pmmeta}
\pmcanonicalname{WienerMeasure}
\pmcreated{2013-03-22 15:55:53}
\pmmodified{2013-03-22 15:55:53}
\pmowner{neldredge}{4974}
\pmmodifier{neldredge}{4974}
\pmtitle{Wiener measure}
\pmrecord{7}{37940}
\pmprivacy{1}
\pmauthor{neldredge}{4974}
\pmtype{Definition}
\pmcomment{trigger rebuild}
\pmclassification{msc}{60G15}
\pmrelated{BrownianMotion}
\pmrelated{CameronMartinSpace}
\pmdefines{Wiener space}
\pmdefines{Wiener measure}

\endmetadata

% this is the default PlanetMath preamble.  as your knowledge
% of TeX increases, you will probably want to edit this, but
% it should be fine as is for beginners.

% almost certainly you want these
\usepackage{amssymb}
\usepackage{amsmath}
\usepackage{amsfonts}

% used for TeXing text within eps files
%\usepackage{psfrag}
% need this for including graphics (\includegraphics)
%\usepackage{graphicx}
% for neatly defining theorems and propositions
\usepackage{amsthm}
% making logically defined graphics
%%%\usepackage{xypic}

% there are many more packages, add them here as you need them

% define commands here
\theoremstyle{definition}
\newtheorem{definition}{Definition}
\begin{document}
\begin{definition}
The \emph{Wiener space} $W(\mathbb{R})$ is just the set of all continuous paths $\omega : [0, \infty) \to \mathbb{R}$ satisfying $\omega(0)=0$.  It may be made into a measurable space by equipping it with the $\sigma$-algebra $\mathcal{F}$ generated by all projection maps $\omega \mapsto \omega(t)$ (or the completion of this under Wiener measure, see below).
\end{definition}

Thus, an $\mathbb{R}$-valued continuous-time stochastic process $X_t$ with continuous sample paths can be thought of as a random variable taking its values in $W(\mathbb{R})$.

\begin{definition}
In the case where $X_t = W_t$ is Brownian motion, the distribution measure $P$ induced on $W(\mathbb{R})$ is called the \emph{Wiener measure}.  That is, $P$ is the unique probability measure on $W(\mathbb{R})$ such that for any finite sequence of times $0<t_1<\ldots<t_n$ and Borel sets $A_1,\ldots,A_n \subset \mathbb{R}$
\begin{eqnarray}
P(\{\omega : \omega(t_1)\in A_1,\ldots,\omega(t_n) \in A_n\}) &=& \int_{A_1}\cdots\int_{A_n} p(t_1,0,x_1)p(t_2-t_1,x_1,x_2)\cdots \\
&& \cdots p(t_n-t_{n-1},x_{n-1},x_n) \; dx_1 \cdots \; dx_n,
\end{eqnarray}
where $p(t,x,y) = \frac{1}{\sqrt{2\pi t}}\exp(-\frac{(x-y)^2}{2t})$ defined for any $x,y\in\mathbb{R}$ and $t>0$.
\end{definition}

This of course corresponds to the defining property of Brownian motion.  The other properties carry over as well; for instance, the set of paths in $W(\mathbb{R})$ which are nowhere differentiable is of $P$-measure $1$.

The Wiener space $W(\mathbb{R}^d)$ and corresponding Wiener measure are defined similarly, in which case $P$ is the distribution of a $d$-dimensional Brownian motion.
%%%%%
%%%%%
\end{document}
