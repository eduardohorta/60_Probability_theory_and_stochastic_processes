\documentclass[12pt]{article}
\usepackage{pmmeta}
\pmcanonicalname{sigmaalgebraAtAStoppingTime}
\pmcreated{2013-03-22 18:38:56}
\pmmodified{2013-03-22 18:38:56}
\pmowner{gel}{22282}
\pmmodifier{gel}{22282}
\pmtitle{$\sigma$-algebra at a stopping time}
\pmrecord{6}{41391}
\pmprivacy{1}
\pmauthor{gel}{22282}
\pmtype{Definition}
\pmcomment{trigger rebuild}
\pmclassification{msc}{60G40}
%\pmkeywords{stopping time}
%\pmkeywords{filtration}
\pmrelated{DoobsOptionalSamplingTheorem}

\endmetadata

% almost certainly you want these
\usepackage{amssymb}
\usepackage{amsmath}
\usepackage{amsfonts}

% used for TeXing text within eps files
%\usepackage{psfrag}
% need this for including graphics (\includegraphics)
%\usepackage{graphicx}
% for neatly defining theorems and propositions
\usepackage{amsthm}
% making logically defined graphics
%%%\usepackage{xypic}

% there are many more packages, add them here as you need them

% define commands here
\newtheorem*{theorem*}{Theorem}
\newtheorem*{lemma*}{Lemma}
\newtheorem*{corollary*}{Corollary}
\newtheorem*{definition*}{Definition}
\newtheorem{theorem}{Theorem}
\newtheorem{lemma}{Lemma}
\newtheorem{corollary}{Corollary}
\newtheorem{definition}{Definition}

\begin{document}
\PMlinkescapeword{properties}
\PMlinkescapeword{index set}

Let $(\mathcal{F}_t)_{t\in\mathbb{T}}$ be a \PMlinkname{filtration}{FiltrationOfSigmaAlgebras} on a measurable space $(\Omega,\mathcal{F})$. For every $t\in\mathbb{T}$, the $\sigma$-algebra $\mathcal{F}_t$ represents the collection of events which are observable up until time $t$. This concept can be generalized to any stopping time $\tau\colon\Omega\rightarrow\mathbb{T}\cup\{\infty\}$.

For a stopping time $\tau$, the collection of events observable up until time $\tau$ is denoted by $\mathcal{F}_\tau$ and is generated by sampling progressively measurable processes
\begin{equation*}
\mathcal{F}_\tau=\sigma\left(\left\{X_{\tau\wedge t}: X\textrm{ is progressive, }t\in\mathbb{T}\right\}\right).
\end{equation*}
The reason for sampling $X$ at time $\tau\wedge t$ rather than at $\tau$ is to include the possibility that $\tau=\infty$, in which case $X_\tau$ is not defined.

A random variable $V$ is $\mathcal{F}_{\tau}$-measurable if and only if it is $\mathcal{F}_\infty$-measurable and the process $X_t\equiv 1_{\{\tau\le t\}}V$ is adapted.

This can be shown as follows. If $X$ is a progressively measurable process, then the stopped process $X^{\tau\wedge s}$ is also progressive. In particular, $V\equiv X_{\tau\wedge s}=X^{\tau\wedge s}_s$ is $\mathcal{F}_\infty$-measurable and $1_{\{\tau\le t\}}V=1_{\{\tau\le t\}}X^{\tau\wedge s}_t$ is $\mathcal{F}_t$-measurable.
Conversely, if $V$ is $\mathcal{F}_t$-measurable then $X_s\equiv 1_{\{s> t\}}V$ is a progressive process and $1_{\{\tau> t\}}V=X_{\tau\wedge t}$ is $\mathcal{F}_\tau$-measurable. By letting $t$ increase to infinity, it follows that $1_{\{\tau=\infty\}}V$ is $\mathcal{F}_\tau$-measurable for every $\mathcal{F}_\infty$-measurable random variable $V$.
Now suppose also that $X_t\equiv 1_{\{\tau\le t\}}V$ is adapted, and hence progressive. Then, $1_{\{\tau\le t\}}V=X_{\tau\wedge t}$ is $\mathcal{F}_\tau$-measurable. Letting $t$ increase to infinity shows that $V=1_{\{\tau<\infty\}}V+1_{\{\tau=\infty\}}V$ is $\mathcal{F}_\tau$-measurable.

As a set $A$ is $\mathcal{F}_\tau$-measurable if and only if $1_A$ is an $\mathcal{F}_\tau$-measurable random variable, this gives the following alternative definition,
\begin{equation*}
\mathcal{F}_\tau=\left\{A\in\mathcal{F}_\infty:A\cap\{\tau\le t\}\in\mathcal{F}_t\textrm{ for all }t\in\mathbb{T}\right\}.
\end{equation*}

From this, it is not difficult to show that the following properties are satisfied
\begin{enumerate}
\item Any stopping time $\tau$ is $\mathcal{F}_\tau$-measurable.
\item If $\tau(\omega)=t\in\mathbb{T}\cup\{\infty\}$ for all $\omega\in\Omega$ then $\mathcal{F}_\tau=\mathcal{F}_t$.
\item If $\sigma,\tau$ are stopping times and $A\in\mathcal{F}_\sigma$ then $A\cap\{\sigma\le\tau\}\in\mathcal{F}_\tau$. In particular, if $\sigma\le\tau$ then $\mathcal{F}_\sigma\subseteq\mathcal{F}_\tau$.
\item If $\sigma,\tau$ are stopping times and $A\in\mathcal{F}_\sigma$ then $A\cap\{\sigma=\tau\}\in\mathcal{F}_\tau$.
\item if the filtration $(\mathcal{F}_t)$ is right-continuous and $\tau_n\ge\tau$ are stopping times with $\tau_n\rightarrow\tau$ then $\mathcal{F}_\tau=\bigcap_n\mathcal{F}_{\tau_n}$. More generally, if $\tau_n=\tau$ eventually then this is true irrespective of whether the filtration is right-continuous.
\item If $\tau_n$ are stopping times with $\tau_n=\tau$ eventually then $\mathcal{F}_{\tau_n}\rightarrow\mathcal{F}_\tau$. That is,
\begin{equation*}
\mathcal{F}_\tau=\bigcap_n\sigma\left(\bigcup_{m\ge n}\mathcal{F}_{\tau_m}\right).
\end{equation*}
\end{enumerate}

In continuous-time, for any stopping time $\tau$ the $\sigma$-algebra $\mathcal{F}_{\tau+}$ is the set of events observable up until time $t$ with respect to the right-continuous filtration $(\mathcal{F}_{t+})$. That is,
\begin{equation*}\begin{split}
\mathcal{F}_{\tau+}&=\left\{A\in\mathcal{F}_\infty:A\cap\{\tau\le t\}\in\mathcal{F}_{t+}\textrm{ for every }t\in\mathbb{T}\right\}\\
&=\left\{A\in\mathcal{F}_\infty:A\cap\{\tau< t\}\in\mathcal{F}_t\textrm{ for every }t\in\mathbb{T}\right\}.
\end{split}\end{equation*}

If $\tau_n\ge\tau$ are stopping times with $\tau_n>\tau$ whenever $\tau<\infty$ is not a maximal element of $\mathbb{T}$, and $\tau_n\rightarrow\tau$ then,
\begin{equation*}
\mathcal{F}_{\tau+}=\bigcap_n\mathcal{F}_{\tau_n}=\bigcap_n\mathcal{F}_{\tau_n+}.
\end{equation*}

The $\sigma$-algebra of events observable up until just before time $\tau$ is denoted by $\mathcal{F}_{\tau-}$ and is generated by sampling predictable processes
\begin{equation*}
\mathcal{F}_{\tau-} = \sigma\left(\left\{X_{\tau\wedge t}: X\textrm{ is predictable, }t\in\mathbb{T}\right\}\right).
\end{equation*}
Suppose that the index set $\mathbb{T}\subseteq\mathbb{R}$ has minimal element $t_0$.
As the predictable $\sigma$-algebra is generated by sets of the form $(s,\infty)\times A$ for $s\in\mathbb{T}$ and $A\in \mathcal{F}_s$, and $\{t_0\}\times A$ for $A\in\mathcal{F}_{t_0}$, the definition above can be rewritten as,
\begin{equation*}
\mathcal{F}_{\tau-} = \sigma\left(\left\{A\cap\{\tau>s\}:s\in\mathbb{T},A\in\mathcal{F}_s\right\}\cup\mathcal{F}_{t_0}\right).
\end{equation*}
Clearly, $\mathcal{F}_{\tau-}\subseteq\mathcal{F}_\tau\subseteq\mathcal{F}_{\tau+}$. Furthermore, for any stopping times $\sigma,\tau$ then $\mathcal{F}_{\sigma+}\subseteq\mathcal{F}_{\tau-}$ when restricted to the set $\{\sigma<\tau\}$.

If $\tau_n$ is a sequence of stopping times \PMlinkname{announcing}{PredictableStoppingTime} $\tau$, so that $\tau$ is predictable, then
\begin{equation*}
\mathcal{F}_{\tau-}=\sigma\left(\bigcup_n\mathcal{F}_{\tau_n}\right).
\end{equation*}

%%%%%
%%%%%
\end{document}
