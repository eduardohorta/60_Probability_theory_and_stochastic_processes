\documentclass[12pt]{article}
\usepackage{pmmeta}
\pmcanonicalname{StoppingTime}
\pmcreated{2013-03-22 14:41:13}
\pmmodified{2013-03-22 14:41:13}
\pmowner{gel}{22282}
\pmmodifier{gel}{22282}
\pmtitle{stopping time}
\pmrecord{11}{36294}
\pmprivacy{1}
\pmauthor{gel}{22282}
\pmtype{Definition}
\pmcomment{trigger rebuild}
\pmclassification{msc}{60K05}
\pmclassification{msc}{60G40}
\pmrelated{DoobsOptionalSamplingTheorem}
\pmrelated{PredictableStoppingTime}

\endmetadata

% this is the default PlanetMath preamble.  as your knowledge
% of TeX increases, you will probably want to edit this, but
% it should be fine as is for beginners.

% almost certainly you want these
\usepackage{amssymb,amscd}
\usepackage{amsmath}
\usepackage{amsfonts}

% used for TeXing text within eps files
%\usepackage{psfrag}
% need this for including graphics (\includegraphics)
%\usepackage{graphicx}
% for neatly defining theorems and propositions
%\usepackage{amsthm}
% making logically defined graphics
%%%\usepackage{xypic}

% there are many more packages, add them here as you need them

% define commands here
\begin{document}
\PMlinkescapeword{cover}
\PMlinkescapeword{inclusion}
\PMlinkescapeword{interval}
\PMlinkescapeword{equivalent}
Let $( \mathcal{F}_t)_{t\in\mathbb{T}}$ be a \PMlinkname{filtration}{FiltrationOfSigmaAlgebras} on a set $\Omega$.
A  random variable $\tau$ taking values in $\mathbb{T}\cup\{\infty\}$ is a \emph{stopping time} for the filtration $(\mathcal{F}_t)$ if the event $\lbrace \tau\le t \rbrace \in \mathcal{F}_t$ for every $t\in\mathbb{T}$. 
\\\\ 
\textbf{Remarks} 
\begin{itemize}
\item The set $\mathbb{T}$ is the index set for the time variable $t$, and the $\sigma$-algebra $\mathcal{F}_t$ is the collection of all events which are observable up to and including time $t$. Then, the condition that $\tau$ is a stopping time means that the outcome of the event $\{\tau\le t\}$ is known at time $t$.

\item In discrete time situations, where $\mathbb{T}=\{0,1,2,\ldots\}$, the condition that $\{\tau\le t\}\in\mathcal{F}_t$ is equivalent to requiring that $\{\tau=t\}\in\mathcal{F}_t$. This is not true for continuous time cases where $\mathbb{T}$ is an interval of the real numbers and hence uncountable, due to the fact that $\sigma$-algebras are not in general closed under taking uncountable unions of events.

\item A random time $\tau$ is a stopping time for a stochastic process $(X_t)$ if it is a stopping time for the natural filtration of $X$. That is, $\{\tau\le t\}\in\sigma(X_s:s\le t)$.

\item The first time that an adapted process $X_t$ hits a given value or set of values is a stopping time. The inclusion of $\infty$ into the range of $\tau$ is to cover the case where $X_t$ never hits the given values.

\item Stopping time is often used in gambling, when a gambler stops the betting process when he reaches a certain goal.  The time it takes to reach this goal is generally not a deterministic one.  Rather, it is a random variable depending on the current result of the bet, as well as the combined information from all previous bets.  
\end{itemize}
\par
\textbf{Examples.}
A gambler has \$1,000 and plays the slot machine at \$1 per play.
\begin{enumerate}
\item The gambler stops playing when his capital is depleted.  The number $\tau=n_1$ of plays that it takes the gambler to stop is a stopping time.
\item The gambler stops playing when his capital reaches \$2,000.  The number $\tau=n_2$ of plays that it takes the gambler to stop is a stopping time.
\item The gambler stops playing when his capital either reaches \$2,000, or is depleted, which ever comes first.  The number $\tau=\operatorname{min}(n_1,n_2)$ of plays that it takes the gambler to stop is a stopping time.
\end{enumerate}
%%%%%
%%%%%
\end{document}
