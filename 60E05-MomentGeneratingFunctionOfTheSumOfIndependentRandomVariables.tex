\documentclass[12pt]{article}
\usepackage{pmmeta}
\pmcanonicalname{MomentGeneratingFunctionOfTheSumOfIndependentRandomVariables}
\pmcreated{2013-03-22 17:17:11}
\pmmodified{2013-03-22 17:17:11}
\pmowner{me_and}{17092}
\pmmodifier{me_and}{17092}
\pmtitle{moment generating function of the sum of independent random variables}
\pmrecord{5}{39628}
\pmprivacy{1}
\pmauthor{me_and}{17092}
\pmtype{Corollary}
\pmcomment{trigger rebuild}
\pmclassification{msc}{60E05}

\endmetadata

%\usepackage{amssymb}
\usepackage{amsmath} %Needed for align & align*, and to correctly render proofs
%\usepackage{amsfonts}
\usepackage{amsthm}

%Named sets
%\newcommand{\R}{\mathbb{R}} %Real numbers (amssymb or amsfonts)
%\newcommand{\C}{\mathbb{C}} %Complex numbers (amssymb or amsfonts)

%Functions
%\newcommand{\modulus}[1]{\left|{#1}\right|} %|z|
%\newcommand{\integral}[4]{\int_{#1}^{#2}\!{#3}\,\mathrm{d}{#4}}

%Numbers
%\newcommand{\I}{\mathrm{i}} %sqrt{-1}
\newcommand{\e}{\mathrm{e}} %exponential

%Greek
%\newcommand{\ve}{\varepsilon} %nice epsilon
\begin{document}
Let $X_i$ be independent random variables for $i=1,\dotsc ,n$, let each $X_i$ have moment generating function $M_{X_i} (t)$, and let $X=\sum_{i=1}^n X_i$. Then the moment generating function of $X$ is
\[
  M_X(t)=\prod_{i=1}^n M_{X_i}(t)
.\]

\begin{proof}
By definition,
\begin{align*}
  M_X(t)&=E\left(\e^{tX}\right)\\
        &=E\left(\e^{t\left(X_1+\dotsb+X_n\right)}\right)\\
        &=E\left(\e^{tX_1}\dotsm\e^{tX_n}\right)
.\end{align*}
Now, since each $X_i$ is independent of the others, this becomes
\begin{align*}
  M_X(t)&=E\left(\e^{tX_1}\right)\dotsm E\left(\e^{tX_n}\right)\\
        &=M_{X_1}(t)\dotsm M_{X_n}(t)\\
        &=\prod_{i=1}^n M_{X_i}(t)
\end{align*}
as required.
\end{proof} 
%%%%%
%%%%%
\end{document}
