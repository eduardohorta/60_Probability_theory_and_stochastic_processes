\documentclass[12pt]{article}
\usepackage{pmmeta}
\pmcanonicalname{JointDiscreteDensityFunction}
\pmcreated{2013-03-22 11:54:55}
\pmmodified{2013-03-22 11:54:55}
\pmowner{mathcam}{2727}
\pmmodifier{mathcam}{2727}
\pmtitle{joint discrete density function}
\pmrecord{10}{30573}
\pmprivacy{1}
\pmauthor{mathcam}{2727}
\pmtype{Definition}
\pmcomment{trigger rebuild}
\pmclassification{msc}{60E05}
\pmsynonym{joint probability function}{JointDiscreteDensityFunction}
\pmsynonym{joint distribution}{JointDiscreteDensityFunction}

\usepackage{amssymb}
\usepackage{amsmath}
\usepackage{amsfonts}
%\usepackage{graphicx}
%%%%%\usepackage{xypic}
\begin{document}
\PMlinkescapeword{continuous}
\PMlinkescapeword{difference}
\PMlinkescapeword{satisfies}

Let $X_1, X_2, ..., X_n$ be $n$ random variables all defined on the same probability space. The \textbf{joint discrete density function} of $X_1, X_2, ..., X_n$, denoted by $f_{X_1, X_2, ..., X_n}(x_1,x_2,...,x_n)$, is the following function:\\
\par
$f_{X_1, X_2, ..., X_n}: R^n \to R$\\
$f_{X_1, X_2, ..., X_n}(x_1,x_2,...,x_n) = P[X_1 = x_1, X_2 = x_2, ... , X_n = x_n]$\\
\par
As in the single variable case, sometimes it's expressed as $p_{X_1, X_2, ..., X_n}(x_1,x_2,...,x_n)$ to mark the difference between this function and the continuous joint density function.\\
\par
Also, as in the case where $n=1$, this function satisfies:\\
\par
\begin{enumerate}
\item $f_{X_1, X_2, ..., X_n}(x_1,...,x_n) \geq  0$    $\forall (x_1,...,x_n)$
\item $\sum_{x_1, ... ,x_n}^{} {  f_{X_1, X_2, ..., X_n}(x_1,...,x_n) }= 1$
\end{enumerate}
\par
In this case, $f_{X_1, X_2, ..., X_n}(x_1,...,x_n) = P[ X_1 = x_1, X_2 = x_2, ... , X_n = x_n ]$.
%%%%%
%%%%%
%%%%%
%%%%%
\end{document}
