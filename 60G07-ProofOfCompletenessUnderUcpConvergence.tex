\documentclass[12pt]{article}
\usepackage{pmmeta}
\pmcanonicalname{ProofOfCompletenessUnderUcpConvergence}
\pmcreated{2013-03-22 18:40:35}
\pmmodified{2013-03-22 18:40:35}
\pmowner{gel}{22282}
\pmmodifier{gel}{22282}
\pmtitle{proof of completeness under ucp convergence}
\pmrecord{5}{41424}
\pmprivacy{1}
\pmauthor{gel}{22282}
\pmtype{Proof}
\pmcomment{trigger rebuild}
\pmclassification{msc}{60G07}
\pmclassification{msc}{60G05}
%\pmkeywords{stochastic process}
%\pmkeywords{ucp convergence}
%\pmkeywords{uniform convergence on compacts}

% almost certainly you want these
\usepackage{amssymb}
\usepackage{amsmath}
\usepackage{amsfonts}

% used for TeXing text within eps files
%\usepackage{psfrag}
% need this for including graphics (\includegraphics)
%\usepackage{graphicx}
% for neatly defining theorems and propositions
\usepackage{amsthm}
% making logically defined graphics
%%%\usepackage{xypic}

% there are many more packages, add them here as you need them

% define commands here
\newtheorem*{theorem*}{Theorem}
\newtheorem*{lemma*}{Lemma}
\newtheorem*{corollary*}{Corollary}
\newtheorem*{definition*}{Definition}
\newtheorem{theorem}{Theorem}
\newtheorem{lemma}{Lemma}
\newtheorem{corollary}{Corollary}
\newtheorem{definition}{Definition}

\begin{document}
\PMlinkescapeword{real}
\PMlinkescapeword{functions}
\PMlinkescapeword{complete}
\PMlinkescapeword{generating}
\PMlinkescapeword{bounded intervals}
\PMlinkescapeword{positive}
\PMlinkescapeword{integers}
\PMlinkescapeword{convergent}
\PMlinkescapeword{infinity}
\PMlinkescapeword{bounded}

Let $(\Omega,\mathcal{F},(\mathcal{F}_t)_{t\in\mathbb{R}_+},\mathbb{P})$ be a filtered probability space, $\mathcal{M}$ be a sub-$\sigma$-algebra of $\mathcal{B}(\mathbb{R}_+)\otimes\mathcal{F}$, and $S$ be a set of real valued functions on $\mathbb{R}_+$ which is \PMlinkname{closed}{Closed} under uniform convergence on compacts. We show that both the set of $\mathcal{M}$-measurable processes and the set of jointly measurable processes with sample paths almost surely in $S$ are \PMlinkname{complete}{Complete} under ucp convergence. The method used will be to show that we can pass to a subsequence which almost surely converges uniformly on compacts.

We start by writing out the metric generating the topology of uniform convergence on compacts (compact-open topology) for functions $\mathbb{R}_+\rightarrow\mathbb{R}$. This is the same as uniform convergence on each of the bounded intervals $[0,n)$ for positive integers $n$,
\begin{equation*}
d(X)\equiv\sum_{n=1}^\infty 2^{-n}\min\left(1,\sup_{t<n}|X_t|\right).
\end{equation*}
Then, the metric is $(X,Y)\mapsto d(X-Y)$.  Convergence under the ucp topology is given by
\begin{equation*}
D^{\rm ucp}(X)=\mathbb{E}[d(X)]
\end{equation*}
for any jointly measurable stochastic process $X$, with the (pseudo)metric being $(X,Y)\mapsto D^{\rm ucp}(X-Y)$.

Now, suppose that $X^n$ is a sequence of jointly measurable processes such that $X^n-X^m\xrightarrow{\rm ucp} 0$ as $m,n\rightarrow\infty$. Then, $D^{\rm ucp}(X^n-X^m)\rightarrow 0$ and we may pass to a subsequence $X^{n_k}$ satisfying $D^{\rm ucp}(X^{n_j}-X^{n_k})\le 2^{-j}$ whenever $k>j$. So,
\begin{equation*}
\mathbb{E}\left[\sum_k d(X^{n_k}-X^{n_{k+1}})\right]=\sum_kD^{\rm ucp}(X^{n_k}-X^{n_{k+1}})\le\sum_k2^{-k}=1.
\end{equation*}
In particular, this shows that $\sum_k d(X^{n_{k}}-X^{n_{k+1}})$ is almost surely finite and, therefore,
\begin{equation*}
d(X^{n_j}-X^{n_k})\le \sum_{i=j}^{k-1}d(X^{n_i}-X^{n_{i+1}})\rightarrow 0
\end{equation*}
as $k>j\rightarrow \infty$, with probability one.

So, the sequence $X^{n_k}$ is almost surely \PMlinkname{Cauchy}{CauchySequence}, under the topology of uniform convergence on compacts. We set
\begin{equation*}
X_t(\omega)\equiv\left\{
\begin{array}{ll}
\lim_{k\rightarrow\infty}X^{n_k}_t(\omega),&\textrm{if the limit exists},\\
0,&\textrm{otherwise}.
\end{array}
\right.
\end{equation*}
As measurability of real valued functions is preserved under pointwise convergence, it follows that if $X^n$ are $\mathcal{M}$-measurable, then so is $X$. In particular, $X$ is a jointly measurable process.
Furthermore, since convergence is almost surely uniform on compacts, if $X^n$ have sample paths in $S$ with probability one then so does $X$.

It only remains to show that $X^n\xrightarrow{\rm ucp}X$. However, we have already shown that $d(X^{n_k}-X)\rightarrow 0$ with probability one, hence $D^{\rm ucp}(X^{n_k}-X)\rightarrow 0$.
\begin{equation*}
D^{\rm ucp}(X^n-X)\le D^{\rm ucp}(X^n-X^{n_{k}})+D^{\rm ucp}(X^{n_k}-X).
\end{equation*}
Letting $k$ go to infinity, this is bounded by $\sup_{m>n}D^{\rm ucp}(X^n-X^m)$, which goes to zero as $n\rightarrow\infty$.

%%%%%
%%%%%
\end{document}
