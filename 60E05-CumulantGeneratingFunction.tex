\documentclass[12pt]{article}
\usepackage{pmmeta}
\pmcanonicalname{CumulantGeneratingFunction}
\pmcreated{2013-03-22 16:16:24}
\pmmodified{2013-03-22 16:16:24}
\pmowner{Andrea Ambrosio}{7332}
\pmmodifier{Andrea Ambrosio}{7332}
\pmtitle{cumulant generating function}
\pmrecord{17}{38384}
\pmprivacy{1}
\pmauthor{Andrea Ambrosio}{7332}
\pmtype{Definition}
\pmcomment{trigger rebuild}
\pmclassification{msc}{60E05}
\pmrelated{MomentGeneratingFunction}
\pmrelated{CharacteristicFunction2}

% this is the default PlanetMath preamble.  as your knowledge
% of TeX increases, you will probably want to edit this, but
% it should be fine as is for beginners.

% almost certainly you want these
\usepackage{amssymb}
\usepackage{amsmath}
\usepackage{amsfonts}

% used for TeXing text within eps files
%\usepackage{psfrag}
% need this for including graphics (\includegraphics)
%\usepackage{graphicx}
% for neatly defining theorems and propositions
%\usepackage{amsthm}
% making logically defined graphics
%%%\usepackage{xypic}

% there are many more packages, add them here as you need them

% define commands here

\begin{document}
Given a random variable $X$, the \emph{cumulant generating function} of $X$ is the following function:\\
\[
H_X(t) = \ln E[e^{tX}]
\]
for all $t \in R$ in which the expectation converges.

In other \PMlinkescapetext{words}, the cumulant generating function is just the logarithm of the moment generating function.

The cumulant generating function of $X$ is defined on a (possibly degenerate) interval containing $t=0$; one has $H_X(0)=0$; moreover, $H_X(t)$ is a \PMlinkname{convex function}{ConvexFunction}. (Indeed, the moment generating function is defined on a possibly degenerate interval containing $t=0$, which image is a positive interval containing $t=1$; so the logarithm is defined on the same interval on which is defined the moment generating function.)

The $k$th-derivative of the cumulant generating function evaluated at zero is the $k$th cumulant of $X$.

%%%%%
%%%%%
\end{document}
