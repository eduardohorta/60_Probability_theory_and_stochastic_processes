\documentclass[12pt]{article}
\usepackage{pmmeta}
\pmcanonicalname{DoobsOptionalSamplingTheorem}
\pmcreated{2013-03-22 16:43:41}
\pmmodified{2013-03-22 16:43:41}
\pmowner{skubeedooo}{5401}
\pmmodifier{skubeedooo}{5401}
\pmtitle{Doob's optional sampling theorem}
\pmrecord{8}{38949}
\pmprivacy{1}
\pmauthor{skubeedooo}{5401}
\pmtype{Theorem}
\pmcomment{trigger rebuild}
\pmclassification{msc}{60G44}
\pmclassification{msc}{60G46}
\pmclassification{msc}{60G42}
\pmrelated{Martingale}
\pmrelated{StoppingTime}
\pmrelated{SigmaAlgebraAtAStoppingTime}

\endmetadata

% this is the default PlanetMath preamble.  as your knowledge
% of TeX increases, you will probably want to edit this, but
% it should be fine as is for beginners.

% almost certainly you want these
\usepackage{amssymb}
\usepackage{amsmath}
\usepackage{amsfonts}

% used for TeXing text within eps files
%\usepackage{psfrag}
% need this for including graphics (\includegraphics)
%\usepackage{graphicx}
% for neatly defining theorems and propositions
\usepackage{amsthm}
% making logically defined graphics
%%%\usepackage{xypic}

% there are many more packages, add them here as you need them

% define commands here

\begin{document}
\PMlinkescapeword{optional}
\PMlinkescapeword{sampling theorem}
\PMlinkescapeword{index set}
\PMlinkescapeword{place}
\PMlinkescapeword{continuous}

Given a filtered probability space $(\Omega,\mathcal{F},(\mathcal{F}_t)_{t\in\mathbb{T}},\mathbb{P})$, a process $(X_t)_{t\in\mathbb{T}}$ is a martingale if it satisfies the equality
\begin{equation*}
\mathbb{E}[X_t\mid\mathcal{F}_s]=X_s
\end{equation*}
for all $s<t$ in the index set $\mathbb{T}$. Doob's optional sampling theorem says that this equality still holds if the times $s,t$ are replaced by bounded stopping times $S,T$. In this case, the $\sigma$-algebra $\mathcal{F}_s$ is replaced by the collection of \PMlinkname{events observable at the random time $S$}{SigmaAlgebraAtAStoppingTime},
\begin{equation*}
\mathcal{F}_S=\left\{A\in\mathcal{F}:A\cap\{S\le t\}\in\mathcal{F}_t\textrm{ for all }t\in\mathbb{T}\right\}.
\end{equation*}
In discrete-time, when the index set $\mathbb{T}$ is countable, the result is as follows.

\newtheorem*{thm}{Doob's Optional Sampling Theorem}
\begin{thm}
Suppose that the index set $\mathbb{T}$ is countable and that $S\le T$ are stopping times bounded above by some constant $c\in\mathbb{T}$.
If $(X_t)$ is a martingale then $X_T$ is an integrable random variable and
\begin{equation}
\mathbb{E}[X_T|\mathcal{F}_S] = X_S,\ \mathbb{P}\textrm{ almost surely}.
\end{equation}
Similarly, if $X$ is a submartingale then $X_T$ is integrable and
\begin{equation}
\mathbb{E}[X_T|\mathcal{F}_S] \ge X_S,\ \mathbb{P}\textrm{ almost surely}.
\end{equation}
If $X$ is a supermartingale then $X_T$ is integrable and
\begin{equation}
\mathbb{E}[X_T|\mathcal{F}_S] \le X_S,\ \mathbb{P}\textrm{ almost surely}.
\end{equation}
\end{thm}


This theorem shows, amongst other things, that in the case of a fair casino, where your return is a martingale, betting strategies involving `knowing when to quit' do not enhance your expected return.

In continuous-time, when the index set $\mathbb{T}$ an interval of the real numbers, then the stopping times $S,T$ can have a continuous distribution and $X_S,X_T$ need not be measurable quantities. Then, it is necessary to place conditions on the sample paths of the process $X$. In particular, Doob's optional sampling theorem holds in continuous-time if $X$ is assumed to be right-continuous.

%%%%%
%%%%%
\end{document}
