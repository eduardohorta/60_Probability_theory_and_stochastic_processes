\documentclass[12pt]{article}
\usepackage{pmmeta}
\pmcanonicalname{FiltrationOfsigmaalgebras}
\pmcreated{2013-03-22 18:37:13}
\pmmodified{2013-03-22 18:37:13}
\pmowner{gel}{22282}
\pmmodifier{gel}{22282}
\pmtitle{filtration of $\sigma$-algebras}
\pmrecord{5}{41355}
\pmprivacy{1}
\pmauthor{gel}{22282}
\pmtype{Definition}
\pmcomment{trigger rebuild}
\pmclassification{msc}{60G05}
\pmsynonym{filtration of sigma-algebras}{FiltrationOfsigmaalgebras}
%\pmkeywords{$\sigma$-algebra}
%\pmkeywords{measurable space}
\pmrelated{FilteredProbabilitySpace}
\pmrelated{Filtration}
\pmdefines{natural filtration}

% almost certainly you want these
\usepackage{amssymb}
\usepackage{amsmath}
\usepackage{amsfonts}

% used for TeXing text within eps files
%\usepackage{psfrag}
% need this for including graphics (\includegraphics)
%\usepackage{graphicx}
% for neatly defining theorems and propositions
\usepackage{amsthm}
% making logically defined graphics
%%%\usepackage{xypic}

% there are many more packages, add them here as you need them

% define commands here
\newtheorem*{theorem*}{Theorem}
\newtheorem*{lemma*}{Lemma}
\newtheorem*{corollary*}{Corollary}
\newtheorem*{definition*}{Definition}
\newtheorem{theorem}{Theorem}
\newtheorem{lemma}{Lemma}
\newtheorem{corollary}{Corollary}
\newtheorem{definition}{Definition}

\begin{document}
\PMlinkescapeword{index set}
\PMlinkescapeword{represents}
\PMlinkescapeword{clear}
\PMlinkescapeword{conversely}
\PMlinkescapeword{generates}
\PMlinkescapeword{satisfy}
\PMlinkescapeword{interval}
\PMlinkescapeword{right limits}
For an ordered set $T$, a filtration of \PMlinkname{$\sigma$-algebras}{SigmaAlgebra} $(\mathcal{F}_t)_{t\in T}$ is a collection of $\sigma$-algebras on an underlying set $\Omega$, satisfying $\mathcal{F}_s\subseteq\mathcal{F}_t$ for all $s<t$ in $T$. Here, $t$ is understood as the time variable, taking values in the index set $T$, and $\mathcal{F}_t$ represents the collection of all events observable up until time $t$. The index set is usually a subset of the real numbers, with common examples being $T=\mathbb{Z}_+$ for discrete-time and $T=\mathbb{R}_+$ for continuous-time scenarios.
The collection $(\mathcal{F}_t)_{t\in T}$ is a filtration on a measurable space $(\Omega,\mathcal{F})$ if $\mathcal{F}_t\subseteq\mathcal{F}$ for every $t$. If, furthermore, there is a probability measure defined on the underlying measurable space then this gives a filtered probability space.
The alternative notation $(\mathcal{F}_t,t\in T)$ is often used for the filtration or, when the index set $T$ is clear from the context, simply $(\mathcal{F}_t)$ or ${\bf F}$.

Filtrations are widely used for studying stochastic processes, where a process $X_t$ with time ranging over the set $T$ is said to be adapted to the filtration if $X_t$ is an $\mathcal{F}_t$-measurable random variable for each time $t$.

Conversely, any stochastic process $(X_t)_{t\in T}$ generates a filtration. Let $\mathcal{F}_t$ be the smallest $\sigma$-algebra with respect to which $X_s$ is measurable for all $s\le t$,
\begin{equation*}
\mathcal{F}_t=\sigma\left(X_s:s\le t\right).
\end{equation*}
This defines the smallest filtration to which $X$ is adapted, known as the \emph{natural filtration} of $X$.

Given a filtration, there are various limiting $\sigma$-algebras which can be defined. The values at plus and minus infinity are
\begin{equation*}
\mathcal{F}_\infty = \sigma\left(\bigcup_t\mathcal{F}_t\right),\
\mathcal{F}_{-\infty} = \bigcap_t\mathcal{F}_t,
\end{equation*}
which satisfy $\mathcal{F}_{-\infty}\subseteq\mathcal{F}_t\subseteq\mathcal{F}_\infty$.
In continuous-time, when the index set is an interval of the real numbers, the left and right limits can be defined at any time. They are,
\begin{equation*}
\mathcal{F}_{t+}=\bigcap_{s>t}\mathcal{F}_s,\ \mathcal{F}_{t-}=\sigma\left(\bigcup_{s<t}\mathcal{F}_s\right),
\end{equation*}
except if $t$ is the maximum of $T$ it is often convenient to set $\mathcal{F}_{t+}=\mathcal{F}_t$ or, if $t$ is the minimum, $\mathcal{F}_{t-}=\mathcal{F}_t$. It is easily verified that $\mathcal{F}_s\subseteq\mathcal{F}_{s+}\subseteq\mathcal{F}_{t-}\subseteq\mathcal{F}_t$ for all times $s<t$.
Furthermore, $(\mathcal{F}_{t+})$ and $(\mathcal{F}_{t-})$ are themselves filtrations.

A filtration is said to be right-continuous if $\mathcal{F}_t=\mathcal{F}_{t+}$ for every $t$ so, in particular, $(\mathcal{F}_{t+})$ is always the smallest right-continuous filtration larger than $(\mathcal{F}_t)$.


%%%%%
%%%%%
\end{document}
