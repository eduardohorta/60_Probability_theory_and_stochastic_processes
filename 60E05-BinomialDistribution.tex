\documentclass[12pt]{article}
\usepackage{pmmeta}
\pmcanonicalname{BinomialDistribution}
\pmcreated{2013-03-22 13:03:01}
\pmmodified{2013-03-22 13:03:01}
\pmowner{Mathprof}{13753}
\pmmodifier{Mathprof}{13753}
\pmtitle{binomial distribution}
\pmrecord{17}{33454}
\pmprivacy{1}
\pmauthor{Mathprof}{13753}
\pmtype{Definition}
\pmcomment{trigger rebuild}
\pmclassification{msc}{60E05}
\pmsynonym{Bernoulli distribution}{BinomialDistribution}
\pmsynonym{binomial random variable}{BinomialDistribution}
\pmsynonym{binomial probability function}{BinomialDistribution}
\pmsynonym{Bernoulli random variable}{BinomialDistribution}
\pmrelated{BinomialCoefficient}
\pmrelated{BinomialTheorem}
\pmrelated{BernoulliRandomVariable}

\endmetadata

\usepackage{graphicx}
%%%\usepackage{xypic} 
\usepackage{bbm}
\newcommand{\Z}{\mathbbmss{Z}}
\newcommand{\C}{\mathbbmss{C}}
\newcommand{\R}{\mathbbmss{R}}
\newcommand{\Q}{\mathbbmss{Q}}
\newcommand{\mathbb}[1]{\mathbbmss{#1}}
\newcommand{\figura}[1]{\begin{center}\includegraphics{#1}\end{center}}
\newcommand{\figuraex}[2]{\begin{center}\includegraphics[#2]{#1}\end{center}}
\begin{document}
Consider an experiment with two possible outcomes (success and failure), which happen randomly. Let $p$ be the probability of success. If the experiment is repeated $n$ times, the probability of having exactly $x$ successes is
$$f(x)=\left({n\atop x}\right)p^x(1-p)^{(n-x)}.$$

The distribution function determined by the probability function $f(x)$ is called a \emph{Bernoulli distribution} or \emph{binomial distribution}.

Here are some plots for $f(x)$ with $n=20$ and $p=0.3$, $p=0.5$.
\figuraex{binom10p3}{scale=0.75}
\figuraex{binom10p5}{scale=0.75}

The corresponding distribution function is
$$F(x)=\sum_{k\leq x}\left({n\atop k}\right)p^k q^{n-k}$$
where $q=1-p$. Notice that if we calculate $F(n)$ we get the binomial expansion for $(p+q)^n$, and this is the reason for the distribution being called binomial.

We will use the moment generating function to calculate the mean and variance for the distribution. The mentioned function is
$$G(t)=\sum_{x=0}^n e^{tx}\left({n\atop x}\right)p^x q^{n-x}$$
which simplifies to
$$G(t)=(pe^t+q)^n.$$
Differentiating gives us 
$$G'(t)=n(pe^t+q)^{n-1}pe^t$$
and therefore the mean is
$$\mu = E[X]=G'(0)=np.$$

Now for the variance we need the second derivative
$$G''(t)=n(n-1)(pe^t+q)^{n-2} + n(pe^t+q)^{n-1}pe^t$$
so we get $$E[X^2]=G''(0)=n(n-1)p^2 + np$$
and finally the variance (recall $q=1-p$):
$$\sigma^2 = E[X^2] - E[X]^2 = npq.$$


For large values of $n$, the binomial coefficients are hard to compute, however in this cases we can use either the Poisson distribution or the normal distribution to approximate the probabilities.
%%%%%
%%%%%
\end{document}
