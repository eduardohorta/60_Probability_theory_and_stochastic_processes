\documentclass[12pt]{article}
\usepackage{pmmeta}
\pmcanonicalname{BooleInequalityProofOf}
\pmcreated{2013-03-22 15:47:18}
\pmmodified{2013-03-22 15:47:18}
\pmowner{Bunder}{13010}
\pmmodifier{Bunder}{13010}
\pmtitle{Boole inequality, proof of}
\pmrecord{6}{37748}
\pmprivacy{1}
\pmauthor{Bunder}{13010}
\pmtype{Proof}
\pmcomment{trigger rebuild}
\pmclassification{msc}{60A99}

% this is the default PlanetMath preamble.  as your knowledge
% of TeX increases, you will probably want to edit this, but
% it should be fine as is for beginners.

% almost certainly you want these
\usepackage{amssymb}
\usepackage{amsmath}
\usepackage{amsfonts}

% used for TeXing text within eps files
%\usepackage{psfrag}
% need this for including graphics (\includegraphics)
%\usepackage{graphicx}
% for neatly defining theorems and propositions
%\usepackage{amsthm}
% making logically defined graphics
%%%\usepackage{xypic}

% there are many more packages, add them here as you need them

% define commands here
\begin{document}
Let $\{B_1, B_2, \cdots \}$ be a sequence defined by:

$$ B_i = A_i \setminus \bigcup_{k=1}^{i-1} A_k$$

Clearly $B_i \in \mathcal{F}, \forall i \in \mathbb{N}$, since $\mathcal{F}$ is  $\sigma$-algebra, they are a disjoint family and :

$$ \bigcup_{n=1}^{i} A_n = \bigcup_{n=1}^{i} B_n ,  \forall i \in \mathbb{N}$$

and since $P$ is a measure over $\mathcal{F}$ it follows that :

$$ P(\bigcup_{n=1}^{i} B_n) = \sum_{n=1}^{i} P(B_n) ,  \forall i \in \mathbb{N}$$

% $$ = \sum_{n=1}^{i} P(A_n \setminus \bigcup_{k=1}^{n-1} A_k)$$

Clearly $ B_i \subset A_i$ , then $P(B_i) \leq P(A_i)$ because measures are \PMlinkid{monotonic}{4460}, then it follows that :

$$ P(\bigcup_{n=1}^{i} B_n) \leq \sum_{n=1}^{i} P(A_n),  \forall i \in \mathbb{N} $$

finally taking $n \to \infty$ :

$$ P(\bigcup_{n=1}^{\infty} A_n) = P(\bigcup_{n=1}^{\infty} B_n) \leq \sum_{n=1}^{\infty} P(A_n) $$

the latter is valid because the measure continuity , and is the proof of the theorem
%%%%%
%%%%%
\end{document}
