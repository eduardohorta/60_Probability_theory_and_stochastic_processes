\documentclass[12pt]{article}
\usepackage{pmmeta}
\pmcanonicalname{FiniteVariationProcess}
\pmcreated{2013-03-22 18:36:41}
\pmmodified{2013-03-22 18:36:41}
\pmowner{gel}{22282}
\pmmodifier{gel}{22282}
\pmtitle{finite variation process}
\pmrecord{5}{41345}
\pmprivacy{1}
\pmauthor{gel}{22282}
\pmtype{Definition}
\pmcomment{trigger rebuild}
\pmclassification{msc}{60G07}
%\pmkeywords{stochastic process}

% almost certainly you want these
\usepackage{amssymb}
\usepackage{amsmath}
\usepackage{amsfonts}

% used for TeXing text within eps files
%\usepackage{psfrag}
% need this for including graphics (\includegraphics)
%\usepackage{graphicx}
% for neatly defining theorems and propositions
\usepackage{amsthm}
% making logically defined graphics
%%%\usepackage{xypic}

% there are many more packages, add them here as you need them

% define commands here
\newtheorem*{theorem*}{Theorem}
\newtheorem*{lemma*}{Lemma}
\newtheorem*{corollary*}{Corollary}
\newtheorem*{definition*}{Definition}
\newtheorem{theorem}{Theorem}
\newtheorem{lemma}{Lemma}
\newtheorem{corollary}{Corollary}
\newtheorem{definition}{Definition}

\begin{document}
\PMlinkescapeword{term}
\PMlinkescapeword{theory}
\PMlinkescapeword{paths}
\PMlinkescapeword{right limits}
In the theory of stochastic processes, the term \emph{finite-variation process} is used to refer to a process $X_t$ whose paths are right-continuous and have finite total variation over every compact time interval, with probability one. See, for example, the Poisson process.

It can be shown that any function on the real numbers with finite total variation has left and right limits everywhere. Consequently, finite variation processes are always cadlag.
%%%%%
%%%%%
\end{document}
