\documentclass[12pt]{article}
\usepackage{pmmeta}
\pmcanonicalname{UniformcontinuousRandomVariable}
\pmcreated{2013-03-22 11:54:18}
\pmmodified{2013-03-22 11:54:18}
\pmowner{mathcam}{2727}
\pmmodifier{mathcam}{2727}
\pmtitle{uniform (continuous) random variable}
\pmrecord{10}{30525}
\pmprivacy{1}
\pmauthor{mathcam}{2727}
\pmtype{Definition}
\pmcomment{trigger rebuild}
\pmclassification{msc}{60-00}
\pmsynonym{uniform random variable}{UniformcontinuousRandomVariable}
\pmsynonym{rectangular distribution}{UniformcontinuousRandomVariable}
\pmsynonym{uniform distribution}{UniformcontinuousRandomVariable}

\endmetadata

\usepackage{amssymb}
\usepackage{amsmath}
\usepackage{amsfonts}
\usepackage{graphicx}
%%%%\usepackage{xypic}
\begin{document}
A random variable $X$ is said to be a \emph{\PMlinkescapetext{uniform} (\PMlinkescapetext{continuous}) random variable} with parameters \textbf{$a$ and $b$} if its probability density function is given by
\begin{align*}
f_X(x) = \frac{1}{b-a},\quad\quad x \in [a,b],
\end{align*}
and is denoted $X\sim U(a,b)$.\\

Notes:
\begin{enumerate}
\item They are also called \emph{rectangular distributions}, considers that all points in the interval $[a,b]$ have the same mass.
\item $E[X] = \frac{a+b}{2}$
\item $Var[X] = \frac{(b-a)^2}{12}$
\item $M_X(t) = \frac{e^{bt} - e^{at}}{(b-a)t}$

\end{enumerate}
%%%%%
%%%%%
%%%%%
%%%%%
\end{document}
