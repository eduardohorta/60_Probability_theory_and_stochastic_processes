\documentclass[12pt]{article}
\usepackage{pmmeta}
\pmcanonicalname{PropertiesForMeasure}
\pmcreated{2013-03-22 13:45:28}
\pmmodified{2013-03-22 13:45:28}
\pmowner{matte}{1858}
\pmmodifier{matte}{1858}
\pmtitle{properties for measure}
\pmrecord{8}{34460}
\pmprivacy{1}
\pmauthor{matte}{1858}
\pmtype{Theorem}
\pmcomment{trigger rebuild}
\pmclassification{msc}{60A10}
\pmclassification{msc}{28A10}

% this is the default PlanetMath preamble.  as your knowledge
% of TeX increases, you will probably want to edit this, but
% it should be fine as is for beginners.

% almost certainly you want these
\usepackage{amssymb}
\usepackage{amsmath}
\usepackage{amsfonts}

% used for TeXing text within eps files
%\usepackage{psfrag}
% need this for including graphics (\includegraphics)
%\usepackage{graphicx}
% for neatly defining theorems and propositions
%\usepackage{amsthm}
% making logically defined graphics
%%%\usepackage{xypic}

% there are many more packages, add them here as you need them

% define commands here

\newcommand{\sR}[0]{\mathbb{R}}
\newcommand{\sC}[0]{\mathbb{C}}
\newcommand{\sN}[0]{\mathbb{N}}
\newcommand{\sZ}[0]{\mathbb{Z}}
\begin{document}
\newcommand{\cB}[0]{\mathcal{B}}

{\bf Theorem} \cite{folland, mukherjea, cohn, friedman}
Let $(E,\cB,\mu)$ be a measure space, i.e., 
let $E$ be a set, let $\cB$ be a $\sigma$-algebra of sets
in $E$, and let $\mu$ be a measure on $\cB$. 
Then the following properties hold:
\begin{enumerate}
\item {\bf Monotonicity:} If $A,B\in \cB$, and $A\subset B$, then $\mu(A)\le \mu(B)$.
\item If $A,B$ in $\cB$,  $A\subset B$, and $\mu(A)< \infty$, then 
$$\mu(B\setminus A) = \mu(B)-\mu(A).$$
\item For any  $A,B$ in $\cB$, we have
$$ \mu(A\cup B)+\mu(A\cap B) = \mu(A) + \mu(B).$$
\item {\bf Subadditivity:} If $\{A_i\}_{i=1}^\infty$ is a collection of sets from $\cB$, then
$$\mu\big(\bigcup_{i=1}^\infty A_i\big) \le \sum_{i=1}^\infty \mu(A_i).$$
\item {\bf Continuity from below:} 
If $\{A_i\}_{i=1}^\infty$ is a collection of sets from $\cB$ such that
$ A_i\subset A_{i+1}$ for all $i$, then 
$$ \mu\big(\bigcup_{i=1}^\infty A_i\big) = \lim_{i\to \infty} \mu(A_i).$$
\item {\bf Continuity from above:} 
If $\{A_i\}_{i=1}^\infty$ is a collection of sets from $\cB$ such that
$\mu(A_1)<\infty$, and $ A_i\supset A_{i+1}$ for all $i$, then 
$$ \mu\big(\bigcap_{i=1}^\infty A_i\big) = \lim_{i\to \infty}  \mu(A_i).$$
\end{enumerate}

{\bf Remarks} In (2), the assumption $\mu(A)<\infty$ assures 
that the right hand side is always well defined, i.e., not of 
the form $\infty-\infty$. Without the assumption we can prove that
$\mu(B) = \mu(A) + \mu(B\setminus A)$ (see below).
In (3), it is tempting to 
move the term $\mu(A\cap B)$ to the other side for aesthetic reasons. 
However, this is only possible if the term is finite. 

\emph{Proof.} For (1), suppose $A\subset B$. We can then 
write $B$ as the disjoint union $B=A\cup (B\setminus A)$, whence
\begin{eqnarray*}
\mu(B) = \mu(A\cup (B\setminus A)) = \mu(A) + \mu(B\setminus A).
\end{eqnarray*}
Since $\mu(B\setminus A)\ge 0$, the claim follows. 
Property (2) follows from the above equation; since
$\mu(A)<\infty$, we can subtract this quantity from both sides. 
For property (3), we can write
$A\cup B  =A\cup (B\setminus A)$, whence
\begin{eqnarray*}
\mu(A\cup B) &=& \mu(A)+\mu(B\setminus A)\\
    &\le & \mu(A)+\mu(B).
\end{eqnarray*}
If $\mu(A\cup B)$ is infinite, the last inequality must 
be equality, and either of $\mu(A)$ or $\mu(B)$ must be infinite. 
Together with (1), we obtain that if any of the quantities 
$\mu(A), \mu(B), \mu(A\cap B)$ or $\mu(A\cup B)$ is infinite, 
both sides in the equation are infinite and the claim holds. 
We can therefore
  without loss of generality assume that all quantities are finite. 
  From $A\cup B = B\cup (A\setminus B)$, we have
$$ \mu(A\cup B) = \mu(B)+\mu(A\setminus B)$$
and thus
$$ 2\mu(A\cup B) = \mu(A)+  \mu(B) + \mu(A\setminus B) + \mu(B\setminus A).$$
For the last two terms we have
\begin{eqnarray*}
\mu(A\setminus B) + \mu(B\setminus A) &=& \mu( (A\setminus B) \cup (B\setminus A)) \\
&=& \mu( (A\cup B) \setminus (A\cap B) ) \\
&=& \mu(A\cup B) - \mu(A\cap B),
\end{eqnarray*}
where, in the second equality we have used properties for the 
\PMlinkname{symmetric set difference}{SymmetricDifference}, and 
the last equality follows from 
property (2). This completes the proof of 
 property (3).
For property (4), let us define the sequence $\{D_i\}_{i=1}^\infty$ as
$$D_1 = A_1, \,\,\,\,\,\,\,\, D_i = A_i \setminus  \bigcup_{k=1}^{i-1} A_k .$$
Now $D_i\cap D_j=\emptyset$ for $i<j$, so $\{D_i\}$ is a sequence of 
disjoint sets.
Since $\cup_{i=1}^\infty D_i =\cup_{i=1}^\infty A_i$,  and since
$D_i\subset A_i$, we have
\begin{eqnarray*}
\mu(\bigcup_{i=1}^\infty A_i) &=& \mu(\bigcup_{i=1}^\infty D_i) \\
     &=& \sum_{i=1}^\infty \mu(D_i) \\
     &\le& \sum_{i=1}^\infty \mu(A_i),
\end{eqnarray*}
and property (4) follows.


{\bf TODO:} proofs for (5)-(6).

\begin{thebibliography}{9}
\bibitem{folland}
 G.B. Folland, \emph{Real Analysis: Modern Techniques and Their Applications}, 2nd ed, John Wiley \& Sons, Inc., 1999.
 \bibitem{mukherjea}
 A. Mukherjea, K. Pothoven,
 \emph{Real and Functional analysis},
 Plenum press, 1978.
\bibitem{cohn}
D.L. Cohn, \emph{Measure Theory}, Birkh\"auser, 1980.
 \bibitem{friedman}
A. Friedman, 
 \emph{Foundations of Modern Analysis},
Dover publications, 1982. 
 \end{thebibliography}
%%%%%
%%%%%
\end{document}
