\documentclass[12pt]{article}
\usepackage{pmmeta}
\pmcanonicalname{CharacteristicFunction1}
\pmcreated{2013-03-22 13:14:28}
\pmmodified{2013-03-22 13:14:28}
\pmowner{Koro}{127}
\pmmodifier{Koro}{127}
\pmtitle{characteristic function}
\pmrecord{5}{33714}
\pmprivacy{1}
\pmauthor{Koro}{127}
\pmtype{Definition}
\pmcomment{trigger rebuild}
\pmclassification{msc}{60E10}
\pmsynonym{joint characteristic function}{CharacteristicFunction1}
\pmrelated{MomentGeneratingFunction}
\pmrelated{CumulantGeneratingFunction}

\endmetadata

% this is the default PlanetMath preamble.  as your knowledge
% of TeX increases, you will probably want to edit this, but
% it should be fine as is for beginners.

% almost certainly you want these
\usepackage{amssymb}
\usepackage{amsmath}
\usepackage{amsfonts}

% used for TeXing text within eps files
%\usepackage{psfrag}
% need this for including graphics (\includegraphics)
%\usepackage{graphicx}
% for neatly defining theorems and propositions
%\usepackage{amsthm}
% making logically defined graphics
%%%\usepackage{xypic}

% there are many more packages, add them here as you need them

% define commands here
\begin{document}
Let $X$ be a random variable. The \emph{characteristic function} of $X$ is a function
$\varphi_X:\mathbb{R}\rightarrow\mathbb{C}$ defined by
\[\varphi_X(t) = Ee^{itX} = E\cos(tX) + iE\sin(tX),\]
that is, $\varphi_X(t)$ is the expectation of the random variable $e^{itX}$.

Given a random vector $\overline{X}=(X_1,\dots,X_n)$, the characteristic function of
$\overline{X}$, also called \emph{joint characteristic function} of $X_1,\dots,X_n$,
is a function $\varphi_{\overline{X}}:\mathbb{R}^n\rightarrow\mathbb{C}$ defined by
\[\varphi_{\overline{X}}(t)=Ee^{i{\overline{t}}\cdot{\overline{X}}},\]
where $\overline{t} = (t_1,\dots,t_n)$ and
${\overline{t}}\cdot{\overline{X}} = t_1X_1+\cdots+t_nX_n$ (the dot product.)

\textbf{Remark.} If $F_X$ is the distribution function associated to $X$, by the
properties of expectation we have
\[\varphi_X(t) = \int_\mathbb{R} e^{itx}dF_X(x),\]
which is known as the Fourier-Stieltjes transform of $F_X$, and provides an alternate
definition of the characteristic function. From this, it is clear that the characteristic
function depends only on the distribution function of $X$, hence one can define the characteristic
function associated to a distribution even when there is no random variable involved.
This implies that two random variables with the same distribution must have the same
characteristic function. It is also true that each characteristic function determines
a unique distribution; hence the \PMlinkescapetext{name}, since it characterizes the distribution function (see property 6.)

\textbf{Properties}

\begin{enumerate}
\item The characteristic function is bounded by $1$, i.e. $|\varphi_X(t)|\leq 1$ for all $t$;
\item $\varphi_X(0)=1$;
\item $\overline{\varphi_X(t)} = \varphi_X(-t)$, where $\overline{z}$ denotes the complex conjugate of $z$;
\item $\varphi_X$ is uniformly continuous in $\mathbb{R}$;
\item If $X$ and $Y$ are independent random variables, then $\varphi_{X+Y} = \varphi_X\varphi_Y$;
\item The characteristic function determines the distribution function; hence, $\varphi_X=\varphi_Y$ if and only if $F_X = F_Y$.
This is a consequence of the \emph{inversion \PMlinkescapetext{formula}}: Given a random variable $X$
with characteristic function $\varphi$ and distribution function $F$, if $x$ and $y$
are continuity points of $F$ such that $x<y$, then
\[F(x) - F(y) = \frac{1}{2\pi}\lim_{s\rightarrow\infty}\int_{-s}^s
\frac{e^{-itx}-e^{-ity}}{it}\varphi(t)dt;\]
\item A random variable $X$ has a symmetrical distribution (i.e. one such that $F_X = F_{-X}$)
if and only if $\varphi_X(t)\in \mathbb{R}$ for all $t\in \mathbb{R}$;
\item For real numbers $a,b$, $\varphi_{aX+b}(t) = e^{itb}\varphi_X(at)$;
\item If $E|X|^n<\infty$, then $\varphi_X$ has continuous $n$-th derivatives and
\[\frac{d^k\varphi_X}{dt^k}(t) = \varphi_X^{(k)}(t)=\int_\mathbb{R}(ix)^ke^{itx}dF_X(x), \;\; 1\leq k\leq n.\]
Particularly, $\varphi_X^{(k)}(0)=i^kEX^k$; characteristic functions are similar to
moment generating functions in this sense.
\end{enumerate}

Similar properties hold for joint characteristic functions.
Other important result related to characteristic functions is the Paul L\'evy
continuity theorem.
%%%%%
%%%%%
\end{document}
