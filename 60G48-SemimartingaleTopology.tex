\documentclass[12pt]{article}
\usepackage{pmmeta}
\pmcanonicalname{SemimartingaleTopology}
\pmcreated{2013-03-22 18:40:41}
\pmmodified{2013-03-22 18:40:41}
\pmowner{gel}{22282}
\pmmodifier{gel}{22282}
\pmtitle{semimartingale topology}
\pmrecord{6}{41428}
\pmprivacy{1}
\pmauthor{gel}{22282}
\pmtype{Definition}
\pmcomment{trigger rebuild}
\pmclassification{msc}{60G48}
\pmclassification{msc}{60G07}
\pmclassification{msc}{60H05}
\pmsynonym{semimartingale convergence}{SemimartingaleTopology}
%\pmkeywords{semimartingale}
\pmrelated{UcpConvergence}
\pmrelated{UcpConvergenceOfProcesses}

% almost certainly you want these
\usepackage{amssymb}
\usepackage{amsmath}
\usepackage{amsfonts}

% used for TeXing text within eps files
%\usepackage{psfrag}
% need this for including graphics (\includegraphics)
%\usepackage{graphicx}
% for neatly defining theorems and propositions
\usepackage{amsthm}
% making logically defined graphics
%%%\usepackage{xypic}

% there are many more packages, add them here as you need them

% define commands here
\newtheorem*{theorem*}{Theorem}
\newtheorem*{lemma*}{Lemma}
\newtheorem*{corollary*}{Corollary}
\newtheorem*{definition*}{Definition}
\newtheorem{theorem}{Theorem}
\newtheorem{lemma}{Lemma}
\newtheorem{corollary}{Corollary}
\newtheorem{definition}{Definition}

\begin{document}
\PMlinkescapeword{sequence}
\PMlinkescapeword{stochastic integration}
\PMlinkescapeword{implies}
\PMlinkescapeword{sequences}
\PMlinkescapeword{integral}
\PMlinkescapeword{equivalent}
\PMlinkescapeword{bounded}
\PMlinkescapeword{restricted}

Let $(\Omega,\mathcal{F},(\mathcal{F}_t)_{t\in\mathbb{R}_+}),\mathbb{P})$ be a filtered probability space and $(X^n_t)$, $(X_t)$ be cadlag adapted processes.
Then, $X^n$ is said to converge to $X$ in the \emph{semimartingale topology} if $X^n_0\rightarrow X_0$ in probability and
\begin{equation*}
\int_0^t\xi^n\,dX^n-\int_0^t\xi^n\,dX\rightarrow 0
\end{equation*}
in probability as $n\rightarrow\infty$, for every $t>0$ and sequence of simple predictable processes $|\xi^n|\le 1$.

This topology occurs with stochastic calculus where, according to the \PMlinkname{dominated convergence theorem}{DominatedConvergenceForStochasticIntegration}, stochastic integrals converge in the semimartingale topology.
Furthermore, stochastic integration with respect to any \PMlinkname{locally bounded}{LocalPropertiesOfProcesses} predictable process $\xi$ is continuous under the semimartingale topology. That is, if $X^n$ are semimartingales converging to $X$ then $\int\xi\,dX^n$ converges to $\int\xi\,dX$, a fact which does not hold under weaker topologies such as ucp convergence.

Also, for cadlag martingales, $L^1$ convergence implies semimartingale convergence.

It can be shown that semimartingale convergence implies ucp convergence. Consequently, $X^n$ converges to $X$ in the semimartingale topology if and only if
\begin{equation*}
X^n_0-X_0+\int\xi^n\,dX^n-\int\xi^n\,dX\xrightarrow{\rm ucp} 0
\end{equation*}
for all sequences of simple predictable processes $|\xi^n|\le 1$.

The topology is described by a metric as follows. First, let $D^{\rm ucp}(X-Y)$ be a metric defining the ucp topology. For example,
\begin{equation*}
D^{\rm ucp}(X)=\sum_{n=1}^\infty 2^{-n}\mathbb{E}\left[\min\left(1,\sup_{t<n}|X_t|\right)\right].
\end{equation*}
Then, a metric $D^{\rm s}(X-Y)$ for semimartingale convergence is given by
\begin{equation*}
D^{\rm s}(X)=\sup\left\{D^{\rm ucp}(X_0+\xi\cdot X):|\xi|\le 1\textrm{ is simple previsible}\right\}
\end{equation*}
($\xi\cdot X$ denotes the integral $\int\xi\,dX$). This is a proper metric under identification of processes with almost surely equivalent sample paths, otherwise it is a pseudometric.


If $\lambda_n\not=0$ is a sequence of real numbers converging to zero and $X$ is a cadlag adapted process then $\lambda_nX\rightarrow 0$ in the semimartingale topology if and only if
\begin{equation*}
\lambda_n\int_0^t\xi^n\,dX\rightarrow 0
\end{equation*}
in probability, for every $t>0$ and simple predictable processes $|\xi^n|\le 1$.
By the \PMlinkname{sequential characterization of boundedness}{SequentialCharacterizationOfBoundedness}, this is equivalent to the statement that
\begin{equation*}
\left\{\int_0^t\xi\,dX:|\xi|\le 1\textrm{ is simple predictable}\right\}
\end{equation*}
is bounded in probability for every $t>0$.
So, $\lambda_nX\rightarrow 0$ in the semimartingale topology if and only if $X$ is a semimartingale. It follows that semimartingale convergence only becomes a \PMlinkname{vector topology}{TopologicalVectorSpace} when restricted to the space of semimartingales. Then, it can be shown that \PMlinkname{the set of semimartingales is a complete topological vector space}{CompletenessOfSemimartingaleConvergence}.
%%%%%
%%%%%
\end{document}
