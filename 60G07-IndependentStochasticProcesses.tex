\documentclass[12pt]{article}
\usepackage{pmmeta}
\pmcanonicalname{IndependentStochasticProcesses}
\pmcreated{2013-03-22 15:24:36}
\pmmodified{2013-03-22 15:24:36}
\pmowner{CWoo}{3771}
\pmmodifier{CWoo}{3771}
\pmtitle{independent stochastic processes}
\pmrecord{6}{37250}
\pmprivacy{1}
\pmauthor{CWoo}{3771}
\pmtype{Definition}
\pmcomment{trigger rebuild}
\pmclassification{msc}{60G07}

\usepackage{amssymb,amscd}
\usepackage{amsmath}
\usepackage{amsfonts}

% used for TeXing text within eps files
%\usepackage{psfrag}
% need this for including graphics (\includegraphics)
%\usepackage{graphicx}
% for neatly defining theorems and propositions
%\usepackage{amsthm}
% making logically defined graphics
%%%\usepackage{xypic}

% define commands here
\begin{document}
Two stochastic processes $\lbrace X(t)\mid t\in T \rbrace$ and
$\lbrace Y(t)\mid t\in T \rbrace$ are said to be \emph{\PMlinkescapetext{independent}}
if for \emph{any} positive integer $n<\infty$, and \emph{any}
sequence $t_1,\ldots,t_n\in T$, the random vectors
$\boldsymbol{X}:=(X(t_1),\ldots,X(t_n))$ and
$\boldsymbol{Y}:=(Y(t_1),\ldots,Y(t_n))$ are independent.  This means,
for any two $n$-dimensional Borel sets $A,B\subseteq\mathbb{R}^n$,
we have
$$P\Big[\boldsymbol{X}^{-1}(A)\cap\boldsymbol{Y}^{-1}(B)\Big]
=P\Big[\boldsymbol{X}^{-1}(A)\Big]P\Big[\boldsymbol{Y}^{-1}(B)\Big].$$
%%%%%
%%%%%
\end{document}
