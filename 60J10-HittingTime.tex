\documentclass[12pt]{article}
\usepackage{pmmeta}
\pmcanonicalname{HittingTime}
\pmcreated{2013-03-22 14:18:18}
\pmmodified{2013-03-22 14:18:18}
\pmowner{PrimeFan}{13766}
\pmmodifier{PrimeFan}{13766}
\pmtitle{hitting time}
\pmrecord{8}{35764}
\pmprivacy{1}
\pmauthor{PrimeFan}{13766}
\pmtype{Definition}
\pmcomment{trigger rebuild}
\pmclassification{msc}{60J10}
\pmrelated{MarkovChain}
\pmrelated{MeanHittingTime}
\pmdefines{absorption probability}

\usepackage{amssymb}
\usepackage{amsmath}
\usepackage{amsfonts}
\begin{document}
\PMlinkescapeword{chain}
Let $(X_n)_{n\ge 0}$ be a Markov Chain. Then the \emph{hitting time} for a subset $A$ of $I$ (the indexing set) is the random variable:
\begin{displaymath}
H^A = \inf \{n\ge 0 : X_n\in A\}
\end{displaymath}
(set $\inf \varnothing = \infty$).

This can be thought of as the time before the chain is first in a state that is a member of $A$.

Wite $h_i^A$ for the probability  that, starting from $i\in I$ the chain ever hits the set A:
\begin{displaymath}
h_i^A = P(H^A <\infty : X_0 = i)
\end{displaymath}
When A is a closed class, $h_i^A$ is the \emph{absorption probability}.
%%%%%
%%%%%
\end{document}
