\documentclass[12pt]{article}
\usepackage{pmmeta}
\pmcanonicalname{LevyProcess1}
\pmcreated{2013-03-22 17:58:09}
\pmmodified{2013-03-22 17:58:09}
\pmowner{juansba}{18789}
\pmmodifier{juansba}{18789}
\pmtitle{Levy process}
\pmrecord{11}{40475}
\pmprivacy{1}
\pmauthor{juansba}{18789}
\pmtype{Definition}
\pmcomment{trigger rebuild}
\pmclassification{msc}{60G20}

% this is the default PlanetMath preamble.  as your knowledge
% of TeX increases, you will probably want to edit this, but
% it should be fine as is for beginners.

% almost certainly you want these
\usepackage{amssymb}
\usepackage{amsmath}
\usepackage{amsfonts}

% used for TeXing text within eps files
%\usepackage{psfrag}
% need this for including graphics (\includegraphics)
%\usepackage{graphicx}
% for neatly defining theorems and propositions
%\usepackage{amsthm}
% making logically defined graphics
%%%\usepackage{xypic}

% there are many more packages, add them here as you need them

% define commands here
\def\cadlag{c\`adl\`ag }
\def\ito{It\^o }
\def\levy{L\`evy }

\begin{document}
Let $(\Omega,\Psi,P,({\cal F})_{0\le t<\infty})$ be a filtered
probability space. A \levy process on that space is an stochastic
process $L\colon[0,\infty)\times\Omega\rightarrow \Re^n$ that has the
following properties:
\begin{enumerate}
\item $L$ has increments independent of the past: for any $t\ge 0$ and for all $s\ge 0$, $L_{t+s}-L_t$ in independent of ${\cal F}_t$
\item $L$ has stationary increments: if $t\ge s\ge 0$ then $L_t-L_s$ and $L_{t-s}$ have the same distribution. This particulary implies that $L_{t+s}-L_{t}$ and $L_{s}$ have the same distribution.
\item $L$ is continous in probability: for any $t,s\in [0,\infty)$, $\lim_{t\rightarrow s}=X_s$, the limit taken in probability.
\end{enumerate}
\vskip 0.5pc

Some important properties of any \levy processes $L$ are:
\begin{enumerate}
\item There exist a modification of $L$ that has \cadlag paths a.s. (\cadlag paths means that the paths are continuous from the right and that the left limits exist for any $t\ge 0$).
\item $L_t$ is an infinite divisible random variable for all $t\in [0,\infty)$
\item {\it \levy-\ito decomposition}: $L$ can be written as the sum of a diffusion, a continuous Martingale and a pure jump process; i.e:$$L_t=\alpha t+\sigma B_t+\int_{\vert x\vert<1}x\,d\tilde N_t(\cdot,dx)+\int_{\vert x\vert\ge 1}x\,dN_t(\cdot,dx)\quad \hbox{for all $t\ge 0$}$$where $\alpha \in \Re$, $B_t$ is a standard brownian motion. $N$ is defined to be the Poisson random measure of the \levy process (the process that counts the jumps): for any Borel $A$ in $\Re^n$ such that $0 \notin cl(A)$ then $N_t(\cdot,A)\colon= \sum_{0<s\le t}1_{A}(\Delta L_s)$, where $\Delta L_s\colon= L_s-L_{s-}$; and $\tilde N_t(\cdot,A)=N_t(\cdot,A)-tE[N_1(\cdot,A)]$ is the compensated jump process, which is a martingale.
\item {\it \levy-Khintchine formula}: from the previous property it can be shown that for any $t\ge 0$ one has that$$E[e^{iuL_t}]=e^{-t\psi(u)}$$where$$\psi(u)=-i\alpha u+{\sigma^2 \over 2}u^2+\int_{\vert x\vert \ge 1}(1-e^{iux})\,d\nu(x)+\int_{\vert x\vert<1}(1-e^{iux}+iux)\,d\nu(x)$$with $\alpha\in \Re$, $\sigma \in [0,\infty)$ and $\nu$ is a positive, borel, $\sigma$-finite measure called {\it \levy measure}. (Actually $\nu(\cdot)=E[N_1(\cdot,A)]$). The second formula is usually called the \levy exponent or \levy symbol of the process.
\item $L$ is a semimartingale (in the classical sense of being a sum of a finite variation process and a local martingale), so it is a {\it good} integrator, in the stochastic sense.
\end{enumerate}

Some important examples of \levy processes include: the Poisson
Process, the Compound Poisson process, Brownian Motion, Stable Processes,
Subordinators, etc.\vskip 1pc

\noindent{\bf Bibliography}
\begin{itemize}
\item{} Protter, Phillip (1992). Stochastic Integration and Differential Equations. A New Approach. Springer-Verlag, Berlin, Germany.
\item{} Applebaum David (2004). \levy Procesess and Stochastic Calculus.  Cambridge University Press, Cambrigde, UK.
\end{itemize}


\end{document}

%%%%%
%%%%%
\end{document}
