\documentclass[12pt]{article}
\usepackage{pmmeta}
\pmcanonicalname{JointContinuousDensityFunction}
\pmcreated{2013-03-22 11:54:58}
\pmmodified{2013-03-22 11:54:58}
\pmowner{mathcam}{2727}
\pmmodifier{mathcam}{2727}
\pmtitle{joint continuous density function}
\pmrecord{11}{30576}
\pmprivacy{1}
\pmauthor{mathcam}{2727}
\pmtype{Definition}
\pmcomment{trigger rebuild}
\pmclassification{msc}{60A10}
\pmsynonym{joint mass function}{JointContinuousDensityFunction}
\pmsynonym{joint density function}{JointContinuousDensityFunction}
\pmsynonym{joint distribution}{JointContinuousDensityFunction}
%\pmkeywords{statistics}

\usepackage{amssymb}
\usepackage{amsmath}
\usepackage{amsfonts}
\usepackage{graphicx}
%%%%\usepackage{xypic}
\begin{document}
Let $X_1, X_2, ..., X_n$ be $n$ random variables all defined on the same probability space. The \textbf{joint continuous density function} of $X_1, X_2, ..., X_n$, denoted by $f_{X_1, X_2, ..., X_n}(x_1,x_2,...,x_n)$, is the function
$f_{X_1, X_2, ..., X_n}: \mathbb{R}^n \to \mathbb{R}$ such that for any domain $D\subset \mathbb{R}^n$, we have
\begin{align*}
\int_D  {  f_{X_1,X_2,..., X_n}(u_1,u_2,...,u_n) du_1 du_2 ... du_n  } = \text{Prob}({X_1,X_2,...,X_n}\in D)
\end{align*}
\par
As in the case where $n=1$, this function satisfies:\\
\par
\begin{enumerate}
\item $f_{X_1, X_2, ..., X_n}(x_1,...,x_n) \geq  0$    $\forall (x_1,...,x_n)$
\item $\int_{x_1, ... ,x_n}^{} {  f_{X_1, X_2, ..., X_n}(u_1,u_2,...,u_n) du_1 du_2 ... du_n }= 1$
\end{enumerate}
\par
As in the single variable case, $f_{X_1, X_2, ..., X_n}$ does not represent the probability that each of the random variables takes on each of the values.
%%%%%
%%%%%
%%%%%
%%%%%
\end{document}
