\documentclass[12pt]{article}
\usepackage{pmmeta}
\pmcanonicalname{StochasticProcess}
\pmcreated{2013-03-22 14:39:10}
\pmmodified{2013-03-22 14:39:10}
\pmowner{gel}{22282}
\pmmodifier{gel}{22282}
\pmtitle{stochastic process}
\pmrecord{14}{36244}
\pmprivacy{1}
\pmauthor{gel}{22282}
\pmtype{Definition}
\pmcomment{trigger rebuild}
\pmclassification{msc}{60G05}
\pmclassification{msc}{60G60}
\pmsynonym{random process}{StochasticProcess}
\pmrelated{DistributionsOfAStochasticProcess}
\pmdefines{discrete-time process}
\pmdefines{continuous-time process}
\pmdefines{state}
\pmdefines{time series}
\pmdefines{state space}
\pmdefines{random function}
\pmdefines{jointly measurable}

\endmetadata

% this is the default PlanetMath preamble.  as your knowledge
% of TeX increases, you will probably want to edit this, but
% it should be fine as is for beginners.

% almost certainly you want these
\usepackage{amssymb,amscd}
\usepackage{amsmath}
\usepackage{amsfonts}

% used for TeXing text within eps files
%\usepackage{psfrag}
% need this for including graphics (\includegraphics)
%\usepackage{graphicx}
% for neatly defining theorems and propositions
%\usepackage{amsthm}
% making logically defined graphics
%%%\usepackage{xypic}

% there are many more packages, add them here as you need them

% define commands here
\begin{document}
\PMlinkescapeword{measurable}
\PMlinkescapeword{terms}
Let $(\Omega,\mathcal{F},\textbf{P})$ be a probability space. A
\emph{stochastic process} is a collection $$\lbrace X_t \mid t\in T
\rbrace$$ of random variables $X_t$ defined on
$(\Omega,\mathcal{F},\textbf{P})$, where $T$ is a set, called the
index set of the process $\lbrace X_t \mid t\in T \rbrace$. $T$ is
usually (but not always) a subset of $\mathbb{R}$. 
  $X$ is sometimes known as a \emph{random function}.
\par
Given any $t$, the possible values of $X_t$ are called the
\emph{states} of the process at $t$. The set of all states (for all
$t$) of a stochastic process is called its \emph{state space}.
\par
If $T$ is discrete, then the stochastic process is a
\emph{discrete-time process}. If $T$ is an interval of $\mathbb{R}$,
then $\lbrace X_t \mid t\in T \rbrace$ is a \emph{continuous-time
process}. If $T$ can be linearly ordered, then $t$ is also known as
the \emph{time}.
  
A stochastic process $X$ with state space $S$ can be thought of in either of following three ways.

\begin{itemize}
\item As a collection of random variables, $X_t$, for each $t$ in the index set $T$.
\item
As a function in two variables $t\in T$ and $\omega\in\Omega$,
\begin{equation*}
X\colon T\times\Omega\rightarrow S,\ (t,\omega)\mapsto X_t(\omega).
\end{equation*}
The process is said to be measurable, or, jointly measurable if it is $\mathcal{B}(T)\otimes\mathcal{F}/\mathcal{B}(S)$-measurable. Here, $\mathcal{B}(T)$ and $\mathcal{B}(S)$ are the Borel $\sigma$-algebras on $T$ and $S$ respectively.
\item In terms of the sample paths. Each $\omega\in\Omega$ maps to a function
\begin{equation*}
T\rightarrow S,\ t\mapsto X_t(\omega).
\end{equation*}
Many common examples of stochastic processes have sample paths which are either continuous or cadlag.
\end{itemize}

\textbf{Examples}. The following list is some of the most common and
important stochastic processes:
\begin{enumerate}
\item a random walk, as well as its limiting case, a Brownian motion, or a Wiener process
\item Poisson process
\item Markov process; a Markov chain is a Markov process whose state space is discrete
\item renewal process
\end{enumerate}

\textbf{Remarks}.
\begin{itemize}
\item Sometimes, a stochastic process is also called a
\emph{random process}, although a stochastic process is generally
linked to any ``time'' dependent process.  In a random process, the
index set may not be linearly ordered, as in the case of a random
field, where the index set may be, for example, the unit sphere $S^2\subseteq\mathbb{R}^3$.
\item In statistics, a stochastic process is often known as a
\emph{time series}, where the index set is a finite (or at most
countable) ordered sequence of real numbers.
\end{itemize}

%%%%%
%%%%%
\end{document}
