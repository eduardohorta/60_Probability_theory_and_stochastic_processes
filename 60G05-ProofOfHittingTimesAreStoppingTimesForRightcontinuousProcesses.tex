\documentclass[12pt]{article}
\usepackage{pmmeta}
\pmcanonicalname{ProofOfHittingTimesAreStoppingTimesForRightcontinuousProcesses}
\pmcreated{2013-03-22 18:39:12}
\pmmodified{2013-03-22 18:39:12}
\pmowner{gel}{22282}
\pmmodifier{gel}{22282}
\pmtitle{proof of hitting times are stopping times for right-continuous processes}
\pmrecord{5}{41396}
\pmprivacy{1}
\pmauthor{gel}{22282}
\pmtype{Proof}
\pmcomment{trigger rebuild}
\pmclassification{msc}{60G05}
\pmclassification{msc}{60G40}
%\pmkeywords{stopping time}
%\pmkeywords{adapted process}
%\pmkeywords{essential supremum}

% almost certainly you want these
\usepackage{amssymb}
\usepackage{amsmath}
\usepackage{amsfonts}

% used for TeXing text within eps files
%\usepackage{psfrag}
% need this for including graphics (\includegraphics)
%\usepackage{graphicx}
% for neatly defining theorems and propositions
\usepackage{amsthm}
% making logically defined graphics
%%%\usepackage{xypic}

% there are many more packages, add them here as you need them

% define commands here
\newtheorem*{theorem*}{Theorem}
\newtheorem*{lemma*}{Lemma}
\newtheorem*{corollary*}{Corollary}
\newtheorem*{definition*}{Definition}
\newtheorem{theorem}{Theorem}
\newtheorem{lemma}{Lemma}
\newtheorem{corollary}{Corollary}
\newtheorem{definition}{Definition}

\begin{document}
\PMlinkescapeword{represents}
\PMlinkescapeword{completion}
\PMlinkescapeword{complete}
\PMlinkescapeword{properties}
\PMlinkescapeword{argument}
\PMlinkescapeword{closed}
\PMlinkescapeword{implies}
\PMlinkescapeword{finite}
\PMlinkescapeword{necessary}

Let $(\mathcal{F})_{t\in\mathbb{T}}$ be a \PMlinkname{filtration}{FiltrationOfSigmaAlgebras} on the measurable space $(\Omega,\mathcal{F})$, It is assumed that $\mathbb{T}$ is a closed subset of $\mathbb{R}$ and that $\mathcal{F}_t$ is universally complete for each $t\in\mathbb{T}$.

Let $X$ be a right-continuous and adapted process taking values in a metric space $E$ and $S\subseteq E$ closed.
We show that
\begin{equation*}
\tau=\inf\left\{t\in\mathbb{T}:X_t\in S\right\}
\end{equation*}
is a stopping time. Assuming $S$ is nonempty and defining the continuous function $d_S(x)\equiv\inf\{d(x,y)\colon y\in S\}$, then $\tau$ is the first time at which the right-continuous process $Y_t=d_S(X_t)$ hits $0$.

Let us start by supposing that $\mathbb{T}$ has a minimum element $t_0$.

If $\mathbb{P}$ is a probability measure on $(\Omega,\mathcal{F})$ and $\mathcal{F}^{\mathbb{P}}_t$ represents the \PMlinkname{completion}{CompleteMeasure} of the $\sigma$-algebra $\mathcal{F}_t$ with respect to $\mathbb{P}$, then it is enough to show that $\tau$ is an $(\mathcal{F}^{\mathbb{P}}_t)$-stopping time. By the universal completeness of $\mathcal{F}_t$ it would then follow that
\begin{equation*}
\left\{\tau\le t\right\}\in\bigcap_{\mathbb{P}}\mathcal{F}^{\mathbb{P}}_t=\mathcal{F}_t
\end{equation*}
for every $t\in\mathbb{T}$ and, therefore, that $\tau$ is a stopping time.
So, by replacing $\mathcal{F}_t$ by $\mathcal{F}^{\mathbb{P}}_t$ if necessary, we may assume without loss of generality that $\mathcal{F}_t$ is complete with respect to the probability measure $\mathbb{P}$ for each $t$.

Let $\mathcal{T}$ consist of the set of measurable times $\sigma\colon\Omega\rightarrow\mathbb{T}\cup\{\infty\}$ such that $\{\sigma<t\}\in\mathcal{F}_t$ for every $t$ and that $\sigma\le\tau$. Then let $\sigma^*$ be the essential supremum of $\mathcal{T}$.
That is, $\sigma^*$ is the smallest (up to sets of zero probability) random variable taking values in $\mathbb{R}\cup\{\pm\infty\}$ such that $\sigma^*\ge\sigma$ (almost surely) for all $\sigma\in\mathcal{T}$.

Then, by the properties of the essential supremum, there is a countable sequence $\sigma_n\in\mathcal{T}$ such that $\sigma^*=\sup_n\sigma_n$. It follows that $\sigma^*\in\mathcal{T}$.

For any $n=1,2,\ldots$ set
\begin{equation*}
\sigma_1=\inf\left\{t\in\mathbb{T}:t\ge\sigma^*,Y_t< 1/n\right\}.
\end{equation*}
Clearly, $\sigma_1\le\tau$ and, choosing any countable dense subset $A$ of $\mathbb{T}$, the right-continuity of $Y$ gives
\begin{equation*}
\{\sigma_1<t\}=\bigcup_{\substack{s<t,\\ s\in A}}\{\sigma^*< s,Y_s<1/n\}\in\mathcal{F}_t.
\end{equation*}
So, $\sigma_1\in\mathcal{T}$, which implies that $\sigma_1\le\sigma^*$ with probability one.
However, by the right-continuity of $Y$, $\sigma_1>\sigma^*$ whenever $\sigma^*$ is finite and $Y_{\sigma^*}>1/n$, so
\begin{equation*}
\mathbb{P}(\sigma^*<\infty,Y_{\sigma^*}>0)\le\sum_n\mathbb{P}(\sigma^*<\infty,Y_{\sigma^*}>1/n)=0.
\end{equation*}
This shows that $Y_{\sigma^*}=0$ and therefore $\sigma^*\ge\tau$ whenever $\sigma^*<\infty$. So, $\sigma^*=\tau$ almost surely and $\tau\in\mathcal{T}$ giving,
\begin{equation*}
\left\{\tau\le t\right\}=\{\tau<t\}\cup\{Y_t=0\}\in\mathcal{F}_t.
\end{equation*}
So, $\tau$ is a stopping time.

Finally, suppose that $\mathbb{T}$ does not have a minimum element. Choosing a sequence $t_n\rightarrow-\infty$ in $\mathbb{T}$ then the above argument shows that
\begin{equation*}
\tau_n=\inf\left\{t\in\mathbb{T}:t\ge t_n, Y_t=0\right\}
\end{equation*}
are stopping times so,
\begin{equation*}
\left\{\tau\le t\right\}=\bigcup_n\left\{\tau_n\le t\right\}\in\mathcal{F}_t
\end{equation*}
as required.

%%%%%
%%%%%
\end{document}
