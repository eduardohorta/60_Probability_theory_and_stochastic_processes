\documentclass[12pt]{article}
\usepackage{pmmeta}
\pmcanonicalname{OuterMeasure}
\pmcreated{2013-03-22 13:45:20}
\pmmodified{2013-03-22 13:45:20}
\pmowner{mathcam}{2727}
\pmmodifier{mathcam}{2727}
\pmtitle{outer measure}
\pmrecord{6}{34456}
\pmprivacy{1}
\pmauthor{mathcam}{2727}
\pmtype{Definition}
\pmcomment{trigger rebuild}
\pmclassification{msc}{60A10}
\pmclassification{msc}{28A10}
\pmrelated{CaratheodorysExtensionTheorem}
\pmrelated{CaratheodorysLemma}
\pmrelated{ProofOfCaratheodorysExtensionTheorem}

% this is the default PlanetMath preamble.  as your knowledge
% of TeX increases, you will probably want to edit this, but
% it should be fine as is for beginners.

% almost certainly you want these
\usepackage{amssymb}
\usepackage{amsmath}
\usepackage{amsfonts}

% used for TeXing text within eps files
%\usepackage{psfrag}
% need this for including graphics (\includegraphics)
%\usepackage{graphicx}
% for neatly defining theorems and propositions
%\usepackage{amsthm}
% making logically defined graphics
%%%\usepackage{xypic}

% there are many more packages, add them here as you need them

% define commands here

\newcommand{\sR}[0]{\mathbb{R}}
\newcommand{\sC}[0]{\mathbb{C}}
\newcommand{\sN}[0]{\mathbb{N}}
\newcommand{\sZ}[0]{\mathbb{Z}}
\begin{document}
{\bf Definition} \cite{mukherjea, friedman, folland}
Let $X$ be a set, and let $\mathcal{P}(X)$ be the
power set of $X$. An \emph{outer measure} on $X$ is a function
$\mu^\ast:\mathcal{P}(X)\to [0,\infty]$ satisfying the properties
\begin{enumerate}
\item $\mu^\ast(\emptyset)=0$. 
\item If $A\subset B$ are subsets in $X$, then $\mu^\ast(A)\le \mu^\ast(B)$. 
\item If $\{A_i\}$ is a countable collection of subsets of $X$, 
then 
$$ \mu^\ast(\bigcup_i A_i) \le \sum_i \mu^\ast (A_i).$$
\end{enumerate}

Here, we can make two remarks. First, from (1) and (2), it follows
that $\mu^\ast$ is a positive function on $\mathcal{P}(X)$. Second, 
property (3) also holds for any finite collection of subsets since
we can always append an infinite sequence of empty sets to 
such a collection.

\begin{thebibliography}{9}
 \bibitem{mukherjea}
 A. Mukherjea, K. Pothoven,
 \emph{Real and Functional analysis},
 Plenum press, 1978.
 \bibitem{friedman}
A. Friedman, 
 \emph{Foundations of Modern Analysis},
Dover publications, 1982. 
\bibitem{folland}
 G.B. Folland, \emph{Real Analysis: Modern Techniques and Their Applications}, 2nd ed, John Wiley \& Sons, Inc., 1999.
 \end{thebibliography}
%%%%%
%%%%%
\end{document}
