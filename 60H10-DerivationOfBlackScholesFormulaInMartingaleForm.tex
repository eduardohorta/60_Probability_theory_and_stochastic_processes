\documentclass[12pt]{article}
\usepackage{pmmeta}
\pmcanonicalname{DerivationOfBlackScholesFormulaInMartingaleForm}
\pmcreated{2013-03-22 17:20:37}
\pmmodified{2013-03-22 17:20:37}
\pmowner{stevecheng}{10074}
\pmmodifier{stevecheng}{10074}
\pmtitle{derivation of Black-Scholes formula in martingale form}
\pmrecord{5}{39699}
\pmprivacy{1}
\pmauthor{stevecheng}{10074}
\pmtype{Derivation}
\pmcomment{trigger rebuild}
\pmclassification{msc}{60H10}
\pmclassification{msc}{91B28}
\pmrelated{BlackScholesFormula}

\endmetadata

% The standard font packages
\usepackage{amssymb}
\usepackage{amsmath}
\usepackage{amsfonts}

% For neatly defining theorems and definitions
%\usepackage{amsthm}

% Including EPS/PDF graphics (\includegraphics)
%\usepackage{graphicx}

% Making matrix-based graphics
%%%\usepackage{xypic}

% Enumeration lists with different styles
%\usepackage{enumerate}

% Set up the theorem environments
%\newtheorem{thm}{Theorem}
%\newtheorem*{thm*}{Theorem}

\newcommand{\defnterm}[1]{\emph{#1}}

% The standard number systems
\newcommand{\complex}{\mathbb{C}}
\newcommand{\real}{\mathbb{R}}
\newcommand{\rat}{\mathbb{Q}}
\newcommand{\nat}{\mathbb{N}}
\newcommand{\intset}{\mathbb{Z}}

% Absolute values and norms
% Normal, wide, and big versions of the delimeters
\newcommand{\abs}[1]{\lvert#1\rvert}
\newcommand{\absW}[1]{\left\lvert#1\right\rvert}
\newcommand{\absB}[1]{\Bigl\lvert#1\Bigr\rvert}
\newcommand{\norm}[1]{\lVert#1\rVert}
\newcommand{\normW}[1]{\left\lVert#1\right\rVert}
\newcommand{\normB}[1]{\Bigl\lVert#1\Bigr\rVert}

% Inverse functions
\newcommand{\inv}[1]{{#1}^{-1}}

% Differentiation operators
\newcommand{\od}[2]{\frac{d #1}{d #2}}
\newcommand{\pd}[2]{\frac{\partial #1}{\partial #2}}
\newcommand{\pdd}[2]{\frac{\partial^2 #1}{\partial #2}}
\newcommand{\ipd}[2]{\partial #1 / \partial #2}

% Differentials on integrals
\newcommand{\dx}{\, dx}
\newcommand{\dt}{\, dt}
\newcommand{\dmu}{\, d\mu}

% Inner products
\newcommand{\ip}[2]{\langle {#1}, {#2} \rangle}

% Complex numbers
\DeclareMathOperator{\zRe}{Re}
\DeclareMathOperator{\zIm}{Im}
\newcommand{\conjug}[1]{\overline{#1}}

% Calligraphic letters
\newcommand{\sF}{\mathcal{F}}
\newcommand{\sD}{\mathcal{D}}

% Standard spaces
\newcommand{\Hilb}{\mathcal{H}}
\newcommand{\Le}{\mathbf{L}}

% Operators and functions occassionally used in my articles
\DeclareMathOperator{\D}{D}
\DeclareMathOperator{\linspan}{span}
\DeclareMathOperator{\rank}{rank}
\DeclareMathOperator{\lindim}{dim}
\DeclareMathOperator{\sinc}{sinc}

% Probability stuff
\newcommand{\PP}{\mathbb{P}}
\newcommand{\E}{\mathbb{E}}

\newcommand{\EQ}{\mathbb{E}^\mathbb{Q}}
\newcommand{\PQ}{\mathbb{Q}}

\begin{document}
\PMlinkescapeword{periods}
\PMlinkescapeword{real}
\PMlinkescapeword{unit}
\PMlinkescapeword{units}
\PMlinkescapeword{thin}
\PMlinkescapeword{term}
\PMlinkescapeword{terms}
\PMlinkescapeword{representation}
\PMlinkescapeword{divisible}
\PMlinkescapeword{specification}
\PMlinkescapeword{terminal}
\PMlinkescapeword{holder}
\PMlinkescapeword{even}

\tableofcontents

\medskip

This entry derives the Black-Scholes formula in martingale form.

The portfolio process $V_t$ representing a stock option
will be shown to satisfy:
\begin{align}\label{eq:port-result}
V_t = e^{-r(T-t)} \, \EQ\bigl[ V_T \mid \sF_t \bigr]\,.
\end{align}
(The quantities appearing here are defined precisely,
in the section on ``Assumptions'' below.)

Equation \eqref{eq:port-result}
can be used in practice to calculate $V_t$ for all times $t$,
because from the specification of a financial contract,
the value of the portfolio at time $T$,
or in other words, its pay-off at time $T$,
will be a known function.
Mathematically speaking, $V_T$ gives the \emph{terminal condition}
for the solution of a stochastic differential equation.

\subsection{Assumptions}

\subsubsection{Asset price}

The asset or stock price $X_t$ is to be modelled by the stochastic
differential equation:
\begin{align}\label{eq:stock}
d X_t = \mu X_t \, dt + \sigma X_t \, dW_t\,,
\end{align}
where $\mu$ and $\sigma > 0$ are constants.

The stochastic process $W_t$ is a standard Brownian motion
adapted to the filtration $\{ \sF_t \}$.

See the main article 
on the \PMlinkname{Black-Scholes formula}{BlackScholesFormula}
for an explanation and justification of this modelling assumption.

\subsubsection{Money-market account}

The money-market account accumulates interest compounded continuously
at a rate of $r$.
It satisfies the stochastic differential equation:
\begin{align}\label{eq:money-sde}
dM_t = rM_t \, dt\,.
\end{align}

This happens to take the same form as an ordinary differential equation,
 for the process $M_t$ has no randomness in it at all,
under the assumption of a fixed interest rate $r$.

The solution is to equation \eqref{eq:money-sde} with initial condition 
$M_0$ is $M_t = M_0 \, e^{rt}$.

\subsubsection{Portfolio process}
The price of the option is derived by following a \emph{replicating portfolio}
consisting of $\Delta_t$ units of the stock $X_t$
and $\Theta_t$ units of the money-market
account.  If $V_t$ denotes the value of this portfolio at time $t$,
then
\begin{align}\label{eq:port}
V_t = \Delta_t \, X_t + \Theta_t \, M_t\,.
\end{align}
A certain ``self-financing condition'' on the portfolio requires that
$V_t$ also satisfy the stochastic differential equation:
\begin{align}\label{eq:port-self-financing}
d V_t = \Delta_t \, dX_t + \Theta_t \, dM_t\,.
\end{align}
This condition essentially says that we cannot input extra amounts
of money out of thin air into our portfolio; we must start with what we have.

Equation \eqref{eq:port-self-financing} is not a mathematically proven statement,
but another modelling assumption, justified by an analogous equation
governing trading in discretized time periods.


\subsection{Derivation}

We first manipulate the
stochastic differential equation \eqref{eq:port}
for the portfolio process $V_t$,
to express it in terms of the Brownian motion $W_t$.
\begin{align*}
dV_t &= \Delta_t \, dX_t + \Theta_t \, r M_t \, dt
& \text{from eq. \eqref{eq:port-self-financing} and \eqref{eq:money-sde}} \\
&= \Delta_t \, dX_t + r\bigl(V_t - \Delta_t X_t\bigr) \, dt
& \text{from eq. \eqref{eq:port}} \\
&= \Delta_t \, \bigl(  \mu X_t \, dt + \sigma X_t \, dW_t \bigr) \\
& \qquad + r \bigl(V_t - \Delta_t X_t \bigr) \, dt 
& \text{from eq. \eqref{eq:stock}} \\
&= rV_t \, dt + \Delta_t X_t \, \bigl( (\mu - r) \, dt + \sigma \, dW_t \bigr) & \text{rearrangement}
\end{align*}

\subsubsection{Change of probability measure}

Define the Brownian motion with drift $\lambda$:
\begin{align}\label{eq:wiener-drift}
\widetilde{W}_t = \lambda t + W_t\,, \quad \lambda = \frac{\mu - r}{\sigma}\,;
\end{align}
so that $d\widetilde{W}_t = \lambda \, dt + dW_t$, and
\begin{align}\label{eq:port-sde}
dV_t = rV_t \, dt + \sigma \, \Delta_t X_t \,  d\widetilde{W}_t\,.
\end{align}

The introduction of the process $\widetilde{W}_t$
is not merely for notational convenience but is mathematically
meaningful.  If the probability space we are working in
is $(\Omega, \sF_T, \PP)$,
and $W_t$, for $0 \leq t \leq T$, is a 
standard Wiener process on $(\Omega, \sF_T, \PP)$,
then $\widetilde{W}_t$ will not be a standard Wiener process
on $(\Omega, \sF_T, \PP)$, but it \emph{will} be a standard Wiener process
under $(\Omega, \sF_T, \PQ)$ \emph{with a different probability measure} 
$\PQ$.

The probability measure $\PQ$ is obtained by Girsanov's theorem.
The exact form for $\PQ$ can be calculated,
but it will not be needed in this derivation.

In finance, $\PQ$ is known as the \emph{risk-neutral measure},
and the quantity $\lambda$ is the \emph{market price of risk}.

\subsubsection{Discounted portfolio process is a martingale}

From equation \eqref{eq:port-sde},
we see that the value of the portfolio grows at the risk-free
interest rate of $r$,
apart from the randomness associated due to the stochastic
differential $d\widetilde{W}_t$.

It is thus reasonable to expect that,
if we normalize the portfolio value amount
by the amount that cash grows due to accumulation of risk-free
interest,
the resulting process, $V_t/M_t$, should have a zero growth rate.
That this is indeed the case can be verified
by a computation with It\^o's formula
--- more specifically, the 
\PMlinkescapetext{product rule} for It\^o integrals:
\begin{align*}
d \left( \frac{V_t}{M_t} \right)
&= d \left(V_t \cdot \frac{1}{M_t} \right) \\
&= \bigl( d V_t \bigr) \, \frac{1}{M_t} + 
V_t \, d \left( \frac{1}{M_t}\right) 
\\
&= r \frac{V_t}{M_t} \, dt + \sigma \Delta_t \frac{X_t}{M_t} \, d\widetilde{W}_t+
V_t \, d \left( \frac{1}{M_t}\right) & \text{from eq. \eqref{eq:port-sde}}
\\
&= r \frac{V_t}{M_t} \, dt + \sigma \Delta_t \frac{X_t}{M_t} \, d\widetilde{W}_t+
\frac{V_t}{M_t} \, dt
& \text{from $\frac{1}{M_t} = \frac{e^{-rt}}{M_0}$.}
\end{align*}

Thus,
\begin{align*}
d\left(\frac{V_t}{M_t}\right) = \sigma \Delta_t \, \frac{X_t}{M_t} \, d\widetilde{W}_t\,.
\end{align*}
Or, in integral form:
\begin{align}\label{eq:port-integral}
\frac{V_{t_1}}{M_{t_1}} = \frac{V_{t_0}}{M_{t_0}} + \int_{t_0}^{t_1} \, \sigma \Delta_t \, \frac{X_t}{M_t} \, d\widetilde{W}_t\,, \quad 0 \leq t_0 \leq t_1 \leq T\,.
\end{align}
Assuming $\Delta_t$ is a $\sF_t$-adapted process
--- where $\{ \sF_t \}$ is the filtration generated by the Brownian
motion $W_t$ (or equivalently $\widetilde{W}_t$) ---
the It\^o integral in equation \eqref{eq:port-integral}
is a martingle under the probability space $(\Omega, \sF_T, \PQ)$.

\subsubsection{Portfolio process as a conditional expectation}

Then by the definition of a martingale,
we have
\[
\frac{V_{t_0}}{M_{t_0}} = \EQ \left[ \frac{V_{t_1}}{M_{t_1}} \mid \sF_{t_0} \right]\,, \quad 0 \leq t_0 \leq t_1 \leq T\,,
\]
where $\EQ[ \cdot \mid \sF_t ]$ denotes the conditional 
expectation, of a random variable
on the measurable space $(\Omega, \sF_T)$, 
under
the probability measure $\PQ$.

In particular, setting $t_0 = t \leq T$ and $t_1 = T$,
and rearranging the factors of $M_t = e^{rt}$,
we obtain the desired result, equation \eqref{eq:port-result}.


\subsection{Existence of solutions}
So far,
we have derived the form of the solution for the portfolio value process 
$V_t$\,, 
\emph{assuming that it exists}.
Actually, if we were to take only
equations \eqref{eq:port} and \eqref{eq:port-self-financing}
as the problem to solve mathematically,
without any reference to the financial motivations,
it is possible to work backwards and deduce the existence
of the solution.

\subsubsection{Proposed construction}

Let $\PQ$ be the risk-neutral probability measure,
and let $U$ be any given $\Le^1(\Omega, \sF_T, \PQ)$ random variable,
representing the \emph{terminal condition}.
Define the family of random variables dependent on time,
\begin{align}\label{eq:port-construct}
V_t = e^{-r(T-t)} \, \EQ[ U \mid \sF_t ]\,, \quad 0 \leq t \leq T\,.
\end{align}
It is easy to verify that, for any $U$,
the process $V_t \, e^{-rt}$ is a martingale
with respect to $\sF_t$, the filtration generated 
by the Wiener process $\widetilde{W}_t$ under the probability
measure $\PQ$.

\subsubsection{Verification}

We now invoke the martingale representation theorem for It\^o processes:
for any martingale $Z_t$, with respect to $\sF_t$ under
the probability measure $\PQ$, 
there exists a $\sF_t$-adapted process $G_t$
such that $Z_t$ has the representation:
\begin{align*}
Z_{t_1} - Z_{t_0} = \int_{t_0}^{t_1} G_t \, d\widetilde{W}_t\,.
\end{align*}

Letting $Z_t = V_t \, e^{-rt}$ and 
comparing with equations \eqref{eq:port-integral}
and \eqref{eq:port},
we are motivated to \emph{define}
the $\sF_t$-adapted processes:
\[
\Delta_t = \frac{G_t \, e^{rt}}{\sigma X_t}\,,
\quad \Theta_t = \frac{V_t - \Delta_t X_t}{M_t} = \frac{Z_t - G_t/\sigma}{M_0}\,.
\]
Then the process $V_t$ constructed
by equation \eqref{eq:port-construct}
trivially satisfies equation \eqref{eq:port}.
And it is a simple matter to check that 
equation \eqref{eq:port-self-financing} holds as well:
\begin{align*}
d V_t = d\bigl( Z_t \, e^{rt} \bigr) 
&= e^{rt} \, dZ_t + re^{rt} \, Z_t \, dt & \text{It\^o's product rule}\\
&= e^{rt} \, G_t \, d\widetilde{W}_t  + r V_t \, dt \\
&= \sigma \Delta_t X_t \, d\widetilde{W}_t + rV_t \, dt \\
&= \Delta_t \, X_t (r \, dt + \sigma d\widetilde{W}_t) 
& \text{add and subtract}
\\
& \qquad + r\bigl( V_t - \Delta_t X_t \bigr) 
& \text{the $dt$ term} \\
&= \Delta_t \, dX_t + r \, \Theta_t \, M_t \, dt\,,
\end{align*}
where in the last equality we have used the SDE for $X_t$
in terms of $d\widetilde{W}_t$ in place of $dW_t$:
\begin{align*}
dX_t = r X_t \, dt + \sigma X_t \, d\widetilde{W}_t\,,
\end{align*}
obtained by substituting 
in equation \eqref{eq:stock},
the differential of equation \eqref{eq:wiener-drift}.

\begin{thebibliography}{6}
\bibitem{Oksendal}
Bernt \O{}ksendal.
\emph{Stochastic Differential Equations,
An Introduction with Applications}, 5th edition. Springer, 1998.
\bibitem{Shreve}
Steven E. Shreve. \emph{Stochastic Calculus for Finance II: 
Continuous-Time Models}. Springer, 2004.
\end{thebibliography}

%%%%%
%%%%%
\end{document}
