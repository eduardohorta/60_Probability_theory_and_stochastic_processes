\documentclass[12pt]{article}
\usepackage{pmmeta}
\pmcanonicalname{6Discussion}
\pmcreated{2014-04-22 15:13:32}
\pmmodified{2014-04-22 15:13:32}
\pmowner{rspuzio}{6075}
\pmmodifier{rspuzio}{6075}
\pmtitle{6. Discussion}
\pmrecord{2}{88090}
\pmprivacy{1}
\pmauthor{rspuzio}{6075}
\pmtype{Feature}

% this is the default PlanetMath preamble.  as your knowledge
% of TeX increases, you will probably want to edit this, but
% it should be fine as is for beginners.

% almost certainly you want these
\usepackage{amssymb}
\usepackage{amsmath}
\usepackage{amsfonts}

% need this for including graphics (\includegraphics)
\usepackage{graphicx}
% for neatly defining theorems and propositions
\usepackage{amsthm}

% making logically defined graphics
%\usepackage{xypic}
% used for TeXing text within eps files
%\usepackage{psfrag}

% there are many more packages, add them here as you need them

% define commands here

\DeclareMathOperator{\Hom}{Hom}

\newcommand{\vecify}{{\mathcal V}}
\newcommand{\Act}{{A}}
\newcommand{\act}{{a}}
\newcommand{\Sit}{{S}}
\newcommand{\occ}{{v}}
\newcommand{\univ}{{\mathbf D}}
\newcommand{\uout}{{d_{out}}}
\newcommand{\uin}{{d_{in}}}
\newcommand{\mangle}{{\mathbf C}}

\newcommand{\psheaf}{{\mathcal F}}
\newcommand{\scat}{{\mathtt{Stoch}}}
\newcommand{\subs}{{\mathtt{Sys}}}
\newcommand{\mcat}{{\mathtt{Meas}}}
\newcommand{\eop}{{$\blacksquare$}}
\newcommand{\eod}{{${}$\\}}
\newcommand{\bra}{{\langle}}
\newcommand{\ket}{{\rangle}}

\newcommand{\cN}{{\mathcal N}}
\newcommand{\bR}{{\mathbb R}}
\newcommand{\fm}{{\mathfrak m}}
\newcommand{\cP}{{\mathcal P}}

\newtheorem{thm}{Theorem}
\newtheorem{prop}[thm]{Proposition}
\newtheorem{cor}[thm]{Corollary}

\theoremstyle{remark}
\newtheorem{eg}{Example}
\newtheorem{rem}{Remark}
\newtheorem{defn}{Definition}

\begin{document}
This paper developed techniques for analyzing the internal 
structure of distributed measurements. We introduced 
entanglement, which quantifies the extent to which a 
measurement is indecomposable. Entanglement can be shown 
to quantify context-dependence. Moreover, positive entanglement 
is necessary for a system to generate more information than the 
sum of its subsystems.  Along the way, we constructed the quale,
which geometrically represents the compositional structure of a 
distributed measurement. The information-theoretic approach 
developed here is dual, in a precise sense, to the algorithmic 
perspective on computation. Studying duals $\fm^\natural$ 
instead of mechanisms $\fm$ shifts the focus from \emph{what} 
the algorithm does to \emph{how} it does it: instead of 
analyzing rules we analyze functional dependencies. 

The intuition driving the paper is that the structure presheaf 
$\psheaf$ is an information-theoretic analogue of a tangent 
space. A particle moving in a manifold $X$ defines a vector 
field -- a section of the tangent space to $X$, which is a 
sheaf. The tangent vector at a point depends on the particle's 
location at ``nearby time-points'': it is computed by taking 
the limit of difference in positions at $t$ and $t+h$ as 
$h\rightarrow 0$. Similarly, a system performing a measurement 
generates a quale, a section of the structure presheaf 
consisting of ``nearby counterfactuals''. The quale is computed 
by applying Bayes' rule to determine which inputs could have led
to the output.\footnote{A counterfactual input is ``nearby'' to 
an output if it causes (leads to) that output.} How far this 
analogy can be developed remains to be seen.

Entanglement can be loosely considered as an 
information-theoretic analogue of curvature: the extent to 
which interactions within a system ``warp'' sections of 
$\psheaf$ away from a product structure. A related approach 
to geometrically analyzing the complexity of interactions was 
proposed in \cite{ay:06}. In fact, this project began as an 
attempt to reformulate \cite{bt:09} in terms of sheaf 
cohomology using ideas from \cite{ay:06}. We failed at the 
first step since the structure presheaf is not a sheaf. 
However, the failure was instructive since it is precisely 
the \emph{obstruction} to forming a sheaf that is of interest 
since it is the obstruction (entanglement) that quantifies 
indecomposability and context-dependence, and only systems 
whose measurements are entangled are able to generate more 
information than the sum of their subsystems.

\begin{thebibliography}{10}
%\providecommand{\bibitemdeclare}[2]{}
\providecommand{\urlprefix}{Available at }
\providecommand{\url}[1]{\texttt{#1}}
\providecommand{\href}[2]{\texttt{#2}}
\providecommand{\urlalt}[2]{\href{#1}{#2}}
\providecommand{\doi}[1]{doi:\urlalt{http://dx.doi.org/#1}{#1}}
\providecommand{\bibinfo}[2]{#2}

%\bibitemdeclare{incollection}{ay:06}
\bibitem{ay:06}
\bibinfo{author}{N~Ay}, \bibinfo{author}{E~Olbrich},
  \bibinfo{author}{N~Bertschinger} \& \bibinfo{author}{J~Jost}
  (\bibinfo{year}{2006}): \emph{\bibinfo{title}{A unifying framework for
  complexity measures of finite systems}}.
\newblock In: {\sl \bibinfo{booktitle}{Proceedings of ECCS06, European Complex
  Systems Society}}, \bibinfo{address}{Oxford, UK}, pp.
  \bibinfo{pages}{ECCS06--174}.
  
%\bibitemdeclare{article}{bt:09}
\bibitem{bt:09}
\bibinfo{author}{David Balduzzi} \& \bibinfo{author}{Giulio Tononi}
  (\bibinfo{year}{2009}): \emph{\bibinfo{title}{Qualia: the geometry of
  integrated information}}.
\newblock {\sl \bibinfo{journal}{PLoS Comput Biol}}
  \bibinfo{volume}{5}(\bibinfo{number}{8}), p. \bibinfo{pages}{e1000462},
  \doi{10.1371/journal.pcbi.1000462}.

\end{thebibliography}
\end{document}
