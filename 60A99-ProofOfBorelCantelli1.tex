\documentclass[12pt]{article}
\usepackage{pmmeta}
\pmcanonicalname{ProofOfBorelCantelli1}
\pmcreated{2013-03-22 14:28:16}
\pmmodified{2013-03-22 14:28:16}
\pmowner{kshum}{5987}
\pmmodifier{kshum}{5987}
\pmtitle{proof of Borel-Cantelli 1}
\pmrecord{6}{35992}
\pmprivacy{1}
\pmauthor{kshum}{5987}
\pmtype{Proof}
\pmcomment{trigger rebuild}
\pmclassification{msc}{60A99}

% this is the default PlanetMath preamble.  as your knowledge
% of TeX increases, you will probably want to edit this, but
% it should be fine as is for beginners.

% almost certainly you want these
\usepackage{amssymb}
\usepackage{amsmath}
\usepackage{amsfonts}

% used for TeXing text within eps files
%\usepackage{psfrag}
% need this for including graphics (\includegraphics)
%\usepackage{graphicx}
% for neatly defining theorems and propositions
%\usepackage{amsthm}
% making logically defined graphics
%%%\usepackage{xypic}

% there are many more packages, add them here as you need them

% define commands here
\begin{document}
Let $B_k$ be the event $\cup_{i=k}^\infty A_i$ for
$k=1,2,\ldots,$. If $x$ is in the event $A_i$'s i.o., then $x\in
B_k$ for all $k$. So $x\in \cap_{k=1}^\infty B_k$.

Conversely, if $x\in B_k$ for all $k$, then we can show that $x$
is in $A_i$'s i.o. Indeed, $x\in B_1 = \cup_{i=1}^\infty A_i$
means that $x\in A_{j(1)}$ for some $j(1)$. However $x\in
B_{j(1)+1}$ implies that $x\in A_{j(2)}$ for some $j(2)$ that is
strictly larger than $j(1)$. Thus we can produce an infinite
sequence of integer $j(1)<j(2)<j(3)<\ldots$ such that $x\in
A_{j(i)}$ for all $i$.

Let $E$ be the event $\{x:\, x\in A_i \mbox{ i.o.}\}$. We have
\[
 E = \bigcap_{k=1}^\infty \bigcup_{i=k}^\infty A_i.
\]

From $E\subseteq B_k$ for all $k$, it follows that $P(E)\leq
P(B_k)$ for all $k$. By union bound, we know that $P(B_k)\leq
\sum_{i=k}^\infty P(A_i)$. So $P(B_k)\rightarrow 0$, by the
hypothesis that $\sum_{i=1}^\infty P(A_i)$ is finite. Therefore, $P(E)=0$.
%%%%%
%%%%%
\end{document}
