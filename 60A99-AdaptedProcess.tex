\documentclass[12pt]{article}
\usepackage{pmmeta}
\pmcanonicalname{AdaptedProcess}
\pmcreated{2013-03-22 16:16:43}
\pmmodified{2013-03-22 16:16:43}
\pmowner{rspuzio}{6075}
\pmmodifier{rspuzio}{6075}
\pmtitle{adapted process}
\pmrecord{19}{38390}
\pmprivacy{1}
\pmauthor{rspuzio}{6075}
\pmtype{Definition}
\pmcomment{trigger rebuild}
\pmclassification{msc}{60A99}
\pmclassification{msc}{60G07}
\pmsynonym{adapted}{AdaptedProcess}

% this is the default PlanetMath preamble.  as your knowledge
% of TeX increases, you will probably want to edit this, but
% it should be fine as is for beginners.

% almost certainly you want these
\usepackage{amssymb}
\usepackage{amsmath}
\usepackage{amsfonts}

% used for TeXing text within eps files
%\usepackage{psfrag}
% need this for including graphics (\includegraphics)
%\usepackage{graphicx}
% for neatly defining theorems and propositions
%\usepackage{amsthm}
% making logically defined graphics
%%%\usepackage{xypic}

% there are many more packages, add them here as you need them

% define commands here

\begin{document}
Let $\lbrace X_t \mid t\in T\rbrace$ be a stochastic process defined on a probability space $(\Omega,\mathcal{F},P)$ and $\lbrace \mathcal{F}_t \mid t\in T\rbrace$ a filtration (an increasing sequence of sigma subalgebras of $\mathcal{F}$), where $T$ is a linearly ordered subset of $\mathbb{R}$ with a minimum $t_0$.  Then the process $\lbrace X_t\rbrace$ is said to be \emph{adapted to} the filtration $\lbrace \mathcal{F}_t\rbrace$ if for each $t\ge t_0$, $X_t$ is \PMlinkname{$\mathcal{F}_t$-measurable}{MathcalFMeasurableFunction}:
$$X_t^{-1}(B)\in \mathcal{F}_t\mbox{ for each Borel set }B\in\mathbb{R}.$$
A stochastic process is an \emph{adapted process} if it is adapted to some filtration.
%%%%%
%%%%%
\end{document}
