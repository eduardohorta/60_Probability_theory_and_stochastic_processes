\documentclass[12pt]{article}
\usepackage{pmmeta}
\pmcanonicalname{FellerProcess}
\pmcreated{2013-03-22 16:12:40}
\pmmodified{2013-03-22 16:12:40}
\pmowner{mcarlisle}{7591}
\pmmodifier{mcarlisle}{7591}
\pmtitle{Feller process}
\pmrecord{6}{38307}
\pmprivacy{1}
\pmauthor{mcarlisle}{7591}
\pmtype{Definition}
\pmcomment{trigger rebuild}
\pmclassification{msc}{60J35}
%\pmkeywords{Feller Markov random stochastic process semigroup}
\pmdefines{Feller semigroup}
\pmdefines{Feller transition function}
\pmdefines{Feller process}
\pmdefines{LCCB}

\endmetadata

% this is the default PlanetMath preamble.  as your knowledge
% of TeX increases, you will probably want to edit this, but
% it should be fine as is for beginners.

% almost certainly you want these
\usepackage{amssymb}
\usepackage{amsmath}
\usepackage{amsfonts}

% used for TeXing text within eps files
%\usepackage{psfrag}
% need this for including graphics (\includegraphics)
%\usepackage{graphicx}
% for neatly defining theorems and propositions
%\usepackage{amsthm}
% making logically defined graphics
%%%\usepackage{xypic}

% there are many more packages, add them here as you need them

% define commands here

\begin{document}
Let $E$ be a LCCB space (locally compact with a countable base; usually a subset of $\mathbb{R}^n$ for some $n \in \mathbb{N}$) and $C_0(E) = C_0(E, \mathbb{R})$ be the space of continuous functions on $E$ that vanish at infinity.  (We may write $C_0$ as shorthand.)  A \emph{Feller semigroup} on $C_0(E)$ is a family of positive linear operators $T_t, t \geq 0$, on $C_0(E)$ such that 

\begin{itemize}
\item $T_0 = Id$ and $||T_t|| \leq 1$ for every $t \in T$, \emph{i.e.} $\{T_t\}_{t \in T}$ is a family of contracting maps; 
\item $T_{t+s} = T_t \circ T_s$ (the semigroup property); 
\item $\lim_{t \downarrow 0} ||T_tf - f|| = 0$ for every $f \in C_0(E)$.
\end{itemize}

A probability transition function associated with a Feller semigroup is called a \emph{Feller transition function}.  A Markov process having a Feller transition function is called a \emph{Feller process}.

\begin{thebibliography}{1}
\bibitem{RY} D. Revuz \& M. Yor, \emph{Continuous Martingales and Brownian Motion}, Third Edition Corrected.  Volume 293, Grundlehren der mathematischen Wissenschaften.  Springer, Berlin, 2005.
\end{thebibliography}

%%%%%
%%%%%
\end{document}
