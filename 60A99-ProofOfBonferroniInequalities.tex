\documentclass[12pt]{article}
\usepackage{pmmeta}
\pmcanonicalname{ProofOfBonferroniInequalities}
\pmcreated{2013-03-22 19:12:44}
\pmmodified{2013-03-22 19:12:44}
\pmowner{csguy}{26054}
\pmmodifier{csguy}{26054}
\pmtitle{Proof of Bonferroni Inequalities}
\pmrecord{6}{42131}
\pmprivacy{1}
\pmauthor{csguy}{26054}
\pmtype{Proof}
\pmcomment{trigger rebuild}
\pmclassification{msc}{60A99}

% this is the default PlanetMath preamble.  as your knowledge
% of TeX increases, you will probably want to edit this, but
% it should be fine as is for beginners.

% almost certainly you want these
\usepackage{amssymb}
\usepackage{amsmath}
\usepackage{amsfonts}
\usepackage{amsthm}

% used for TeXing text within eps files
%\usepackage{psfrag}
% need this for including graphics (\includegraphics)
%\usepackage{graphicx}
% for neatly defining theorems and propositions
%\usepackage{amsthm}
% making logically defined graphics
%%%\usepackage{xypic}

% there are many more packages, add them here as you need them

% define commands here

\begin{document}
\newtheorem*{defs}{Definitions and Notation}
\newtheorem{lem}{Lemma}
\newtheorem{cor}{Corollary}
\newtheorem{thm}{Theorem}

\begin{defs}
A {\em measure space} is a triple $(X,\Sigma,\mu)$, where $X$ is a set, $\Sigma$ is a $\sigma$-algebra over $X$, and $\mu \colon \Sigma \rightarrow [0,\infty]$ is a measure, that is, a non-negative function that is countably additive.  If $A \in \Sigma$, the {\em characteristic function} of $A$ is the function $\chi_A \colon X \rightarrow \mathbb{R}$ defined by $\chi_A(x) = 1$ if $x \in A$, $\chi_A(x) = 0$ if $x \notin A$.  A {\em unimodal sequence} is a sequence of real numbers $a_0,a_1,\ldots,a_n$ for which there is an index $k$ such that $a_{i} \leq a_{i+1}$ for $i < k$ and $a_i \geq a_{i+1}$ for $i \geq k$.
\end{defs}

The proof of the following easy lemma is left to the reader:

\begin{lem}
\label{unimodal}
If $a_0 \leq a_1 \leq \ldots \leq a_k \geq a_{k+1} \geq a_{k+2} \geq \ldots \geq a_n$ is a unimodal sequence of non-negative real numbers with $\sum_{i=0}^{n} (-1)^i a_i = 0$, then $\sum_{i=0}^j (-1)^i a_i \geq 0$ for even $j$ and $\leq 0$ for odd $j$.
\end{lem}

Since the binomial sequence $(\binom{a}{i})_{0 \leq i \leq n}$ with integer $a > 0$ and integer $n \geq a$ satisfies the hypothesis of Lemma~\ref{unimodal}, we have:

\begin{cor}
\label{bincor}
If $a$ is a positive integer, $\sum_{i=0}^j (-1)^i \binom{a}{i} \geq 0$ for even $j$ and $\leq 0$ for odd $j$.
\end{cor}

\begin{lem}
\label{combinat}
Let $(A_i)_{1 \leq i \leq n}$ be a sequence of sets and let $X = \bigcup_{1 \leq i \leq n} A_i$. For $x \in X$, let $I(x)$ be the set of indices $j$ such that $x \in A_j$. If $1 \leq k \leq n$, 
\begin{equation*}
\sum_{1 \leq i_1 < i_2 < \ldots < i_k \leq n} \chi_{A_{i_1} \cap A_{i_2} \cap \ldots \cap A_{i_k}} (x) = \binom{|I(x)|}{k}
\end{equation*}
for all $x \in X$.
\end{lem}
\begin{proof}
$\chi_{A_{i_1} \cap A_{i_2} \cap \ldots \cap A_{i_k}} (x) = 1$ if $\{i_1, i_2, \ldots, i_k\} \subseteq I(x)$, and $= 0$ otherwise. Therefore the sum equals the number of $k$-subsets of $I(x)$, which is $\binom{|I(x)|}{k}$.
\end{proof}

\begin{thm}
Let $(X, \Sigma, \mu)$ be a measure space.  If $(A_i)_{1 \leq i \leq n}$ is a finite sequence of measurable sets all having finite measure, and 
\begin{equation*}
S_j = \mu(A_1 \cup A_2 \cup \ldots \cup A_n) + \sum_{k=1}^j (-1)^k \sum_{1 \leq i_1<i_2<\ldots<i_k \leq n} \mu(A_{i_1}\cap A_{i_2} \cap \ldots \cap A_{i_k})
\end{equation*}
 then $S_j \geq 0$ for even $j$, and $\leq 0$ for odd $j$.  Moreover, $S_n = 0$ (Principle of Inclusion-Exclusion).
\end{thm}

\begin{proof}
Let $Y = \bigcup_{1 \leq i \leq n} A_i$.
\begin{eqnarray*}
S_j & = & \int_Y d\mu + \sum_{k=1}^j (-1)^k \sum_{1 \leq i_1<i_2<\ldots <i_k\leq n} \int_Y \chi_{A_{i_1} \cap A_{i_2} \cap \ldots \cap A_{i_k}} d\mu \\
    & = & \int_Y d\mu +  \sum_{k=1}^j (-1)^k \int_Y (\sum_{1 \leq i_1<i_2<\ldots <i_k\leq n} \chi_{A_{i_1} \cap A_{i_2} \cap \ldots \cap A_{i_k}}) d\mu
\end{eqnarray*}
By Lemma~\ref{combinat},
\begin{eqnarray*}
S_j & = & \int_Y d\mu + \sum_{k=1}^j (-1)^k \int_Y \binom{|I(x)|}{k} d\mu \\
    & = &  \sum_{k=0}^j (-1)^k \int_Y \binom{|I(x)|}{k} d\mu \\
    & = & \int_Y \sum_{k=0}^j (-1)^k \binom{|I(x)|}{k} d\mu
\end{eqnarray*}
Since $|I(x)| >0$ for $x \in Y$, it follows from Corollary~\ref{bincor} that, in the last integral, the integrand is $\geq 0$ for even $j$ and $\leq 0$ for odd $j$. Therefore the same is true for the integral itself. In addition, the integrand is identically 0 for $j=n$, hence $S_n = 0$.
\end{proof}

This proof shows that at the heart of Bonferroni's inequalities lie similar inequalities governing the binomial coefficients.

%%%%%
%%%%%
\end{document}
