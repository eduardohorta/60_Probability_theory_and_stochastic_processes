\documentclass[12pt]{article}
\usepackage{pmmeta}
\pmcanonicalname{Martingale}
\pmcreated{2013-03-22 13:33:09}
\pmmodified{2013-03-22 13:33:09}
\pmowner{CWoo}{3771}
\pmmodifier{CWoo}{3771}
\pmtitle{martingale}
\pmrecord{25}{34157}
\pmprivacy{1}
\pmauthor{CWoo}{3771}
\pmtype{Definition}
\pmcomment{trigger rebuild}
\pmclassification{msc}{60G46}
\pmclassification{msc}{60G44}
\pmclassification{msc}{60G42}
%\pmkeywords{martingale}
%\pmkeywords{supermartingale}
%\pmkeywords{submartingale}
\pmrelated{LocalMartingale}
\pmrelated{DoobsOptionalSamplingTheorem}
\pmrelated{ConditionalExpectationUnderChangeOfMeasure}
\pmrelated{MartingaleConvergenceTheorem}
\pmdefines{martingale}
\pmdefines{supermartingale}
\pmdefines{submartingale}
\pmdefines{reverse submartingale}
\pmdefines{reverse supermartingale}

\endmetadata

% this is the default PlanetMath preamble.  as your knowledge
% of TeX increases, you will probably want to edit this, but
% it should be fine as is for beginners.

% almost certainly you want these
\usepackage{amssymb}
\usepackage{amsmath}
\usepackage{amsfonts}
\usepackage{verbatim}

% define commands here
%\newtheorem{defn}{Definition}
\newtheorem{rem}{Remark}
\numberwithin{equation}{section}
\newcommand{\Real}{\mathbb R}
\newcommand{\Prob}{\mathbb P}
\newcommand{\F}{\mathcal{F}}
\begin{document}
\title{Martingales definition}%

\textbf{Definition}.  Let $(\Omega, \F,(\F_t)_{t\in\mathbb{T}},\Prob)$ be a filtered probability space and $(X_t)$ be a stochastic process such that $X_t$ is \PMlinkname{integrable}{Integral2} for all $t\in\mathbb{T}$. Then, $X=(X_t, \F_t)$ is called a \emph{submartingale} if
$$\mathbb{E}^{\Prob}[X_t|\F_s] \geq X_s,\, \mbox{for every $s < t$, a.e.[$\Prob$],}$$
and a \emph{supermartigale} if 
$$\mathbb{E}^{\Prob}[X_t|\F_s] \leq X_s,\, \mbox{for every $s < t$, a.e.[$\Prob$].}$$

\noindent A submartingale that is also a supermartingale is called a
\emph{martingale}, i.e., a martingale satisfies
$$\mathbb{E}^{\Prob}[X_t|\F_s] = X_s,\, \mbox{for
every $s < t$, a.e.[$\Prob$].}$$

\noindent Similarly, if the $\{\F_t\}$ form a decreasing collection of $\sigma$-subalgebras of $\F$, then $X$ is called a \emph{reverse submartingale} if 
$$\mathbb{E}^{\Prob}[X_s|\F_t] \geq X_t,\, \mbox{for every $s < t$, a.e.[$\Prob$],}$$
and a \emph{reverse supermartingale} if 
$$\mathbb{E}^{\Prob}[X_s|\F_t] \leq X_t,\, \mbox{for every $s < t$, a.e.[$\Prob$].}$$

\medskip

\textbf{Remarks}
\begin{itemize}
\item
The martingale property captures the idea of a fair bet, where the
expected future value is equal to the current value.
\item
The submartingale property is equivalent to $$\int_A X_t \, d\Prob \geq
\int_A X_s \, d\Prob \,\,\, \mbox{for every $A \in \F_s$ and $s <
t$}$$ and similarly for the other definitions. This is immediate
from the definition of conditional expectation.
\end{itemize}


%%%%%
%%%%%
\end{document}
