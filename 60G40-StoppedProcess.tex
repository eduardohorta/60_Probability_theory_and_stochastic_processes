\documentclass[12pt]{article}
\usepackage{pmmeta}
\pmcanonicalname{StoppedProcess}
\pmcreated{2013-03-22 18:37:38}
\pmmodified{2013-03-22 18:37:38}
\pmowner{gel}{22282}
\pmmodifier{gel}{22282}
\pmtitle{stopped process}
\pmrecord{5}{41364}
\pmprivacy{1}
\pmauthor{gel}{22282}
\pmtype{Definition}
\pmcomment{trigger rebuild}
\pmclassification{msc}{60G40}
\pmclassification{msc}{60G05}
\pmsynonym{optional stopping}{StoppedProcess}
%\pmkeywords{stochastic process}
%\pmkeywords{stopping time}
\pmdefines{pre-stopped process}
\pmdefines{prestopped process}

\endmetadata

% almost certainly you want these
\usepackage{amssymb}
\usepackage{amsmath}
\usepackage{amsfonts}

% used for TeXing text within eps files
%\usepackage{psfrag}
% need this for including graphics (\includegraphics)
%\usepackage{graphicx}
% for neatly defining theorems and propositions
\usepackage{amsthm}
% making logically defined graphics
%%%\usepackage{xypic}

% there are many more packages, add them here as you need them

% define commands here
\newtheorem*{theorem*}{Theorem}
\newtheorem*{lemma*}{Lemma}
\newtheorem*{corollary*}{Corollary}
\newtheorem*{definition*}{Definition}
\newtheorem{theorem}{Theorem}
\newtheorem{lemma}{Lemma}
\newtheorem{corollary}{Corollary}
\newtheorem{definition}{Definition}

\begin{document}
\PMlinkescapeword{stable}
\PMlinkescapeword{equivalent}
A stochastic process $(X_t)_{t\in\mathbb{T}}$ defined on a measurable space $(\Omega,\mathcal{F})$ can be stopped at a random time $\tau\colon\Omega\rightarrow\mathbb{T}\cup\{\infty\}$. The resulting stopped process is denoted by $X^{\tau}$,
\begin{equation*}
X^\tau_t\equiv X_{\min(t,\tau)}.
\end{equation*}
The random time $\tau$ used is typically a stopping time.

If the process $X_t$ has \PMlinkname{left limits}{CadlagProcess} for every $t\in\mathbb{T}$, then it can alternatively be stopped just before the time $\tau$, resulting in the pre-stopped process
\begin{equation*}
X^{\tau-}\equiv\left\{
\begin{array}{ll}
X_t,&\textrm{if }t<\tau,\\
X_{\tau-},&\textrm{if }t\ge\tau.
\end{array}
\right.
\end{equation*}

Stopping is often used to enforce boundedness or integrability constraints on a process.
For example, if $B$ is a Brownian motion and $\tau$ is the first time at which $|B_{\tau}|$ hits some given positive value, then the stopped process $B^{\tau}$ will be a continuous and bounded martingale.
It can be shown that many properties of stochastic processes, such as the martingale property, are stable under stopping at any stopping time $\tau$. On the other hand, a pre-stopped martingale need not be a martingale.

For continuous processes, stopping and pre-stopping are equivalent procedures.
If $\tau$ is the first time at which $|X_\tau|\ge K$, for any given real number $K$, then the pre-stopped process $X^{\tau-}$ will be uniformly bounded.
However, for some noncontinuous processes it is not possible to find a stopping time $\tau>0$ making $X^\tau$ into a uniformly bounded process. For example, this is the case for any \PMlinkname{Levy process}{LevyProcess} with unbounded jump distribution.

Stopping is used to generalize properties of stochastic processes to obtain the related localized property. See, for example, local martingales.

%%%%%
%%%%%
\end{document}
