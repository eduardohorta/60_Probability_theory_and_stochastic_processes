\documentclass[12pt]{article}
\usepackage{pmmeta}
\pmcanonicalname{2StochasticMaps}
\pmcreated{2014-04-23 0:51:47}
\pmmodified{2014-04-23 0:51:47}
\pmowner{rspuzio}{6075}
\pmmodifier{rspuzio}{6075}
\pmtitle{2. Stochastic maps}
\pmrecord{11}{88086}
\pmprivacy{1}
\pmauthor{rspuzio}{6075}
\pmtype{Feature}
\pmclassification{msc}{60J20}

\endmetadata

% this is the default PlanetMath preamble.  as your knowledge
% of TeX increases, you will probably want to edit this, but
% it should be fine as is for beginners.

% almost certainly you want these
\usepackage{amssymb}
\usepackage{amsmath}
\usepackage{amsfonts}

% need this for including graphics (\includegraphics)
\usepackage{graphicx}
% for neatly defining theorems and propositions
\usepackage{amsthm}

% making logically defined graphics
%\usepackage{xypic}
% used for TeXing text within eps files
%\usepackage{psfrag}

% there are many more packages, add them here as you need them

% define commands here

\DeclareMathOperator{\Hom}{Hom}

\newcommand{\vecify}{{\mathcal V}}
\newcommand{\Act}{{A}}
\newcommand{\act}{{a}}
\newcommand{\Sit}{{S}}
\newcommand{\occ}{{v}}
\newcommand{\univ}{{\mathbf D}}
\newcommand{\uout}{{d_{out}}}
\newcommand{\uin}{{d_{in}}}
\newcommand{\mangle}{{\mathbf C}}

\newcommand{\psheaf}{{\mathcal F}}
\newcommand{\scat}{{\mathtt{Stoch}}}
\newcommand{\subs}{{\mathtt{Sys}}}
\newcommand{\mcat}{{\mathtt{Meas}}}
\newcommand{\eop}{{$\blacksquare$}}
\newcommand{\eod}{{${}$\\}}
\newcommand{\bra}{{\langle}}
\newcommand{\ket}{{\rangle}}

\newcommand{\cN}{{\mathcal N}}
\newcommand{\bR}{{\mathbb R}}
\newcommand{\fm}{{\mathfrak m}}
\newcommand{\cP}{{\mathcal P}}

\newtheorem{thm}{Theorem}
\newtheorem{prop}[thm]{Proposition}
\newtheorem{cor}[thm]{Corollary}

\theoremstyle{remark}
\newtheorem{eg}{Example}
\newtheorem{rem}{Remark}
\newtheorem{defn}{Definition}

\setcounter{eg}{0}
\setcounter{rem}{0}
\setcounter{defn}{1}
\begin{document}
Any conditional distribution $p(y|x)$ on finite sets $X$ and $Y$ can be 
represented as a matrix as follows. Let $\vecify X=\{\varphi:X\rightarrow 
\bR\}$ denote the vector space of real valued functions on $X$ and similarly
for $Y$. $\vecify X$ is equipped with Dirac basis 
$\{\delta_x:X\rightarrow\bR|x\in X\}$, where
\begin{equation*}
    \delta_x(x') = \left\{\begin{matrix}
		1 & \mbox{if }x=x'\\
		0 & \mbox{else}.
	\end{matrix}\right.
\end{equation*}
Given a conditional distribution $p(y|x)$ construct matrix $\fm_p$ with 
entry $p(y|x)$ in column $\delta_x$ and row $\delta_y$. Matrix $\fm_p$ is 
\emph{stochastic}: it has nonnegative entries and its columns sum to 1. 
Alternatively, given a stochastic matrix $\fm:\vecify X\rightarrow \vecify 
Y$, we can recover the conditional distribution. The Dirac basis induces 
Euclidean metric
\begin{equation}
	\bra\bullet|\bullet\ket:\vecify X\otimes\vecify X\rightarrow\bR:
	\left\langle\sum\alpha_x\delta_x\left|\sum\beta_x\delta_x\right\rangle\right.=\sum\alpha_x\beta_x
	\label{e:metric}
\end{equation}
which identifies vector spaces with their duals $\vecify X\approx 
(\vecify X)^*$. Let $p_\fm(y|x) := \bra\delta_y|\fm(\delta_x)\ket$.

\begin{defn}
	\label{d:catstoch}
	The \emph{category of stochastic maps} $\scat$ has function spaces 
    $\vecify X$ for objects and stochastic matrices $\fm:\vecify 
    X\rightarrow \vecify Y$ with respect to Dirac bases for arrows. We 
    identify of $(\vecify X)^*$ with $\vecify X$ using the Dirac basis 
    without further comment below.
\end{defn}
	
\begin{defn}
	\label{d:dual}
	The \emph{dual} of surjective stochastic map $\fm:\vecify X\rightarrow 
    \vecify Y$ is the composition $\fm^\natural:= \vecify 
    Y\xrightarrow{\fm^*\circ ren} \vecify X$, where $ren$ is the unique map 
    making diagram
	% $\fm^\natural:\vecify Y\rightarrow \vecify X$, the transpose 
    $\fm^\perp$ of $\fm$ with columns renormalized to sum to 1. The 
    stochastic dual $\fm^\natural$ is
\begin{figure}
 \centering}
 \includegraphics{information-theoretic-distributed-measurement-2.1.png}
\end{figure}
%\begin{equation*}
%	\xymatrix{
%	& (\vecify Y)^*\ar[rr]^{\fm^*} & & 
%	(\vecify X)^*\ar[d]_{\omega_X}\\ 
%	& (\vecify Y)^*\ar[u]^{ren}\ar[rr]_{\omega_Y}  & & \bR 
%	}
%	\end{equation*}
	commute. Precomposing $\fm^*$ with $ren$ renormalizes
    \footnote{If $\fm$ is not surjective, i.e. if one of the rows has all 
    zero entries, then the renormalization is not well-defined.} its 
    columns to sum to 1. The stochastic dual of a stochastic transform is 
    stochastic; further, if $\fm$ is stochastic then 
    $(\fm^\natural)^\natural=\fm$.
\end{defn}

Category $\scat$ is described in terms of braid-like generators and 
relations in \cite{fritz:09}. A more general, but also more complicated, 
category of conditional distributions was introduced by Giry \cite{giry:81},
see \cite{panangaden:98}. 

\begin{eg}[deterministic functions]
    \label{eg:det}
	Let $\mathtt{FSet}$ be the category of finite sets. Define faithful 
    functor $\vecify:\mathtt{FSet}\rightarrow \scat$ taking set $X$ to 
    $\vecify X$ and function $f:X\rightarrow Y$ to stochastic map 
    $\vecify f:\vecify X\rightarrow \vecify Y:\delta_x\mapsto 
    \delta_{f(x)}$. It is easy to see that $\vecify(X\times Y)=\vecify 
    X\otimes \vecify Y$ and $\vecify (X\cup Y)=\vecify X\times \vecify Y$.

We introduce special notation for commonly used functions:
\begin{itemize}
	\item \emph{Set inclusion.} 
	For any inclusion $i:X\hookrightarrow Y$ of sets, let $\iota:=\vecify 
    i:\vecify X\rightarrow \vecify Y$ denote the corresponding stochastic 
    map. Two important examples are
	\begin{itemize}
		\item \emph{Point inclusion.} 
		Given $x\in X$ define $\iota_x:\bR\rightarrow \vecify X:1\mapsto 
        \delta_x$.
		\item \emph{Diagonal map.} 
		Inclusion $\Delta:X\hookrightarrow X\times X:x\mapsto(x,x)$ 
        induces $\iota_\Delta:\vecify X\rightarrow \vecify X\otimes 
        \vecify X:\delta_x\mapsto \delta_x\otimes \delta_x$.		
	\end{itemize}
	\item \emph{Terminal map.}
	Let $\omega_X:\vecify X\rightarrow \bR:\delta_x\mapsto 1$ denote the 
    terminal map induced by $X\rightarrow\{\bullet\}$.
	\item \emph{Projection.}
	Let $\pi_{XY,X}:\vecify X\otimes \vecify Y\rightarrow \vecify 
    X:\delta_x\otimes \delta_y\mapsto \delta_x$ denote the projection 
    induced by $pr_{X\times Y,X}:X\times Y\rightarrow X:(x,y)\mapsto x$.
\end{itemize}
\end{eg}

\begin{prop}
	[dual is Bayes over uniform distribution]
	\label{t:dual}
	The dual of a stochastic map applies Bayes rule to compute the 
    posterior distribution $\bra \fm^\natural(\delta_y)|\delta_x\ket=
    p_\fm(x|y)$ using the uniform probability distribution.
\end{prop}

\noindent
Proof: 
The uniform distribution is the dual $\omega_X^\natural:
\bR\rightarrow\vecify X:1\mapsto \frac{1}{|X|}\sum_x \delta_x$ 
of the terminal map $\omega_X:\vecify X\rightarrow \bR$. It assigns 
equal probability $p_{\omega^\natural}(x)=\frac{1}{|X|}$ to all of $X$'s 
elements, and can be characterized as the maximally uninformative 
distribution \cite{jaynes:57}. Let $\fm:\vecify X\rightarrow \vecify Y$. 
The normalized transpose is 
\begin{equation*}
	\fm^\natural(\delta_y)= \sum_x \frac{p_\fm(y|x)}{\sum_{x'}p_\fm(y|x')}
    \delta_x
	= \sum_x \frac{p_\fm(y|x)\cdot p_{\omega^\natural}(x)}
    {\sum_{x'}p_\fm(y|x')p_{\omega^\natural}(x')}\delta_x
	= \sum_x p_\fm(x|y) \cdot \delta_x.
	\,\,\blacksquare
\end{equation*}

\begin{rem}
    \label{r:notdirac}
	Note that $p_\fm(x|y):=\bra \fm^\natural(\delta_y)|
    \delta_x\ket\neq\bra \delta_y|\fm(\delta_x)\ket=:p_\fm(y|x)$. 
    Dirac's bra-ket notation must be used with care since stochastic 
    matrices are not necessarily symmetric \cite{dirac:58}.
\end{rem}

\begin{cor}
	[preimages]
	\label{t:preimage}
	The dual $(\vecify f)^\natural:\vecify Y\rightarrow \vecify X$ of 
    stochastic map $\vecify f:\vecify X\rightarrow \vecify Y$ is 
    conditional distribution
	\begin{equation}
		\label{e:preimage}
		p_{\vecify f}(x|y) = \left\{\begin{matrix}
			\frac{1}{|f^{-1}(y)|} & \mbox{if } f(x)=y\\
			0 &\mbox{else}.
		\end{matrix}\right.
	\end{equation}
\end{cor}

\noindent
Proof:
By the proof of Proposition~\ref{t:dual}
\begin{equation*}
		(\vecify f)^\natural(\delta_y)=
		\frac{1}{|f^{-1}(y)|}\sum_{\{x|f(x)=y\}}\delta_x.
		\,\,\blacksquare
\end{equation*}

The support of $p_{\vecify f}(X|y)$ is $f^{-1}(y)$. Elements in the 
support are assigned equal probability, thereby treating them as an 
undifferentiated list. Dual $(\vecify f)^\natural$ thus generalizes the 
inverse image $f^{-1}:Y\rightarrow \underline{2}^X$. Conveniently however, 
the dual $(\vecify X)^\natural$ simply flips the domain and range of 
$\vecify f$, whereas the inverse image maps to powerset $\underline{2}^X$, 
an entirely new object.

\begin{cor}
	[marginalization with respect to uniform distribution]
	\label{t:marginalize}
	Precomposing $\vecify X\otimes \vecify Y\xrightarrow{\fm}\vecify Z$ 
    with the dual $\pi_X^\natural$ to $\vecify X\otimes\vecify 
    Y\xrightarrow{\pi_X}\vecify X$ marginalizes $p_\fm(z|x,y)$ 
    over the uniform distribution on $Y$.
\end{cor}

\noindent
Proof:
By Corollary~\ref{t:preimage} we have $\pi^\natural_X:\vecify X
\rightarrow \vecify X\otimes \vecify Y:\delta_y\mapsto 
\frac{1}{|Y|}\sum_{y\in Y}\delta_x\otimes \delta_y$. It follows 
immediately that
\begin{equation*}
	p_{\fm\circ\pi^\natural_X}(z|x)=\frac{1}{|Y|}\sum_{y\in Y} 
    p_\fm(z|x,y).
	\,\,\blacksquare
\end{equation*}

Precomposing with $\pi_X^\natural$ treats inputs from $Y$ as 
extrinsic noise. Although duals can be defined so that they 
implement Bayes' rule with respect to other probability distributions, 
this paper restricts attention to the simplest possible renormalization 
of columns, Definition \ref{d:catstoch}. The uniform distribution is 
convenient since it uses minimal prior knowledge (it depends only on the 
number of elements in the set) to generalize pre-images to the stochastic 
case, Proposition~\ref{t:preimage}.

\begin{thebibliography}{99}
%\providecommand{\bibitemdeclare}[2]{}
\providecommand{\urlprefix}{Available at }
\providecommand{\url}[1]{\texttt{#1}}
\providecommand{\href}[2]{\texttt{#2}}
\providecommand{\urlalt}[2]{\href{#1}{#2}}
\providecommand{\doi}[1]{doi:\urlalt{http://dx.doi.org/#1}{#1}}
\providecommand{\bibinfo}[2]{#2}

%\bibitemdeclare{book}{dirac:58}
\bibitem{dirac:58}
\bibinfo{author}{P~A~M Dirac} (\bibinfo{year}{1958}): \emph{\bibinfo{title}{The
  {P}rinciples of {Q}uantum {M}echanics}}.
\newblock \bibinfo{publisher}{Oxford University Press}.

%\bibitemdeclare{article}{fritz:09}
\bibitem{fritz:09}
\bibinfo{author}{Tobias Fritz} (\bibinfo{year}{2009}): \emph{\bibinfo{title}{A
  presentation of the category of stochastic matrices}}.
\newblock {\sl \bibinfo{journal}{arXiv:0902.2554v1}} .

%\bibitemdeclare{incollection}{giry:81}
\bibitem{giry:81}
\bibinfo{author}{M~Giry} (\bibinfo{year}{1981}): \emph{\bibinfo{title}{A
  categorical approach to probability theory}}.
\newblock In \bibinfo{editor}{B~Banaschewski}, editor: {\sl
  \bibinfo{booktitle}{Categorical {A}spects of {T}opology and {A}nalysis}},
  \bibinfo{publisher}{Springer}.

%\bibitemdeclare{article}{jaynes:57}
\bibitem{jaynes:57}
\bibinfo{author}{E~T Jaynes} (\bibinfo{year}{1957}):
  \emph{\bibinfo{title}{Information theory and statistical mechanics}}.
\newblock {\sl \bibinfo{journal}{Phys. Rev.}}
  \bibinfo{volume}{106}(\bibinfo{number}{4}), pp. \bibinfo{pages}{620--630}.
  
%\bibitemdeclare{inproceedings}{panangaden:98}
\bibitem{panangaden:98}
\bibinfo{author}{P~Panangaden} (\bibinfo{year}{1998}):
  \emph{\bibinfo{title}{Probabilistic relations}}.
\newblock In \bibinfo{editor}{C~Baier}, \bibinfo{editor}{M~Huth},
  \bibinfo{editor}{M~Kwiatkowska} \& \bibinfo{editor}{M~Ryan}, editors: {\sl
  \bibinfo{booktitle}{PROBMIV'98}}, pp. \bibinfo{pages}{59--74}.

\end{thebibliography}

\end{document}
