\documentclass[12pt]{article}
\usepackage{pmmeta}
\pmcanonicalname{CriterionForAlmostsureConvergence}
\pmcreated{2013-03-22 15:54:45}
\pmmodified{2013-03-22 15:54:45}
\pmowner{stevecheng}{10074}
\pmmodifier{stevecheng}{10074}
\pmtitle{criterion for almost-sure convergence}
\pmrecord{15}{37916}
\pmprivacy{1}
\pmauthor{stevecheng}{10074}
\pmtype{Corollary}
\pmcomment{trigger rebuild}
\pmclassification{msc}{60A99}
\pmsynonym{corollary of Borel-Cantelli lemma}{CriterionForAlmostsureConvergence}

\endmetadata

\usepackage{amssymb}
\usepackage{amsmath}
\usepackage{amsfonts}
\usepackage{amsthm}
\usepackage{enumerate}
%\usepackage{graphicx}
%\usepackage{psfrag}
%%%\usepackage{xypic}

% define commands here
\newcommand{\real}{\mathbb{R}}
\newcommand{\PP}{\mathbb{P}}

\providecommand{\abs}[1]{\lvert#1\rvert}
\providecommand{\absW}[1]{\left\lvert#1\right\rvert}
\providecommand{\absB}[1]{\Bigl\lvert#1\Bigr\rvert}


\begin{document}
Let $X_1, X_2, \dotsc$ and $X$ be random variables.
If, for every $\epsilon > 0$, the sum $\sum_{n=1}^\infty \PP( \abs{X_n - X} > \epsilon )$  is finite,
then $X_n$ converge to $X$ almost surely.

\begin{proof}
By the Borel-Cantelli lemma, we have $\PP(\limsup_n \{ \abs{X_n - X} > \epsilon \})=0 $.
But $\limsup_n \{ \abs{X_n - X} > \epsilon \}$ is the same as the event $\{ \limsup_n \abs{X_n - X} > \epsilon \}$.
(The latter event involves the \PMlinkname{limit superior of \emph{numbers}}{LimitSuperior}; the former involves the 
\PMlinkname{limit superior of \emph{sets}}{InfinitelyOften}.)
So taking the limit $\epsilon \searrow 0$,
we have $\PP( \limsup_n \abs{X_n - X} > 0) = 0$, 
or equivalently
$\PP( \limsup_n \abs{X_n - X} = 0) = 1$.
\end{proof}

%%%%%
%%%%%
\end{document}
