\documentclass[12pt]{article}
\usepackage{pmmeta}
\pmcanonicalname{EigenvaluesOfStochasticMatrix}
\pmcreated{2013-03-22 16:18:02}
\pmmodified{2013-03-22 16:18:02}
\pmowner{Andrea Ambrosio}{7332}
\pmmodifier{Andrea Ambrosio}{7332}
\pmtitle{eigenvalues of stochastic matrix}
\pmrecord{7}{38421}
\pmprivacy{1}
\pmauthor{Andrea Ambrosio}{7332}
\pmtype{Theorem}
\pmcomment{trigger rebuild}
\pmclassification{msc}{60G99}
\pmclassification{msc}{15A51}

% this is the default PlanetMath preamble.  as your knowledge
% of TeX increases, you will probably want to edit this, but
% it should be fine as is for beginners.

% almost certainly you want these
\usepackage{amssymb}
\usepackage{amsmath}
\usepackage{amsfonts}

% used for TeXing text within eps files
%\usepackage{psfrag}
% need this for including graphics (\includegraphics)
%\usepackage{graphicx}
% for neatly defining theorems and propositions
\usepackage{amsthm}
% making logically defined graphics
%%%\usepackage{xypic}

% there are many more packages, add them here as you need them

% define commands here

\begin{document}
Theorem:
The spectrum of a stochastic matrix is contained in the unit disc in the complex plane.

\begin{proof}
Let $A$ be a stochastic matrix and let $m$ be an eigenvalue of $A$, with $v$ eigenvector; then, for any self-consistent matrix norm $\left\Vert .\right\Vert $, we have:
\[
\left|m\right|\left\Vert v\right\Vert =\left\Vert mv\right\Vert =\left\Vert Av\right\Vert \leq\left\Vert A\right\Vert \left\Vert v\right\Vert ,
\]
that is, since $v$ is nonzero,
\[
\left|m\right|\leq\left\Vert A\right\Vert .
\]
Now, for a (doubly) stochastic matrix,
\[
\left\Vert A\right\Vert _1 = \max_j \left(\sum_i \left|a_{ij}\right|\right)=1
\]
whence the conclusion.
\end{proof}
%%%%%
%%%%%
\end{document}
