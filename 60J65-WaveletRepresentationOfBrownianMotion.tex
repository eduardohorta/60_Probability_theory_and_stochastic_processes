\documentclass[12pt]{article}
\usepackage{pmmeta}
\pmcanonicalname{WaveletRepresentationOfBrownianMotion}
\pmcreated{2013-03-22 15:12:51}
\pmmodified{2013-03-22 15:12:51}
\pmowner{PrimeFan}{13766}
\pmmodifier{PrimeFan}{13766}
\pmtitle{wavelet representation of Brownian motion}
\pmrecord{7}{36976}
\pmprivacy{1}
\pmauthor{PrimeFan}{13766}
\pmtype{Theorem}
\pmcomment{trigger rebuild}
\pmclassification{msc}{60J65}
\pmsynonym{construction of Brownian motion}{WaveletRepresentationOfBrownianMotion}
%\pmkeywords{Brownian motion}
%\pmkeywords{Wiener process}
%\pmkeywords{wavelet}

\endmetadata

% this is the default PlanetMath preamble.  as your knowledge
% of TeX increases, you will probably want to edit this, but
% it should be fine as is for beginners.

% almost certainly you want these
\usepackage{amssymb}
\usepackage{amsmath}
\usepackage{amsfonts}

% used for TeXing text within eps files
%\usepackage{psfrag}
% need this for including graphics (\includegraphics)
%\usepackage{graphicx}
% for neatly defining theorems and propositions
\usepackage{amsthm}
% making logically defined graphics
%%%\usepackage{xypic}

% there are many more packages, add them here as you need them

% define commands here
\begin{document}
First we define the function
\begin{equation}
H(t) =
  \begin{cases}
    1 & \text{for } 0 \leq t < \tfrac{1}{2} \\
    -1 & \text{for } \tfrac{1}{2} \leq t \leq 1 \\
    0 & \text{otherwise.}
  \end{cases}
\end{equation}
and the sequence of functions
\begin{equation}
H_n(t) = 2^{j/2}H(2^j t - k) \end{equation} for $n=2^j+k$ where $j>0$ and $0\leq k \leq 2^j$.  We also set $H_0(t)=1$.
\newtheorem*{thm}{Wavelet Representation of Brownian Motion}
\begin{thm}
If $\{Z_n:0\leq n < \infty \}$ is a sequence of independent Gaussian random variables with mean
$0$ and variance $1$, then the series defined by
\begin{equation}
X_t = \sum^{\infty}_{n=0} \left(Z_n \int_0^t H_n(s)\;ds\right)
\end{equation}
converges uniformly on $[0,1]$ with probability one.  Moreover, the process $\{X_t\}$ defined by
the limit is a \emph{Brownian motion} for $0 \leq t \leq 1$.
\end{thm}
%%%%%
%%%%%
\end{document}
