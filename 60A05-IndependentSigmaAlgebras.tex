\documentclass[12pt]{article}
\usepackage{pmmeta}
\pmcanonicalname{IndependentSigmaAlgebras}
\pmcreated{2013-03-22 16:22:58}
\pmmodified{2013-03-22 16:22:58}
\pmowner{CWoo}{3771}
\pmmodifier{CWoo}{3771}
\pmtitle{independent sigma algebras}
\pmrecord{9}{38527}
\pmprivacy{1}
\pmauthor{CWoo}{3771}
\pmtype{Definition}
\pmcomment{trigger rebuild}
\pmclassification{msc}{60A05}
\pmsynonym{mutually independent $\sigma$-algebras}{IndependentSigmaAlgebras}
\pmdefines{mutually independent sigma algebras}

\usepackage{amssymb,amscd}
\usepackage{amsmath}
\usepackage{amsfonts}

% used for TeXing text within eps files
%\usepackage{psfrag}
% need this for including graphics (\includegraphics)
%\usepackage{graphicx}
% for neatly defining theorems and propositions
%\usepackage{amsthm}
% making logically defined graphics
%%\usepackage{xypic}
\usepackage{pst-plot}
\usepackage{psfrag}

% define commands here

\begin{document}
\PMlinkescapeword{independent}

Let $(\Omega, \mathcal{B}, P)$ be a probability space.  Let $\mathcal{B}_1$ and $\mathcal{B}_2$ be two sub sigma algebras of $\mathcal{B}$.  Then $\mathcal{B}_1$ and $\mathcal{B}_2$ are said to be \emph{\PMlinkescapetext{independent}} if for any pair of events $B_1\in\mathcal{B}_1$ and $B_2\in\mathcal{B}_2$:
$$P(B_1\cap B_2)=P(B_1)P(B_2).$$

More generally, a finite set of sub-$\sigma$-algebras $\mathcal{B}_1,\ldots, \mathcal{B}_n$ is \emph{independent} if for any set of events $B_i\in \mathcal{B}_i$, $i=1,\ldots,n$:
$$P(B_1\cap\cdots\cap B_n)=P(B_1)\cdots P(B_n).$$

An arbitrary set $\mathcal{S}$ of sub-$\sigma$-algebras is \emph{mutually independent} if any finite subset of $\mathcal{S}$ is independent.

The above definitions are generalizations of the notions of \PMlinkname{independence}{Independent} for events and for random variables: 
\begin{quote}
\begin{enumerate}
\item 
Events $B_1,\ldots,B_n$ (in $\Omega$) are \emph{mutually independent} if the sigma algebras $\sigma(B_i):=\lbrace \varnothing, B_i, \Omega-B_i, \Omega\rbrace$ are mutually independent.  
\item 
Random variables $X_1,\ldots,X_n$ defined on $\Omega$ are \emph{mutually independent} if the \PMlinkname{sigma algebras $\mathcal{B}_{X_i}$ generated by}{MathcalFMeasurableFunction} the $X_i$'s are mutually independent.  
\end{enumerate}
\end{quote}
In general, mutual independence among events $B_i$, random variables $X_j$, and sigma algebras $\mathcal{B}_k$ means the mutual independence among $\sigma(B_i)$, $\mathcal{B}_{X_j}$, and $\mathcal{B}_k$.

\textbf{Remark}.  Even when random variables $X_1,\ldots, X_n$ are defined on different probability spaces $(\Omega_i,\mathcal{B}_i,P_i)$, we may form the \PMlinkname{product}{InfiniteProductMeasure} of these spaces $(\Omega,\mathcal{B},P)$ so that $X_i$ (by abuse of notation) are now defined on $\Omega$ and their independence can be discussed.
%%%%%
%%%%%
\end{document}
