\documentclass[12pt]{article}
\usepackage{pmmeta}
\pmcanonicalname{BrownianMotion}
\pmcreated{2013-03-22 15:12:46}
\pmmodified{2013-03-22 15:12:46}
\pmowner{skubeedooo}{5401}
\pmmodifier{skubeedooo}{5401}
\pmtitle{Brownian motion}
\pmrecord{16}{36974}
\pmprivacy{1}
\pmauthor{skubeedooo}{5401}
\pmtype{Definition}
\pmcomment{trigger rebuild}
\pmclassification{msc}{60J65}
\pmsynonym{Wiener process}{BrownianMotion}
%\pmkeywords{Levy characterization}
\pmrelated{WienerMeasure}
\pmrelated{StochasticCalculusAndSDE}

\endmetadata

% this is the default PlanetMath preamble.  as your 
% of TeX increases, you will probably want to edit this, but
% it should be fine as is for beginners.

% almost certainly you want these
\usepackage{amssymb}
\usepackage{amsmath}
\usepackage{amsfonts}

% used for TeXing text within eps files
%\usepackage{psfrag}
% need this for including graphics (\includegraphics)
\usepackage{graphicx}
% for neatly defining theorems and propositions
\usepackage{amsthm}
% making logically defined graphics
%%%\usepackage{xypic}

% there are many more packages, add them here as you need 

% define commands here

\begin{document}
\theoremstyle{definition}
\newtheorem*{mydefn}{Definition}
\begin{mydefn}
One-dimensional \emph{Brownian motion} is a stochastic process $W(t)$, defined for $t\in [0,\infty)$ such that
\begin{enumerate}
\item $W(0) = 0$ almost surely
\item The sample paths $t \mapsto W(t)$ are almost surely continuous.
\item For any finite sequence of times $t_0 < t_1 < \cdots <t_n$,
the increments 
\[
W(t_1)-W(t_0), W(t_2)-W(t_1), \dotsc, W(t_n)-W(t_{n-1})
\]
are independent.
\item
For any times $s < t$, $W(t)-W(s)$ is normally distributed with mean zero
and variance $t-s$.
\end{enumerate}
\end{mydefn}

\begin{mydefn}
A $d$-dimensional Brownian motion is a stochastic process 
$W(t) = (W_1(t), \dots, W_d(t))$ in 
$\mathbb{R}^d$ whose coordinate processes $W_i(t)$ are 
independent one-dimensional Brownian motions.
\end{mydefn}

\begin{figure}
\includegraphics{brownian.eps}
\caption{Sample paths of a standard Brownian motion}
\end{figure}


%%%%%
%%%%%
\end{document}
