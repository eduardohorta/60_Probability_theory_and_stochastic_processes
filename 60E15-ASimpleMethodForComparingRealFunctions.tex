\documentclass[12pt]{article}
\usepackage{pmmeta}
\pmcanonicalname{ASimpleMethodForComparingRealFunctions}
\pmcreated{2013-03-22 16:10:47}
\pmmodified{2013-03-22 16:10:47}
\pmowner{Andrea Ambrosio}{7332}
\pmmodifier{Andrea Ambrosio}{7332}
\pmtitle{a simple method for comparing real functions}
\pmrecord{10}{38267}
\pmprivacy{1}
\pmauthor{Andrea Ambrosio}{7332}
\pmtype{Result}
\pmcomment{trigger rebuild}
\pmclassification{msc}{60E15}

% this is the default PlanetMath preamble.  as your knowledge
% of TeX increases, you will probably want to edit this, but
% it should be fine as is for beginners.

% almost certainly you want these
\usepackage{amssymb}
\usepackage{amsmath}
\usepackage{amsfonts}

% used for TeXing text within eps files
%\usepackage{psfrag}
% need this for including graphics (\includegraphics)
%\usepackage{graphicx}
% for neatly defining theorems and propositions
\usepackage{amsthm}
% making logically defined graphics
%%%\usepackage{xypic}

% there are many more packages, add them here as you need them

% define commands here

\begin{document}
Theorem:

Let $f(x)$ and $g(x)$ be real-valued, twice differentiable
functions on $[a,b]$, and let $x_{0}$ $\in [a,b]$.

If $f(x_{0})=g(x_{0})$, $f^{\prime }(x_{0})=g^{\prime }(x_{0})$, $f^{\prime \prime }(x)\leq g^{\prime \prime }(x)$ for all $x$ in $[a,b]$, then $f(x)\leq g(x)$ for all $x$ in $[a,b]$.

\begin{proof}

Let $h(x)=g(x)-f(x)$; by our hypotheses, $h(x)$ is a twice differentiable
function on $[a,b]$, and by the Taylor formula with \PMlinkname{Lagrange form remainder}{RemainderVariousFormulas} one has for any $x\in [a,b]$:
\[
h(x)=h(x_{0})+h^{\prime }(x_{0})(x-x_{0})+\frac{1}{2}h^{\prime \prime }(\xi
)(x-x_{0})^{2}
\]

where $\xi =\xi (x)\in [x,x_{0}]$.

Then by hypotheses,
\begin{eqnarray*}
h(x_{0}) &=&g(x_{0})-f(x_{0})=0 \\
h^{\prime }(x_{0}) &=&g^{\prime }(x_{0})-f^{\prime }(x_{0})=0 \\
h^{\prime \prime }(\xi ) &=&g^{\prime \prime }(\xi )-f^{\prime \prime }(\xi
)\geq 0
\end{eqnarray*}
so that
\[
h(x)=\frac{1}{2}h^{\prime \prime }(\xi )(x-x_{0})^{2}\geq 0
\]
whence the thesis.
\end{proof}

%%%%%
%%%%%
\end{document}
