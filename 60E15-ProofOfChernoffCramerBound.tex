\documentclass[12pt]{article}
\usepackage{pmmeta}
\pmcanonicalname{ProofOfChernoffCramerBound}
\pmcreated{2013-03-22 16:09:05}
\pmmodified{2013-03-22 16:09:05}
\pmowner{Andrea Ambrosio}{7332}
\pmmodifier{Andrea Ambrosio}{7332}
\pmtitle{proof of Chernoff-Cramer bound}
\pmrecord{26}{38231}
\pmprivacy{1}
\pmauthor{Andrea Ambrosio}{7332}
\pmtype{Proof}
\pmcomment{trigger rebuild}
\pmclassification{msc}{60E15}

\endmetadata

% this is the default PlanetMath preamble.  as your knowledge
% of TeX increases, you will probably want to edit this, but
% it should be fine as is for beginners.

% almost certainly you want these
\usepackage{amssymb}
\usepackage{amsmath}
\usepackage{amsfonts}

% used for TeXing text within eps files
%\usepackage{psfrag}
% need this for including graphics (\includegraphics)
%\usepackage{graphicx}
% for neatly defining theorems and propositions
%\usepackage{amsthm}
% making logically defined graphics
%%%\usepackage{xypic}

% there are many more packages, add them here as you need them

% define commands here

\begin{document}
Let $h(x)$ be the step function ($h(x)=1$ for\ $x\geq 0
$, $h(x)=0$  for $x<0$); then, by generalized Markov inequality, for any $t > 0$ and
any $\varepsilon \geq 0$,
\begin{eqnarray*}
\Pr\left\{ \sum_{i=1}^{n}\left( X_{i}-E[X_{i}]\right) >\varepsilon \right\} 
&=&E\left[ h\left( \sum_{i=1}^{n}\left( X_{i}-E[X_{i}]\right) -\varepsilon
\right) \right] \leq  \\
&\leq &E\left[ e^{t\left( \sum_{i=1}^{n}\left( X_{i}-E[X_{i}]\right)
-\varepsilon \right) }\right] = \\
&=&\exp (-\varepsilon t)E\left[ e^{\sum_{i=1}^{n}t\left(
X_{i}-E[X_{i}]\right) }\right] = \\
&=&\exp (-\varepsilon t)E\left[ \prod_{i=1}^{n}e^{t\left(
X_{i}-E[X_{i}]\right) }\right] = \\
\text{(by independence)} &=&\exp (-\varepsilon t)\prod_{i=1}^{n}E\left[
e^{t\left( X_{i}-E[X_{i}]\right) }\right] = \\
&=&\exp \left( -\varepsilon t+\sum_{i=1}^{n}\ln E\left[ e^{t\left(
X_{i}-E[X_{i}]\right) }\right] \right) = \\
&=&\exp \left[ -\left( t\varepsilon -\psi (t)\right) \right]. 
\end{eqnarray*}


Since this expression is valid for any $t > 0$, the best bound is obtained taking the supremum:

\[
\Pr\left\{ \sum_{i=1}^{n}\left( X_{i}-E[X_{i}]\right) >\varepsilon \right\}
\leq e^{-\sup_{t > 0}\left( t\varepsilon -\psi (t)\right)}
\]

which proves part c).

To prove part a), let's observe that $\Psi (0)=\sup_{t > 0}(-\psi (t))=-\inf_{t > 0}(\psi (t))$
and that
\begin{eqnarray*}
E\left[ e^{t\left( X_{i}-E[X_{i}]\right) }\right]  &\geq &E[1+t\left(
X_{i}-E[X_{i}]\right) ]= \\
&=&E[1]+tE[X_{i}]-tE[E[X_{i}]]= \\
&=&1=E\left[ e^{t\left( X_{i}-E[X_{i}]\right) }\right] _{t=0}
\end{eqnarray*}
that is, $t=0$ is the infimum point for $E\left[ e^{t\left(
X_{i}-E[X_{i}]\right) }\right] $ $\forall i$ and consequently for $\psi
(t)=\sum_{i=1}^{n}\ln E\left[ e^{t\left( X_{i}-EX_{i}\right) }\right] $, so as a conclusion $\Psi (0)=-\psi(0)=0$

b) Let $x>0$ be fixed and let $t_{0}$ be the supremum point for $tx-\psi (t)
$; we have to show that $t_{0}x-\psi (t_{0})>0$.

By differentiation, $\psi ^{\prime }(t_{0})=x$.

Let's recall that the moment generating function is convex, so $\psi
^{\prime \prime }(t)>0$. Writing the Taylor expansion for $\psi (t)$
around  $t=t_{0}$, we have, with a suitable $t_1<t_{0}$,
\[
0=\psi (0)=\psi (t_{0})-\psi ^{\prime }(t_{0})t_{0}+\frac{1}{2}\psi ^{\prime
\prime }(t_{1})t_{0}^{2}
\]
that is
\[
\Psi (x)=t_{0}x-\psi (t_{0})=t_{0}\psi ^{\prime }(t_{0})-\psi (t_{0})=\frac{1%
}{2}\psi ^{\prime \prime }(t_{1})t_{0}^{2}>0
\]

The convexity of $\Psi (x)$ follows from the fact that $\Psi (x)$ is the supremum of the linear (and hence convex) functions ${tx-\psi (t)}$ and so must be convex itself.

Eventually, in \PMlinkescapetext{order} to prove that $ \Psi (x)$ is an increasing function, let's note that
\[
\Psi ^{\prime }(0)=\lim_{x\rightarrow 0}\frac{\Psi (x)-\Psi (0)}{x}%
=\lim_{x\rightarrow 0}\frac{\Psi (x)}{x}>0
\]
and that, by Taylor formula with Lagrange form remainder, for a $\xi=\xi(x)$
\[
\Psi ^{\prime }(x)=\Psi ^{\prime }(0)+\Psi ^{\prime \prime }(\xi )x\geq 0
\]
since $\Psi ^{\prime \prime }(\xi )\geq 0$ by convexity and $x\geq 0$ by
hypotheses.
%%%%%
%%%%%
\end{document}
