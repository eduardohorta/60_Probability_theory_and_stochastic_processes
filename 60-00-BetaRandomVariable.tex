\documentclass[12pt]{article}
\usepackage{pmmeta}
\pmcanonicalname{BetaRandomVariable}
\pmcreated{2013-03-22 11:54:30}
\pmmodified{2013-03-22 11:54:30}
\pmowner{mathcam}{2727}
\pmmodifier{mathcam}{2727}
\pmtitle{beta random variable}
\pmrecord{11}{30530}
\pmprivacy{1}
\pmauthor{mathcam}{2727}
\pmtype{Definition}
\pmcomment{trigger rebuild}
\pmclassification{msc}{60-00}
\pmsynonym{beta distribution}{BetaRandomVariable}

\usepackage{amssymb}
\usepackage{amsmath}
\usepackage{amsfonts}
\usepackage{graphicx}
%%%%\usepackage{xypic}
\begin{document}
$X$ is a \textbf{beta random variable} with parameters \textbf{a and b} if\\
\par
$f_X(x) = \frac{x^{a-1} (1-x)^{b-1} }{\beta(a,b)}$,     $x \in [0,1]$	\\
\par
Parameters:\\
\par
\begin{list}{$\star$ }{}
\item $a > 0$
\item $b > 0$
\end{list}
\par
Syntax:\\
\par
$X\sim Beta(a,b)$\\
\par
Notes:\\
\par
\begin{enumerate}

\item $X$ is used in many statistical models.
\item The function $\beta: R\times R \to R$ is defined as $\beta(a,b) = \int_{0}^{1}{x^{a-1} (1-x)^{b-1} dx}$. $\beta(a,b)$ can be calculated as $\beta(a,b) = \frac {\Gamma(a) \Gamma(b) } {\Gamma(a+b)}$ (For information on the $\Gamma$ function, see the gamma random variable) 
\item $E[X] = \frac{a}{a+b}$
\item $Var[X] = \frac{ab}{(a+b+1)(a+b)^2}$
\item $M_X(t)$ not useful

\end{enumerate}
%%%%%
%%%%%
%%%%%
%%%%%
\end{document}
