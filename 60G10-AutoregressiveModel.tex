\documentclass[12pt]{article}
\usepackage{pmmeta}
\pmcanonicalname{AutoregressiveModel}
\pmcreated{2013-03-22 18:33:37}
\pmmodified{2013-03-22 18:33:37}
\pmowner{camillio}{22337}
\pmmodifier{camillio}{22337}
\pmtitle{autoregressive model}
\pmrecord{5}{41283}
\pmprivacy{1}
\pmauthor{camillio}{22337}
\pmtype{Definition}
\pmcomment{trigger rebuild}
\pmclassification{msc}{60G10}
\pmclassification{msc}{62J05}
\pmsynonym{AR}{AutoregressiveModel}
%\pmkeywords{regression}
%\pmkeywords{autoregression}
%\pmkeywords{autoregression model}
%\pmkeywords{model}

\endmetadata

% this is the default PlanetMath preamble.  as your knowledge
% of TeX increases, you will probably want to edit this, but
% it should be fine as is for beginners.

% almost certainly you want these
\usepackage{amssymb}
\usepackage{amsmath}
\usepackage{amsfonts}

% used for TeXing text within eps files
%\usepackage{psfrag}
% need this for including graphics (\includegraphics)
%\usepackage{graphicx}
% for neatly defining theorems and propositions
%\usepackage{amsthm}
% making logically defined graphics
%%%\usepackage{xypic}

% there are many more packages, add them here as you need them

% define commands here

\begin{document}
The {\it autoregressive model} of order $p$, denoted {\it AR}($p$), is a random process model described by
\begin{equation}
  y_{t} = \sum_{i=1}^{p} a_{i} y_{t-i} + c + e_{t}, \qquad t = 1, 2,\ldots
\end{equation}
where $a_i$ are model parameters, $y_{t}$ is the model output in discrete time instant $t$. Term $c$ is an absolute term (constant) and $e_{t}$ denotes discrete white noise.

A first-order autoregression model {\it AR}(1) in the form $y_{t} = a y_{t-1} + c +e _{t}$ is one major example.
%%%%%
%%%%%
\end{document}
