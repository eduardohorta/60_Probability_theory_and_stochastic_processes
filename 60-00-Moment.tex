\documentclass[12pt]{article}
\usepackage{pmmeta}
\pmcanonicalname{Moment}
\pmcreated{2013-03-22 11:53:54}
\pmmodified{2013-03-22 11:53:54}
\pmowner{CWoo}{3771}
\pmmodifier{CWoo}{3771}
\pmtitle{moment}
\pmrecord{11}{30515}
\pmprivacy{1}
\pmauthor{CWoo}{3771}
\pmtype{Definition}
\pmcomment{trigger rebuild}
\pmclassification{msc}{60-00}
\pmclassification{msc}{62-00}
\pmclassification{msc}{81-00}
\pmdefines{central moment}
\pmdefines{skewness}
\pmdefines{kurtosis}
\pmdefines{platykurtic}
\pmdefines{leptokurtic}

\usepackage{amssymb}
\usepackage{amsmath}
\usepackage{amsfonts}
\usepackage{graphicx}
%%%%\usepackage{xypic}
\begin{document}
\textit{Moments}\\
\par
Given a random variable $X$, the \textbf{$k$th moment} of $X$ is the value $E[X^k]$, if the expectation exists.\\
\\
Note that the expected value is the first moment of a random variable, and the variance is the second moment minus the first moment squared.\\
\par
The $k$th moment of $X$ is usually obtained by using the moment generating function.
\par
\par
\textit{Central moments}\\
\par
Given a random variable $X$, the \textbf{$k$th central moment} of $X$ is the value $E\big[(X-E[X])^k\big]$, if the expectation exists.  It is denoted by $\mu_k$.\\
\\
Note that the $\mu_1=0$ and $\mu_2=Var[X]=\sigma^2$. The third central moment divided by the standard deviation cubed is called the \emph{skewness} $\tau$:  $$\tau=\frac{\mu_3}{\sigma^3}$$  The skewness measures how ``symmetrical'', or rather, how ``skewed'', a distribution is with respect to its mode.  A non-zero $\tau$ means there is some degree of skewness in the distribution.  For example, $\tau>0$ means that the distribution has a longer positive tail.  
\par
The fourth central moment divided by the fourth power of the standard deviation is called the \emph{kurtosis} $\kappa$:
$$\kappa=\frac{\mu_4}{\sigma^4}$$  The kurtosis measures how ``peaked'' a distribution is compared to the standard normal distribution.  The standard normal distribution has $\kappa=3$.  $\kappa<3$ means that the distribution is ``flatter'' than then standard normal distribution, or \emph{platykurtic}.  On the other hand, a distribution with $\kappa>3$ can be characterized as being more ``peaked'' than $N(0,1)$, or \emph{leptokurtic}.
%%%%%
%%%%%
%%%%%
%%%%%
\end{document}
