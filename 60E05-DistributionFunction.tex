\documentclass[12pt]{article}
\usepackage{pmmeta}
\pmcanonicalname{DistributionFunction}
\pmcreated{2013-03-22 13:02:51}
\pmmodified{2013-03-22 13:02:51}
\pmowner{Mathprof}{13753}
\pmmodifier{Mathprof}{13753}
\pmtitle{distribution function}
\pmrecord{16}{33451}
\pmprivacy{1}
\pmauthor{Mathprof}{13753}
\pmtype{Definition}
\pmcomment{trigger rebuild}
\pmclassification{msc}{60E05}
\pmsynonym{cumulative distribution function}{DistributionFunction}
\pmsynonym{distribution}{DistributionFunction}
\pmrelated{DensityFunction}
\pmrelated{CumulativeDistributionFunction}
\pmrelated{RandomVariable}
\pmrelated{Distribution}
\pmrelated{GeometricDistribution2}
\pmdefines{law of a random variable}

\endmetadata

\usepackage{graphicx}
%%%\usepackage{xypic} 
\usepackage{bbm}
\newcommand{\Z}{\mathbbmss{Z}}
\newcommand{\C}{\mathbbmss{C}}
\newcommand{\R}{\mathbbmss{R}}
\newcommand{\Q}{\mathbbmss{Q}}
\newcommand{\mathbb}[1]{\mathbbmss{#1}}
\newcommand{\figura}[1]{\begin{center}\includegraphics{#1}\end{center}}
\newcommand{\figuraex}[2]{\begin{center}\includegraphics[#2]{#1}\end{center}}
\begin{document}
[this entry is currently being revised, so hold off on corrections until
this line is removed]\\

Let $F: \mathbb{R}\to \mathbb{R}$. Then $F$ is a \emph{distribution function} if
\begin{enumerate}
\item
$F$ is nondecreasing,
\item
$F$ is continuous from the right,
\item
$\lim_{x \rightarrow -\infty} F(x) = 0$, and $\lim_{x \rightarrow \infty} F(x) = 1$.
\end{enumerate}

As an example, suppose that $\Omega = \mathbb{R}$ and that $\mathcal{B}$
is the $\sigma$-algebra of Borel subsets of $\mathbb{R}$. 
Let $P$ be a probability measure on $(\Omega, \mathcal{B})$. 
Define $F$ by
$$
F(x) = P((-\infty, x]).
$$
This particular $F$ is called the \emph{distribution function} of $P$. It is
easy to verify that 1,2, and 3 hold for this $F$.

In fact, every distribution function is the distribution function of some
probability measure on the Borel subsets of $\mathbb{R}$. To see this, 
suppose that $F$ is a distribution function. We can define $P$ on a single half-open 
interval by 
$$
P((a,b]) = F(b) - F(a)
$$
and extend  $P$ to unions of disjoint intervals by 
$$
P( \cup_{i=1}^\infty (a_i, b_i])= \sum_{i=1}^\infty P((a_i, b_i]).
$$
and then further extend $P$ to all the Borel subsets of $\mathbb{R}$.
It is clear that the distribution function of $P$ is $F$.

\subsection{Random Variables}

Suppose that $(\Omega, \mathcal{B}, P)$ is a probability space and
$X: \Omega \to \mathbb{R}$ is a random variable. Then there is an
\emph{induced} probability measure $P_X$ on $\mathbb{R}$ defined as 
follows: \\
$$
P_X(E) = P(X^{-1}(E))
$$
for every Borel subset $E$ of $\mathbb{R}$. $P_X$ is called the
\emph{distribution} of $X$. The \emph{distribution function}
of $X$ is 
$$
F_X(x) = P(\omega | X(\omega) \leq x).
$$
The distribution function of $X$ is also known as the law of $X$.
Claim: $F_X$ = the distribution function of $P_X$.


\begin{eqnarray*}
F_X(x) &=& P(\omega | X(\omega) \leq x) \\
&=& P(X^{-1}((-\infty, x]) \\
&=& P_X((-\infty, x]) \\
&=& F(x).
\end{eqnarray*}



\subsection{Density Functions}

Suppose that $f: \mathbb{R} \to \mathbb{R}$ is a nonnegative  function
such that
$$
\int_{-\infty}^\infty f(t)dt=1.
$$
Then one can define $F: \mathbb{R} \to \mathbb{R}$ by
$$
F(x) = \int_{-\infty}^x f(t)dt.
$$
Then it is clear that $F$ satisfies the conditions 1,2,and 3 so $F$ 
is a distribution function. The function $f$ is called a density function
for the distribution $F$.

If $X$ is a discrete random variable with density function $f$ and distribution 
function $F$ then

$$F(x)=\sum_{x_j\leq x} f(x_j).$$




%%%%%
%%%%%
\end{document}
