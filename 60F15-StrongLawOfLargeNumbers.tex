\documentclass[12pt]{article}
\usepackage{pmmeta}
\pmcanonicalname{StrongLawOfLargeNumbers}
\pmcreated{2013-03-22 13:13:10}
\pmmodified{2013-03-22 13:13:10}
\pmowner{Koro}{127}
\pmmodifier{Koro}{127}
\pmtitle{strong law of large numbers}
\pmrecord{11}{33685}
\pmprivacy{1}
\pmauthor{Koro}{127}
\pmtype{Definition}
\pmcomment{trigger rebuild}
\pmclassification{msc}{60F15}
\pmrelated{MartingaleProofOfKolmogorovsStrongLawForSquareIntegrableVariables}

\endmetadata

% this is the default PlanetMath preamble.  as your knowledge
% of TeX increases, you will probably want to edit this, but
% it should be fine as is for beginners.

% almost certainly you want these
\usepackage{amssymb}
\usepackage{amsmath}
\usepackage{amsfonts}

% used for TeXing text within eps files
%\usepackage{psfrag}
% need this for including graphics (\includegraphics)
%\usepackage{graphicx}
% for neatly defining theorems and propositions
%\usepackage{amsthm}
% making logically defined graphics
%%%\usepackage{xypic}

% there are many more packages, add them here as you need them

% define commands here
\begin{document}
A sequence of random variables $X_1, X_2,\dots$ with finite expectations
in a probability space is said to satisfiy the \textit{strong law of large numbers} if

$$ \frac{1}{n}\sum_{k=1}^n (X_k -\operatorname{E}[X_k]) \xrightarrow[]{a.s.} 0, $$

where $a.s.$ stands for convergence almost surely.

When the random variables are identically distributed, with expectation $\mu$,
the law becomes:

$$ \frac{1}{n}\sum_{k=1}^n X_k\xrightarrow[]{a.s.} \mu.$$

Kolmogorov's strong law of large numbers theorems give conditions on the random variables under which the law is satisfied.
%%%%%
%%%%%
\end{document}
