\documentclass[12pt]{article}
\usepackage{pmmeta}
\pmcanonicalname{MomentGeneratingFunction}
\pmcreated{2013-03-22 11:53:51}
\pmmodified{2013-03-22 11:53:51}
\pmowner{mathcam}{2727}
\pmmodifier{mathcam}{2727}
\pmtitle{moment generating function}
\pmrecord{10}{30512}
\pmprivacy{1}
\pmauthor{mathcam}{2727}
\pmtype{Definition}
\pmcomment{trigger rebuild}
\pmclassification{msc}{60E05}
\pmclassification{msc}{46L05}
\pmclassification{msc}{82-00}
\pmclassification{msc}{83-00}
\pmclassification{msc}{81-00}
\pmrelated{CharacteristicFunction2}
\pmrelated{CumulantGeneratingFunction}

\endmetadata

\usepackage{amssymb}
\usepackage{amsmath}
\usepackage{amsfonts}
%\usepackage{graphicx}
%%%%%\usepackage{xypic}
\begin{document}
\PMlinkescapeword{words}

Given a random variable $X$, the \emph{moment generating function} of $X$ is the following function:\\
\par
$M_X(t) = E[e^{tX}]$ for $t \in R$ (if the expectation converges).
\par
\par
It can be shown that if the moment generating function of $X$ is defined on an interval around the origin, then\\
\par
$E[X^k] = M_X^{(k)}(t) |_{t=0} $\\
\par
In other words, the $k$th-derivative of the moment generating function evaluated at zero is the $k$th moment of $X$.
%%%%%
%%%%%
%%%%%
%%%%%
\end{document}
