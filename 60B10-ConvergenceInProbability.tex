\documentclass[12pt]{article}
\usepackage{pmmeta}
\pmcanonicalname{ConvergenceInProbability}
\pmcreated{2013-03-22 15:01:05}
\pmmodified{2013-03-22 15:01:05}
\pmowner{CWoo}{3771}
\pmmodifier{CWoo}{3771}
\pmtitle{convergence in probability}
\pmrecord{7}{36725}
\pmprivacy{1}
\pmauthor{CWoo}{3771}
\pmtype{Definition}
\pmcomment{trigger rebuild}
\pmclassification{msc}{60B10}
\pmsynonym{converge in probability}{ConvergenceInProbability}
\pmsynonym{converges in measure}{ConvergenceInProbability}
\pmsynonym{converge in measure}{ConvergenceInProbability}
\pmsynonym{convergence in measure}{ConvergenceInProbability}
\pmdefines{converges in probability}

% this is the default PlanetMath preamble.  as your knowledge
% of TeX increases, you will probably want to edit this, but
% it should be fine as is for beginners.

% almost certainly you want these
\usepackage{amssymb,amscd}
\usepackage{amsmath}
\usepackage{amsfonts}

% used for TeXing text within eps files
%\usepackage{psfrag}
% need this for including graphics (\includegraphics)
%\usepackage{graphicx}
% for neatly defining theorems and propositions
%\usepackage{amsthm}
% making logically defined graphics
%%%\usepackage{xypic}

% there are many more packages, add them here as you need them

% define commands here
\begin{document}
Let $\lbrace X_i \rbrace$ be a sequence of random variables defined on a probability space $(\Omega,\mathcal{F},P)$ taking values in a separable metric
space $(Y,d)$, where $d$ is the metric.  Then we say the sequence
$X_i$ \emph{converges in probability} or \emph{converges in measure} to a random variable $X$ if
for every $\varepsilon>0$,
$$\lim_{i\rightarrow\infty}P(d(X_i,X)\geq\varepsilon)=0.$$
We denote convergence in probability of $X_i$ to $X$ by
$$X_i\stackrel{pr}{\longrightarrow} X.$$
Equivalently, $X_i\stackrel{pr}{\longrightarrow} X$ iff every subsequence of $\lbrace X_i\rbrace$ contains a subsequence which converges to $X$ almost surely.

\textbf{Remarks}.  
\begin{itemize}
\item Unlike ordinary convergence, $X_i\stackrel{pr}{\longrightarrow} X$ and $X_i\stackrel{pr}{\longrightarrow} Y$ only implies that $X=Y$ almost surely.
\item The need for separability on $Y$ is to ensure that the metric, $d(X_i,X)$, is a random variable, for all random variables $X_i$ and $X$.
\item Convergence almost surely implies convergence in probability but not conversely.
\end{itemize}

\begin{thebibliography}{8}
\bibitem{dudley} R. M. Dudley, {\em Real Analysis and Probability}, Cambridge University Press (2002).
\bibitem{feller} W. Feller, {\em An Introduction to Probability Theory and Its Applications. Vol. 1}, Wiley, 3rd ed. (1968).
\end{thebibliography}
%%%%%
%%%%%
\end{document}
