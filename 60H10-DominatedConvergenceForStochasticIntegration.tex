\documentclass[12pt]{article}
\usepackage{pmmeta}
\pmcanonicalname{DominatedConvergenceForStochasticIntegration}
\pmcreated{2013-03-22 18:41:03}
\pmmodified{2013-03-22 18:41:03}
\pmowner{gel}{22282}
\pmmodifier{gel}{22282}
\pmtitle{dominated convergence for stochastic integration}
\pmrecord{8}{41435}
\pmprivacy{1}
\pmauthor{gel}{22282}
\pmtype{Theorem}
\pmcomment{trigger rebuild}
\pmclassification{msc}{60H10}
\pmclassification{msc}{60G07}
\pmclassification{msc}{60H05}
%\pmkeywords{stochastic integral}
%\pmkeywords{dominated convergence}
%\pmkeywords{ucp convergence}
%\pmkeywords{semimartingale topology}
\pmdefines{locally bounded convergence theorem}

\endmetadata

% almost certainly you want these
\usepackage{amssymb}
\usepackage{amsmath}
\usepackage{amsfonts}

% used for TeXing text within eps files
%\usepackage{psfrag}
% need this for including graphics (\includegraphics)
%\usepackage{graphicx}
% for neatly defining theorems and propositions
\usepackage{amsthm}
% making logically defined graphics
%%%\usepackage{xypic}

% there are many more packages, add them here as you need them

% define commands here
\newtheorem*{theorem*}{Theorem}
\newtheorem*{lemma*}{Lemma}
\newtheorem*{corollary*}{Corollary}
\newtheorem*{definition*}{Definition}
\newtheorem{theorem}{Theorem}
\newtheorem{lemma}{Lemma}
\newtheorem{corollary}{Corollary}
\newtheorem{definition}{Definition}

\begin{document}
\PMlinkescapeword{states}
\PMlinkescapeword{sequence}
\PMlinkescapeword{limit}
\PMlinkescapeword{similar}
\PMlinkescapeword{bounded}
\PMlinkescapeword{locally bounded}

The dominated convergence theorem for standard integration states that if a sequence of measurable functions converge to a limit, and are dominated by an integrable function, then their integrals converge to the integral of the limit. That is, the limit commutes with integration. A similar result holds for stochastic integration with respect to a semimartingale $X$, except the integrals are random variables, and the integrals converge in probability.

\begin{theorem*}[Dominated convergence]
If $\xi^n$ are predictable processes converging pointwise to $\xi$, and $|\xi^n|\le\alpha$ for every $n$ and some $X$-integrable process $\alpha$, then
\begin{equation}\label{eq:1}
\int_0^t\xi^n\,dX\rightarrow\int_0^t\xi\,dX
\end{equation}
in probability as $n\rightarrow\infty$. Furthermore, ucp convergence and semimartingale convergence hold.
\end{theorem*}

Note that as $\xi$ and $\xi^n$ are bounded by an $X$-integrable process, they are guaranteed to also be $X$-integrable. Convergence in probability for each $t$ was taken as part of the definition of the stochastic integral, but the dominated convergence theorem stated here says that the stronger ucp and semimartingale convergence also hold.

If $\alpha$ is a \PMlinkname{locally bounded}{LocalPropertiesOfProcesses} predictable process, then it is automatically $X$-integrable for any semimartingale $X$. It follows that if $\xi^n$ are predictable processes converging to $\xi$ and if $\sup_n|\xi^n|$ is locally bounded then the limit (\ref{eq:1}) holds. This result is sometimes known as the \emph{locally bounded convergence theorem}.


To prove this result, it is enough to show that semimartingale convergence holds, as semimartingale convergence implies ucp convergence.
So, let $|\alpha^n|\le 1$ be a sequence of simple predictable processes and set $Y^n=\int\xi^n\,dX$, $Y=\int\xi\,dX$. Associativity of stochastic integration gives
\begin{equation*}
\int_0^t\alpha^n\,dY^n-\int_0^t\alpha^n\,dY
=\int_0^t \alpha^n(\xi^n-\xi)\,dX
\end{equation*}
However, $|\alpha^n(\xi^n-\xi)|\le 2\alpha$, which is $X$-integrable. So, this converges to zero in probability by the definition of the stochastic integral, and $Y^n\rightarrow Y$ in the semimartingale topology.

%%%%%
%%%%%
\end{document}
