\documentclass[12pt]{article}
\usepackage{pmmeta}
\pmcanonicalname{StochasticIntegrationByParts}
\pmcreated{2013-03-22 18:41:35}
\pmmodified{2013-03-22 18:41:35}
\pmowner{gel}{22282}
\pmmodifier{gel}{22282}
\pmtitle{stochastic integration by parts}
\pmrecord{4}{41453}
\pmprivacy{1}
\pmauthor{gel}{22282}
\pmtype{Theorem}
\pmcomment{trigger rebuild}
\pmclassification{msc}{60H05}
\pmclassification{msc}{60G07}
\pmclassification{msc}{60H10}
%\pmkeywords{stochastic integral}
%\pmkeywords{quadratic covariation}
%\pmkeywords{semimartingale}

% almost certainly you want these
\usepackage{amssymb}
\usepackage{amsmath}
\usepackage{amsfonts}

% used for TeXing text within eps files
%\usepackage{psfrag}
% need this for including graphics (\includegraphics)
%\usepackage{graphicx}
% for neatly defining theorems and propositions
\usepackage{amsthm}
% making logically defined graphics
%%%\usepackage{xypic}

% there are many more packages, add them here as you need them

% define commands here
\newtheorem*{theorem*}{Theorem}
\newtheorem*{lemma*}{Lemma}
\newtheorem*{corollary*}{Corollary}
\newtheorem*{definition*}{Definition}
\newtheorem{theorem}{Theorem}
\newtheorem{lemma}{Lemma}
\newtheorem{corollary}{Corollary}
\newtheorem{definition}{Definition}

\begin{document}
\PMlinkescapeword{satisfies}
\PMlinkescapeword{formula}
\PMlinkescapeword{integral}
\PMlinkescapeword{difference}
\PMlinkescapeword{term}
\PMlinkescapeword{size}
\PMlinkescapeword{product}
\PMlinkescapeword{simple}
\PMlinkescapeword{calculus}
\PMlinkescapeword{even}
\PMlinkescapeword{order}
\PMlinkescapeword{finite}
\PMlinkescapeword{variation}
\PMlinkescapeword{identity}

The stochastic integral satisfies a version of the classical integration by parts formula, which is just the integral version of the product rule. The only difference here is the existence of a quadratic covariation term.

\begin{theorem*}
Let $X,Y$ be semimartingales. Then,
\begin{equation}\label{eq:1}
X_tY_t = X_0Y_0+\int_0^t X_{s-}\,dY_s+\int_0^tY_{s-}\,dX_s + [X,Y]_t.
\end{equation}
\end{theorem*}

Alternatively, in differential notation, this reads
\begin{equation*}
d(X_tY_t)=X_{t-}dY_t + Y_{t-}dX_t + d[X,Y]_t.
\end{equation*}
The existence of the quadratic covariation term $[X,Y]$ in the integration by parts formula, and also in It\^o's lemma, is an important difference between standard calculus and stochastic calculus. To see the need for this term, consider the following. Choosing any $h>0$, write the increment of a process over a time step of size $h$ as $\delta X_t\equiv X_{t+h}-X_t$. The increment of a product of processes satisfies the following simple identity,
\begin{equation}\label{eq:2}
\delta (XY)_t = X_t\delta Y_t + Y_t\delta X_t + \delta X_t\delta Y_t. 
\end{equation}
As we let $h$ tend to zero, for differentiable processes the final term of (\ref{eq:2}) is of \PMlinkname{order}{LandauNotation} $O(h^2)$, so can be neglected in the limit. However, when $X$ and $Y$ are semimartingales, such as Brownian motion, the final term will be of order $h$, and needs to be retained even in the limit.

The proof of equation (\ref{eq:1}) is given by the proof of the \PMlinkname{existence of the quadratic variation of semimartingales}{QuadraticVariationOfASemimartingale} and, in particular, is just a rearrangement of the formula given for the quadratic covariation of semimartingales. Whenever either of $X$ or $Y$ is a continuous finite variation process, the quadratic covariation term $[X,Y]$ is zero, so (\ref{eq:1}) becomes the standard integration by parts formula. More generally, for noncontinuous processes we have the following.

\begin{corollary*}
Let $X$ be a semimartingale and $Y$ be an adapted finite variation process. Then,
\begin{equation}\label{eq:3}
X_tY_t = X_0Y_0 + \int_0^t X_s\,dY_s + \int_0^t Y_{s-}\,dX_s.
\end{equation}
\end{corollary*}

As $Y$ is a finite variation process, the first integral on the right hand side of (\ref{eq:3}) makes sense as a Lebesgue-Stieltjes integral.
Equation (\ref{eq:3}) follows from the integration by parts formula by first substituting the following formula for the covariation whenever $Y$ has finite variation into (\ref{eq:1})
\begin{equation*}
[X,Y]_t=\sum_{s\le t}\Delta X_s\Delta Y_s
\end{equation*}
and then using the following identity
\begin{equation*}
\begin{split}
\int_0^t X_s\,dY_s - \int_0^t X_{s-}\,dY_s
&=\int_0^t \Delta X_s\,dY_s
=\int_0^t \sum_{u}\Delta X_u 1_{\{u=s\}}\,dY_s\\
&= \sum_{u\le t}\Delta X_u\Delta Y_u.
\end{split}\end{equation*}

%%%%%
%%%%%
\end{document}
