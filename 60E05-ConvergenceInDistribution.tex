\documentclass[12pt]{article}
\usepackage{pmmeta}
\pmcanonicalname{ConvergenceInDistribution}
\pmcreated{2013-03-22 13:14:12}
\pmmodified{2013-03-22 13:14:12}
\pmowner{Koro}{127}
\pmmodifier{Koro}{127}
\pmtitle{convergence in distribution}
\pmrecord{11}{33708}
\pmprivacy{1}
\pmauthor{Koro}{127}
\pmtype{Definition}
\pmcomment{trigger rebuild}
\pmclassification{msc}{60E05}
\pmrelated{WeakConvergence}

% this is the default PlanetMath preamble.  as your knowledge
% of TeX increases, you will probably want to edit this, but
% it should be fine as is for beginners.

% almost certainly you want these
\usepackage{amssymb}
\usepackage{amsmath}
\usepackage{amsfonts}

% used for TeXing text within eps files
%\usepackage{psfrag}
% need this for including graphics (\includegraphics)
%\usepackage{graphicx}
% for neatly defining theorems and propositions
%\usepackage{amsthm}
% making logically defined graphics
%%%\usepackage{xypic}

% there are many more packages, add them here as you need them

% define commands here
\begin{document}
A sequence of distribution functions $F_1,F_2,\dots$ converges \emph{weakly} to a distribution 
function $F$ if $F_n(t)\rightarrow F(t)$ for each point $t$ at which $F$ is continuous. 

If the random variables $X,X_1,X_2,\dots$ have associated distribution functions 
$F,F_1,F_2,\dots$, respectively, then we say that $X_n$ converges \emph{in distribution} to 
$X$, and denote this by $X_n\xrightarrow[]{D} X$.

This definition holds for joint distribution functions and random vectors as well.

This is probably the weakest \PMlinkescapetext{type} of convergence of random variables. Some results involving this \PMlinkescapetext{type} of convergence 
are the central limit theorems, Helly-Bray theorem, Paul L\'evy continuity theorem, Cram\'er-Wold theorem and Scheff\'e's theorem.
%%%%%
%%%%%
\end{document}
