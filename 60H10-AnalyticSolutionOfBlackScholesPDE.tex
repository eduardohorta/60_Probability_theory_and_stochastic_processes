\documentclass[12pt]{article}
\usepackage{pmmeta}
\pmcanonicalname{AnalyticSolutionOfBlackScholesPDE}
\pmcreated{2013-03-22 16:31:34}
\pmmodified{2013-03-22 16:31:34}
\pmowner{stevecheng}{10074}
\pmmodifier{stevecheng}{10074}
\pmtitle{analytic solution of Black-Scholes PDE}
\pmrecord{6}{38706}
\pmprivacy{1}
\pmauthor{stevecheng}{10074}
\pmtype{Derivation}
\pmcomment{trigger rebuild}
\pmclassification{msc}{60H10}
\pmclassification{msc}{91B28}
\pmrelated{ExampleOfSolvingTheHeatEquation}
\pmrelated{BlackScholesPDE}
\pmrelated{BlackScholesFormula}

% The standard font packages
\usepackage{amssymb}
\usepackage{amsmath}
\usepackage{amsfonts}

% For neatly defining theorems and definitions
%\usepackage{amsthm}

% Including EPS/PDF graphics (\includegraphics)
%\usepackage{graphicx}

% Making matrix-based graphics
%%%\usepackage{xypic}

% Enumeration lists with different styles
%\usepackage{enumerate}

% Set up the theorem environments
%\newtheorem{thm}{Theorem}
%\newtheorem*{thm*}{Theorem}

\providecommand{\defnterm}[1]{\emph{#1}}

% The standard number systems
\newcommand{\complex}{\mathbb{C}}
\newcommand{\real}{\mathbb{R}}
\newcommand{\rat}{\mathbb{Q}}
\newcommand{\nat}{\mathbb{N}}
\newcommand{\intset}{\mathbb{Z}}

% Absolute values and norms
% Normal, wide, and big versions of the delimeters
\providecommand{\abs}[1]{\lvert#1\rvert}
\providecommand{\absW}[1]{\left\lvert#1\right\rvert}
\providecommand{\absB}[1]{\Bigl\lvert#1\Bigr\rvert}
\providecommand{\norm}[1]{\lVert#1\rVert}
\providecommand{\normW}[1]{\left\lVert#1\right\rVert}
\providecommand{\normB}[1]{\Bigl\lVert#1\Bigr\rVert}

% Differentiation operators
\providecommand{\od}[2]{\frac{d #1}{d #2}}
\providecommand{\pd}[2]{\frac{\partial #1}{\partial #2}}
\providecommand{\pdd}[2]{\frac{\partial^2 #1}{\partial #2}}
\providecommand{\ipd}[2]{\partial #1 / \partial #2}

% Differentials on integrals
\newcommand{\dx}{\, dx}
\newcommand{\dt}{\, dt}
\newcommand{\dmu}{\, d\mu}

% Inner products
\providecommand{\ip}[2]{\langle {#1}, {#2} \rangle}

% Calligraphic letters
\newcommand{\sF}{\mathcal{F}}
\newcommand{\sD}{\mathcal{D}}

% Standard spaces
\newcommand{\Hilb}{\mathcal{H}}
\newcommand{\Le}{\mathbf{L}}

% Operators and functions occassionally used in my articles
\DeclareMathOperator{\D}{D}
\DeclareMathOperator{\linspan}{span}
\DeclareMathOperator{\rank}{rank}
\DeclareMathOperator{\lindim}{dim}
\DeclareMathOperator{\sinc}{sinc}

% Probability stuff
\newcommand{\PP}{\mathbb{P}}
\newcommand{\E}{\mathbb{E}}

\begin{document}
\PMlinkescapeword{relations}
\PMlinkescapeword{terms}
\PMlinkescapeword{terminal}

Here we present
an analytical 
solution for the \defnterm{Black-Scholes partial differential equation},
\begin{align}\label{eq:bs-pde}
rf = \pd{f}{t} + rx \, \pd{f}{x} + \frac12 \sigma^2 x^2 \, \pdd{f}{x^2}\,,
\quad f = f(t,x)\,, 
\end{align}
over the domain $0 < x < \infty, \: 0 \leq t \leq T$,
with terminal condition $f(T,x) = \psi(x)$,
by reducing this parabolic PDE to
the heat equation of physics.

We begin by making the substitution:
\[
u = e^{-rt} \, f\,,
\]
which is motivated by the fact that it is the portfolio value
\emph{discounted by the interest rate $r$} (see the derivation of the 
Black-Scholes formula)
that  is a martingale.
Using the product rule on $f = e^{rt} \, u$,
we derive the PDE that the function $u$ must satisfy:
\[
rf = re^{rt} \, u = re^{rt} \, u + e^{rt} \pd{u}{t}
+ rx e^{rt} \pd{u}{x} + \frac12 \sigma^2 x^2 e^{rt} \pdd{u}{x^2}\,;
\]
or simply,
\begin{align}\label{eq:pde-u}
0 = \pd{u}{t} + rx \pd{u}{x} + \frac12 \sigma^2 x^2 \pdd{u}{x^2}\,.
\end{align}

Next, we make the substitutions:
\[
y = \log x\,, \quad s = T-t\,.
\]
These changes of variables can be motivated
by observing that:
\begin{itemize}
\item
The underlying process described by the variable $x$
is a geometric Brownian motion
(as explained in the derivation of the Black-Scholes formula itself),
so that $\log x$ describes a Brownian motion, possibly
with a drift.
Then $\log x$ should satisfy some sort of diffusion equation
(well-known in physics).
\item
The evolution of the system
is backwards from the terminal state of the system.  Indeed,
the boundary condition is given as a terminal state,
and the coefficient of $\ipd{u}{t}$ is positive in equation \eqref{eq:pde-u}.
(Compare with the standard heat equation,
$0 = -\ipd{u}{t} + \ipd{u}{x}$, which describes a
temperature \PMlinkescapetext{distribution} evolving forwards in time.)
So to get to the heat equation, 
we have to use a substitution to reverse time.
\end{itemize}

Since
\[
\pd{u}{s} = -\pd{u}{t}\,, \quad
 \pd{u}{x} = \pd{u}{y} \, \od{y}{x} = \frac{1}{x} \, \pd{u}{y}\,,
\]
and
\[
\pdd{u}{x^2} = \pd{}{x} \left( \frac{1}{x} \pd{u}{y} \right)
= -\frac{1}{x^2} \, \pd{u}{y} + \frac{1}{x^2} \pdd{u}{y^2}\,,
\]
substituting in equation \eqref{eq:pde-u},
we find:
\begin{align}\label{eq:pde-y}
0 = -\pd{u}{s} + (r- \tfrac12 \sigma^2) \pd{u}{y} + \frac12 \sigma^2 \, \pdd{u}{y^2}\,.
\end{align}
The first partial derivative with respect to $y$
does not cancel (unless $r = \tfrac12 \sigma^2$) 
because we have not take into account
the drift of the Brownian motion.
To cancel the drift (which is linear in time),
we make the substitutions:
\[
z = y + (r - \tfrac12 \sigma^2) \tau\,, \quad \tau = s\,.
\]
Under the new coordinate system $(z,\tau)$, we have the relations
amongst vector fields:
\[
\pd{}{z} = \pd{}{y}\,, \quad \pd{}{\tau} = -(r-\tfrac12 \sigma^2) \pd{}{y} + \pd{}{s}\,,
\]
leading to the following \PMlinkescapetext{transformation}
of equation \eqref{eq:pde-y}:
\begin{align*}
0 = -\pd{u}{\tau} -(r-\tfrac12 \sigma^2) \pd{u}{z}
+ (r-\tfrac12 \sigma^2) \pd{u}{z} + \frac12 \sigma^2 \pdd{u}{z^2}\,;
\end{align*}
or:
\begin{align}\label{eq:diffusion}
\pd{u}{\tau} = \frac12 \sigma^2 \pdd{u}{z^2}\,, \quad u = u(\tau, z)\,,
\end{align}
which is one form of the diffusion equation.
The domain is on
$-\infty < z < \infty$
and $0 \leq \tau \leq T$;
the initial condition is to be:
\begin{align*}
u(0, z) = e^{-rT} \, \psi(e^{z} ) := u_0(z)\,.
\end{align*}
The original function $f$ can be recovered
by \[
f(t, x) = e^{rt} \, u\Bigl(T-t,\: \log x + (r-\tfrac12 \sigma^2) \tau  \Bigr)\,.
\]

The fundamental solution of the PDE \eqref{eq:diffusion}
is known to be:
\[
G_\tau(z) = \frac{1}{\sqrt{2\pi \sigma^2 \tau}} \exp\Bigl( -\frac{z}{2\sigma^2 \tau} \Bigr)
\]
(derived using the Fourier transform);
and the solution $u$ with initial condition $u_0$
is given by the convolution:
\[
u(\tau, z) = u_0 * G_\tau(z)
= \frac{e^{-rT}}{\sqrt{2\pi \sigma^2 \tau}} \, \int_{-\infty}^\infty \psi(e^\zeta) \,
\exp \Bigl( - \frac{(z-\zeta)^2}{2\sigma^2 \tau} \Bigr) \, d\zeta\,.
\]
In terms of the original function $f$:
\[
f(t, x) = \frac{e^{-r\tau}}{\sqrt{2\pi \sigma^2 \tau}}
\int_{-\infty}^\infty
\psi (e^\zeta)
\,
\exp \Bigl( - \frac{\bigl(\log x + (r-\frac12 \sigma^2)\tau -\zeta\bigr)^2}
{2\sigma^2 \tau} \Bigr) \, d\zeta\,,
\]
($\tau = T -t$) which agrees with the \PMlinkname{result derived using probabilistic methods}{BlackScholesFormula}.


%%%%%
%%%%%
\end{document}
