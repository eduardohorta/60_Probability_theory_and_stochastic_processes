\documentclass[12pt]{article}
\usepackage{pmmeta}
\pmcanonicalname{CadlagProcess}
\pmcreated{2013-03-22 18:36:36}
\pmmodified{2013-03-22 18:36:36}
\pmowner{gel}{22282}
\pmmodifier{gel}{22282}
\pmtitle{c\`adl\`ag process}
\pmrecord{7}{41343}
\pmprivacy{1}
\pmauthor{gel}{22282}
\pmtype{Definition}
\pmcomment{trigger rebuild}
\pmclassification{msc}{60G07}
\pmsynonym{cadlag process}{CadlagProcess}
\pmsynonym{rcll process}{CadlagProcess}
\pmsynonym{R-process}{CadlagProcess}
\pmsynonym{right-process}{CadlagProcess}
%\pmkeywords{stochastic process}
\pmrelated{UcpConvergenceOfProcesses}
\pmdefines{cadlag}
\pmdefines{rcll}
\pmdefines{R-process}
\pmdefines{right-process}
\pmdefines{c\`agl\`ad}
\pmdefines{lcrl}
\pmdefines{L-process}

% almost certainly you want these
\usepackage{amssymb}
\usepackage{amsmath}
\usepackage{amsfonts}

% used for TeXing text within eps files
%\usepackage{psfrag}
% need this for including graphics (\includegraphics)
%\usepackage{graphicx}
% for neatly defining theorems and propositions
\usepackage{amsthm}
% making logically defined graphics
%%%\usepackage{xypic}

% there are many more packages, add them here as you need them

% define commands here
\newtheorem*{theorem*}{Theorem}
\newtheorem*{lemma*}{Lemma}
\newtheorem*{corollary*}{Corollary}
\newtheorem*{definition*}{Definition}
\newtheorem{theorem}{Theorem}
\newtheorem{lemma}{Lemma}
\newtheorem{corollary}{Corollary}
\newtheorem{definition}{Definition}

\begin{document}
\PMlinkescapeword{word}
\PMlinkescapeword{types}
\PMlinkescapeword{terms}
A c\`adl\`ag process $X$ is a stochastic process for which the paths $t\mapsto X_t$ are right-continuous with left limits everywhere, with probability one. The word \emph{c\`adl\`ag} is an acronym from the French for ``continu \`a droite, limites \`a gauche''.
Such processes are widely used in the theory of noncontinuous stochastic processes. For example, semimartingales are c\`adl\`ag, and continuous-time martingales and many types of Markov processes have c\`adl\`ag modifications.

Given a c\`adl\`ag process $X_t$ with time index $t$ ranging over the nonnegative real numbers, its left limits are often denoted by
\begin{equation*}
X_{t-}=\lim_{\substack{s\rightarrow t,\\ s<t}}X_s
\end{equation*}
for every $t>0$. Also, the jump at time $t$ is written as
\begin{equation*}
\Delta X_t = X_t-X_{t-}.
\end{equation*}

Alternative terms used to refer to a c\`adl\`ag process are \emph{rcll} (right-continuous with left limits), \emph{R-process} and \emph{right-process}.

Although used less frequently, a process whose paths are almost surely left-continuous with right limits everywhere are known as \emph{c\`agl\`ad}, \emph{lcrl} or \emph{L-processes}.

%%%%%
%%%%%
\end{document}
