\documentclass[12pt]{article}
\usepackage{pmmeta}
\pmcanonicalname{AlmostEverywhere}
\pmcreated{2013-03-22 12:20:58}
\pmmodified{2013-03-22 12:20:58}
\pmowner{mathcam}{2727}
\pmmodifier{mathcam}{2727}
\pmtitle{almost everywhere}
\pmrecord{7}{32002}
\pmprivacy{1}
\pmauthor{mathcam}{2727}
\pmtype{Definition}
\pmcomment{trigger rebuild}
\pmclassification{msc}{60A10}
\pmsynonym{almost surely}{AlmostEverywhere}
\pmsynonym{a.s.}{AlmostEverywhere}
\pmsynonym{a.e.}{AlmostEverywhere}
\pmsynonym{almost all}{AlmostEverywhere}

\endmetadata

\usepackage{amssymb}
\usepackage{amsmath}
\usepackage{amsfonts}

% used for TeXing text within eps files
%\usepackage{psfrag}
% need this for including graphics (\includegraphics)
%\usepackage{graphicx}
% for neatly defining theorems and propositions
%\usepackage{amsthm}
% making logically defined graphics
%%%\usepackage{xypic} 

% there are many more packages, add them here as you need them

% define commands here
\newcommand{\mv}[1]{\mathbf{#1}}	% matrix or vector
\newcommand{\mvt}[1]{\mv{#1}^{\mathrm{T}}}
\newcommand{\mvi}[1]{\mv{#1}^{-1}}
\newcommand{\mpderiv}[1]{\frac{\partial}{\partial {#1}}}
\newcommand{\borel}{\mathfrak{B}}
\newcommand{\reals}{\mathbb{R}}
\newcommand{\defined}{:=}
\newcommand{\var}{\mathrm{var}}
\newcommand{\cov}{\mathrm{cov}}
\newcommand{\corr}{\mathrm{corr}}
\newcommand{\set}[1]{\{#1\}}
\begin{document}
Let $(X, \borel, \mu)$ be a measure space.  A condition holds \emph{almost everywhere} on $X$ if it holds ``with probability $1$,'' i.e. if it holds everywhere except for a subset of $X$ with measure $0$.  For example, let $f$ and $g$ be nonnegative functions on $X$.  Suppose we want a sufficient condition on functions $f(x)$ and $g(x)$ such that the relation
\begin{equation}
\int_{X} f d\mu(x) \le \int_{X} g d\mu(x) 
\end{equation}
holds.  Certainly $f(x)\leq g(x)$ for all $x\in X$ is a sufficient condition, but in fact it's enough to have $f(x)\leq g(x)$ almost surely on $X$.  In fact, we can loosen the above non-negativity condition to only require that $f$ and $g$ are almost surely nonnegative as well.

If $X = [0,1]$, then $g$ might be less than $f$ on the Cantor set, an uncountable set with measure $0$, and still satisfy the condition.  We say that $f \le g$ almost everywhere (often abbreviated \emph{a.e.}).   

Note that this \PMlinkescapetext{term} is the \PMlinkescapetext{equivalent} of the \PMlinkescapetext{term} ``almost surely'' from probabilistic measure \PMlinkescapetext{theory}.
%%%%%
%%%%%
\end{document}
