\documentclass[12pt]{article}
\usepackage{pmmeta}
\pmcanonicalname{AssociativityOfStochasticIntegration}
\pmcreated{2013-03-22 18:41:06}
\pmmodified{2013-03-22 18:41:06}
\pmowner{gel}{22282}
\pmmodifier{gel}{22282}
\pmtitle{associativity of stochastic integration}
\pmrecord{4}{41436}
\pmprivacy{1}
\pmauthor{gel}{22282}
\pmtype{Theorem}
\pmcomment{trigger rebuild}
\pmclassification{msc}{60H05}
\pmclassification{msc}{60G07}
\pmclassification{msc}{60H10}
%\pmkeywords{stochastic integral}
%\pmkeywords{semimartingale}

\endmetadata

% almost certainly you want these
\usepackage{amssymb}
\usepackage{amsmath}
\usepackage{amsfonts}

% used for TeXing text within eps files
%\usepackage{psfrag}
% need this for including graphics (\includegraphics)
%\usepackage{graphicx}
% for neatly defining theorems and propositions
\usepackage{amsthm}
% making logically defined graphics
%%%\usepackage{xypic}

% there are many more packages, add them here as you need them

% define commands here
\newtheorem*{theorem*}{Theorem}
\newtheorem*{lemma*}{Lemma}
\newtheorem*{corollary*}{Corollary}
\newtheorem*{definition*}{Definition}
\newtheorem{theorem}{Theorem}
\newtheorem{lemma}{Lemma}
\newtheorem{corollary}{Corollary}
\newtheorem{definition}{Definition}

\begin{document}
\PMlinkescapeword{terms}
\PMlinkescapeword{derivative}
\PMlinkescapeword{satisfies}

The chain rule for expressing the derivative of a variable $z$ with respect to $x$ in terms of a third variable $y$ is
\begin{equation*}
\frac{dz}{dx}=\frac{dz}{dy}\frac{dy}{dx}.
\end{equation*}
Equivalently, if $dy=\alpha\,dx$ and $dz=\beta\,dy$ then $dz=\beta\alpha\,dx$.
The following theorem shows that the stochastic integral satisfies a generalization of this.

\begin{theorem*}
Let $X$ be a semimartingale and $\alpha$ be an $X$-integrable process. Setting $Y=\int\alpha\,dX$ then $Y$ is a semimartingale. Furthermore, a predictable process $\beta$ is $Y$-integrable if and only if $\beta\alpha$ is $X$-integrable, in which case
\begin{equation}\label{eq:1}
\int\beta\,dY=\int\beta\alpha\,dX.
\end{equation}
\end{theorem*}
Note that expressed in alternative notation, (\ref{eq:1}) becomes
\begin{equation*}
\beta\cdot(\alpha\cdot X)=(\beta\alpha)\cdot X
\end{equation*}
or, in differential notional,
\begin{equation*}
\beta(\alpha\,dX)=(\beta\alpha)\,dX.
\end{equation*}
That is, stochastic integration is associative.

%%%%%
%%%%%
\end{document}
