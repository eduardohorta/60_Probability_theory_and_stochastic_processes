\documentclass[12pt]{article}
\usepackage{pmmeta}
\pmcanonicalname{StochasticDifferentialEquation}
\pmcreated{2013-03-22 16:10:07}
\pmmodified{2013-03-22 16:10:07}
\pmowner{stevecheng}{10074}
\pmmodifier{stevecheng}{10074}
\pmtitle{stochastic differential equation}
\pmrecord{13}{38252}
\pmprivacy{1}
\pmauthor{stevecheng}{10074}
\pmtype{Definition}
\pmcomment{trigger rebuild}
\pmclassification{msc}{60H10}
\pmclassification{msc}{34-00}
\pmsynonym{SDE}{StochasticDifferentialEquation}
\pmrelated{ItoIntegral}
\pmrelated{WienerProcess}
\pmrelated{BrownianMotion}

\endmetadata

% The standard font packages
\usepackage{amssymb}
\usepackage{amsmath}
\usepackage{amsfonts}

% For neatly defining theorems and definitions
%\usepackage{amsthm}

% Including EPS/PDF graphics (\includegraphics)
%\usepackage{graphicx}

% Making matrix-based graphics
%%%\usepackage{xypic}

% Enumeration lists with different styles
%\usepackage{enumerate}

% Set up the theorem environments
%\newtheorem{thm}{Theorem}
%\newtheorem*{thm*}{Theorem}

\providecommand{\defnterm}[1]{\emph{#1}}

% The standard number systems
\newcommand{\complex}{\mathbb{C}}
\newcommand{\real}{\mathbb{R}}
\newcommand{\rat}{\mathbb{Q}}
\newcommand{\nat}{\mathbb{N}}
\newcommand{\intset}{\mathbb{Z}}

% Absolute values and norms
% Normal, wide, and big versions of the delimeters
\providecommand{\abs}[1]{\lvert#1\rvert}
\providecommand{\absW}[1]{\left\lvert#1\right\rvert}
\providecommand{\absB}[1]{\Bigl\lvert#1\Bigr\rvert}
\providecommand{\norm}[1]{\lVert#1\rVert}
\providecommand{\normW}[1]{\left\lVert#1\right\rVert}
\providecommand{\normB}[1]{\Bigl\lVert#1\Bigr\rVert}

% Differentiation operators
\providecommand{\od}[2]{\frac{d #1}{d #2}}
\providecommand{\pd}[2]{\frac{\partial #1}{\partial #2}}
\providecommand{\pdd}[2]{\frac{\partial^2 #1}{\partial #2}}
\providecommand{\ipd}[2]{\partial #1 / \partial #2}

% Differentials on integrals
\newcommand{\dx}{\, dx}
\newcommand{\dt}{\, dt}
\newcommand{\dmu}{\, d\mu}

% Inner products
\providecommand{\ip}[2]{\langle {#1}, {#2} \rangle}

% Calligraphic letters
\newcommand{\sF}{\mathcal{F}}
\newcommand{\sD}{\mathcal{D}}

% Standard spaces
\newcommand{\Hilb}{\mathcal{H}}
\newcommand{\Le}{\mathbf{L}}

% Operators and functions occassionally used in my articles
\DeclareMathOperator{\D}{D}
\DeclareMathOperator{\linspan}{span}
\DeclareMathOperator{\rank}{rank}
\DeclareMathOperator{\lindim}{dim}
\DeclareMathOperator{\sinc}{sinc}

% Probability stuff
\newcommand{\PP}{\mathbb{P}}
\newcommand{\E}{\mathbb{E}}

\begin{document}
\PMlinkescapeword{term}
\PMlinkescapeword{types}
\PMlinkescapeword{extensions}
\PMlinkescapeword{theory}
\PMlinkescapeword{satisfy}
\PMlinkescapeword{argument}
\PMlinkescapeword{relation}

Consider the ordinary differential equation, for 
example, the population growth model 
\[
\frac{dX(t)}{dt}=a(t)X(t),\, X(0)=X_0\,,
\]
where $a(t)$ is the relative rate of growth at time $t$, and $X(t)$ 
is the solution-trajectory
of the system. 

But we may want to take into account, in our model, 
the randomness or the uncertainty 
of our knowledge of the data.
In this case we may introduce the data $a(t)$
as:
\[
a(t)=r(t)+ N(t)\,,
\]
where $N(t)$ is a noise term, represented by a random variable
with some postulated probability distribution.

In general, stochastic differential equations
can be posed in the case that the infinitesimal increment $dX(t)$
is a Gaussian random variable.  (Other types of random variables are 
also possible, but require extensions of the basic theory.)
A \emph{stochastic differential equation} (SDE) is an equation
of the form:
\[
dX(t; \omega) = \mu(t; \omega) \, dt + \sigma (t; \omega) \, dW(t; \omega)
\]
where $\omega$ lives in some probability space, and $W(t)$
is a Wiener process on that probability space.
The real-valued functions $\mu$ and $\sigma$ are to satisfy certain measurability requirements, and are usually assumed to be known, with the process $X(t)$ being sought.

The argument $\omega$ is usually suppressed in the notation:
\begin{align}\label{eq:sde}
dX(t) = \mu(t) \, dt + \sigma (t) \, dW(t)\,,
\end{align}
with the understanding that $X(t)$, $W(t)$, $\mu(t)$ and $\sigma(t)$ denote
random variables for each time $t$.

The interpretation of 
the stochastic differential equation \eqref{eq:sde} is that 
a process $X(t)$ satisfies it if and only if it satisfies
this relation amongst integrals:
\begin{align}\label{eq:ito}
X(t_1) - X(t_0) = \int_{t_0}^{t_1} \mu(t) \, dt
+ \int_{t_0}^{t_1} \sigma(t) \, dW(t)
\end{align}
for all times $t_0$ and $t_1$.
The last integral is an It\^o integral.

In many cases, the coefficients $\mu$ and $\sigma$
depend on $X(t)$ itself:
\[
dX(t) = \mu(t, X(t)) \, dt + \sigma(t, X(t)) \, dW(t)\,.
\]
In this case, equation \eqref{eq:ito} does not give
an explicit solution for the stochastic differential equation.
Nevertheless, there are theorems analogous to those
of ordinary differential equations,
that guarantee existence of solutions given certain 
bounds on the growth of the coefficients $\mu(t,x)$ and $\sigma(t,x)$.

In simpler cases, stochastic differential equations that
involve unknowns on the right-hand side may still be solved
explicitly using changes of variables (often called It\^o's formula
in this context).
For example, 
\[
X(t) = X_0 \, e^{-\kappa t} + \theta \, (1 - e^{-\kappa t})
+ \sigma \int_0^t e^{-\kappa (t-s)} \, dW(s)
\]
(for any initial condition $X_0$) provides a solution to:
\[
dX(t) = \kappa\, (\theta - X(t)) \, dt + \sigma \, dW(t)\,.
\]

\begin{thebibliography}{9}
\bibitem{Oksendal:several}
Bernt \O ksendal.
{\em \PMlinkescapetext{Stochastic Differential Equations},
An Introduction with Applications.} 5th ed. Springer 1998.
\bibitem{Evans:several}
Lawrence Evans. {\em \PMlinkescapetext{An Introduction to Stochastic Differential Equations}.} Department of Mathematics, 
U.C. Berkeley.
\end{thebibliography}

%%%%%
%%%%%
\end{document}
