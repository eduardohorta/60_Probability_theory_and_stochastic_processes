\documentclass[12pt]{article}
\usepackage{pmmeta}
\pmcanonicalname{KolmogorovsMartingaleInequality}
\pmcreated{2013-03-22 17:20:22}
\pmmodified{2013-03-22 17:20:22}
\pmowner{stevecheng}{10074}
\pmmodifier{stevecheng}{10074}
\pmtitle{Kolmogorov's martingale inequality}
\pmrecord{8}{39694}
\pmprivacy{1}
\pmauthor{stevecheng}{10074}
\pmtype{Theorem}
\pmcomment{trigger rebuild}
\pmclassification{msc}{60G44}
\pmclassification{msc}{60G07}
\pmsynonym{Kolmogorov's submartingale inequality}{KolmogorovsMartingaleInequality}
\pmrelated{MarkovsInequality}
\pmrelated{DoobsInequalities}

% The standard font packages
\usepackage{amssymb}
\usepackage{amsmath}
\usepackage{amsfonts}

% For neatly defining theorems and definitions
\usepackage{amsthm}

% Including EPS/PDF graphics (\includegraphics)
%\usepackage{graphicx}

% Making matrix-based graphics
%%%\usepackage{xypic}

% Enumeration lists with different styles
%\usepackage{enumerate}

% Set up the theorem environments
%\newtheorem{thm}{Theorem}
\newtheorem*{thm*}{Theorem}

\newcommand{\defnterm}[1]{\emph{#1}}

% The standard number systems
\newcommand{\complex}{\mathbb{C}}
\newcommand{\real}{\mathbb{R}}
\newcommand{\rat}{\mathbb{Q}}
\newcommand{\nat}{\mathbb{N}}
\newcommand{\intset}{\mathbb{Z}}

% Absolute values and norms
% Normal, wide, and big versions of the delimeters
\newcommand{\abs}[1]{\lvert#1\rvert}
\newcommand{\absW}[1]{\left\lvert#1\right\rvert}
\newcommand{\absB}[1]{\Bigl\lvert#1\Bigr\rvert}
\newcommand{\norm}[1]{\lVert#1\rVert}
\newcommand{\normW}[1]{\left\lVert#1\right\rVert}
\newcommand{\normB}[1]{\Bigl\lVert#1\Bigr\rVert}

% Inverse functions
\newcommand{\inv}[1]{{#1}^{-1}}

% Differentiation operators
\newcommand{\od}[2]{\frac{d #1}{d #2}}
\newcommand{\pd}[2]{\frac{\partial #1}{\partial #2}}
\newcommand{\pdd}[2]{\frac{\partial^2 #1}{\partial #2}}
\newcommand{\ipd}[2]{\partial #1 / \partial #2}

% Differentials on integrals
\newcommand{\dx}{\, dx}
\newcommand{\dt}{\, dt}
\newcommand{\dmu}{\, d\mu}

% Inner products
\newcommand{\ip}[2]{\langle {#1}, {#2} \rangle}

% Complex numbers
\DeclareMathOperator{\zRe}{Re}
\DeclareMathOperator{\zIm}{Im}
\newcommand{\conjug}[1]{\overline{#1}}

% Calligraphic letters
\newcommand{\sF}{\mathcal{F}}
\newcommand{\sD}{\mathcal{D}}

% Standard spaces
\newcommand{\Hilb}{\mathcal{H}}
\newcommand{\Le}{\mathbf{L}}

% Operators and functions occassionally used in my articles
\DeclareMathOperator{\D}{D}
\DeclareMathOperator{\linspan}{span}
\DeclareMathOperator{\rank}{rank}
\DeclareMathOperator{\lindim}{dim}
\DeclareMathOperator{\sinc}{sinc}

% Probability stuff
\newcommand{\PP}{\mathbb{P}}
\newcommand{\E}{\mathbb{E}}

\newcommand{\indc}{\mathbf{1}}
\newtheorem*{cor*}{Corollary}
\DeclareMathOperator{\Var}{Var}

\begin{document}
\begin{thm*}[Kolmogorov's martingale inequality]
Let $X(t)$, for $0 \leq t \leq T$, be a submartingale
with continuous sample paths.
Then for any constant $\alpha > 0$,
\[
\PP \Bigl( \max_{0 \leq t \leq T} X(t) \geq \alpha \Bigr)
\leq \frac{\E[ X(T)^+ ]}{\alpha}\,.
\]
\end{thm*}
(The notation $X(T)^+$ means $\max(X(T),0)$, the positive part of $X(T)$.)

Notice the analogy with Markov's inequality.
Of course, the conclusion is much stronger than Markov's inequality,
as the probabilistic bound applies to an uncountable number
of random variables.  The continuity and submartingale hypotheses
are used to establish the stronger bound.

\begin{proof}
Let $\{ t_i \}_{i=1}^n$ be a partition of the interval $[0,T]$.
Let 
\[
B = \Bigl\{ \max_{1 \leq i \leq n} X(t_i) \geq \alpha \Bigr\}
\]
and split $B$ into disjoint parts $B_i$,
defined by
\[
B_i = \Bigl\{ X(t_j) < \alpha \text{ for all } j < i
                              \text{ but } X(t_i) \geq \alpha \Bigr\}\,.
\]
Also let $\{ \sF_t \}$ be the filtration under which $X(t)$ 
is a submartingale.

Then
\begin{align*}
\PP(B) 
 &=
   \sum_{i=1}^n \E\bigl[1(B_i)\bigr] \\
 &\leq 
  \sum_{i=1}^n \E \left[ \frac{X(t_i)}{\alpha} \, \indc(B_i) \right] 
  & \text{definition of $B_i$} \\
 &\leq 
  \frac{1}{\alpha} \sum_{i=1}^n 
    \E\Bigl[ \E \bigl[X(T) \mid \sF_{t_i} \bigr] \, \indc(B_i) \Bigr] 
  & \text{$X(t)$ is submartingale} \\
 &=
  \frac{1}{\alpha} \sum_{i=1}^n
    \E\Bigl[ \E\bigl[X(T) \, \indc(B_i) \mid \sF_{t_i} \bigr] \Bigr] 
  & \text{$B_i$ is $\sF_{t_i}$-measurable} \\
 &=
  \frac{1}{\alpha} \sum_{i=1}^n
    \E\bigl[ X(T) \, \indc(B_i) \bigr] 
  & \text{iterated expectation} \\
 &=
  \frac{1}{\alpha} \E\bigl[ X(T) \, \indc(B) \bigr] \\
 &\leq
  \frac{1}{\alpha} \E\bigl[ X(T)^+ \, \indc(B) \bigr] \\
 &\leq
  \frac{1}{\alpha} \E\bigl[ X(T)^+ \bigr] &
  \text{monotonicity.} 
\end{align*}

Since the sample paths are continuous by hypothesis,
the event 
\[
A = \Bigl\{ \max_{0 \leq t \leq T} X(t) \geq \alpha \Bigr\}
\]
can be expressed as an countably infinite intersection
of events of the form $B$ with finer and finer partitions $\{ t_i \}$
of the time interval $[0,T]$.
By taking limits, it follows
$\PP(A)$
has the same bound as the probabilities $\PP(B)$.
\end{proof}

\begin{cor*}
Let $X(t)$, for $0 \leq t \leq T$, be a square-integrable martingale
possessing continuous sample paths, whose
 unconditional mean is $m = \E[X(0)]$.
For any constant $\alpha > 0$,
\[
\PP \Bigl( \max_{0 \leq t \leq T} \abs{X(t)-m} \geq \alpha \Bigr)
\leq \frac{\Var[X(T)]}{\alpha^2}\,.
\]
\begin{proof}
Apply Kolmogorov's martingale inequality to $(X(t)-m)^2$,
which is a submartingale by Jensen's inequality.
\end{proof}
\end{cor*}

%%%%%
%%%%%
\end{document}
