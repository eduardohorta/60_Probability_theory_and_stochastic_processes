\documentclass[12pt]{article}
\usepackage{pmmeta}
\pmcanonicalname{ReverseMarkovInequality}
\pmcreated{2013-03-22 17:48:08}
\pmmodified{2013-03-22 17:48:08}
\pmowner{kshum}{5987}
\pmmodifier{kshum}{5987}
\pmtitle{reverse Markov inequality}
\pmrecord{6}{40264}
\pmprivacy{1}
\pmauthor{kshum}{5987}
\pmtype{Definition}
\pmcomment{trigger rebuild}
\pmclassification{msc}{60A99}
\pmrelated{MarkovsInequality}

% this is the default PlanetMath preamble.  as your knowledge
% of TeX increases, you will probably want to edit this, but
% it should be fine as is for beginners.

% almost certainly you want these
\usepackage{amssymb}
\usepackage{amsmath}
\usepackage{amsfonts}

% used for TeXing text within eps files
%\usepackage{psfrag}
% need this for including graphics (\includegraphics)
%\usepackage{graphicx}
% for neatly defining theorems and propositions
%\usepackage{amsthm}
% making logically defined graphics
%%%\usepackage{xypic}

% there are many more packages, add them here as you need them

% define commands here

\begin{document}
Let $X$ be a random variable that satisfies $\Pr(X \leq a) = 1$ for some constant $a$.
Then, for $d < E[X]$,
\[
\Pr( X > d) \geq \frac{E[X] - d}{a - d}
\]



{\em Proof:}
Apply the Markov's inequality to the random variable $\tilde{X} = a-X$,
\[\Pr(X\leq d) = \Pr(\tilde{X}\geq a-d) 
\leq \frac{E[\tilde{X}]}{a-d} 
= \frac{a-E[X]}{a-d}.
\]

Hence
\[
\Pr(X> d) \geq 1 - \frac{a-E[X]}{a-d} = \frac{E[X]-d}{a-d}.
\]
%%%%%
%%%%%
\end{document}
